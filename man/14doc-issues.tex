
\section Compatibility issues

1.\enspace Earlier versions of \jTree\ had the syntax in (\nextx b), rather
than the syntax (\nextx a) which has been explained and used in
this documentation.

\pex[interpartskip=!.5ex]
\a \vtop{\hsize=4.4in  \leftskip=0pt
|\jftree|\par
{\sl preliminary definitions. parameter settings\/}\par
|\! = |{\sl simple tree description}|.|\par
{\sl definitions, parameter settings, dimensionless graphics}\par
|\!a = |{\sl simple tree description}|.|\par
{\sl definitions, parameter settings, dimensionless graphics}\par
|\!b = |{\sl simple tree description}|.|\par
{\sl dimensionless graphics}\par
|\endjtree|\par}
\a \vtop{\hsize=4.4in  \leftskip=0pt
|\jftree|\par
{\sl preliminary definitions. parameter settings\/}\par
|\start |{\sl simple tree description}|.|\par
{\sl definitions, parameter settings, dimensionless graphics}\par
|\adjoin at !a |{\sl simple tree description}|.|\par
{\sl definitions, parameter settings, dimensionless graphics}\par
|\adjoin at !b |{\sl simple tree description}|.|\par
{\sl dimensionless graphics}\par
|\endjtree|\par}
\xe

{\it The old syntax can still be used, if desired.}  It is even
possible to mix the old and new syntax.

\medskip
2.\enspace Tex uses the control sequence |\!| for negative
spacing in math mode.  \jTree\ redefines |\!| inside |\jtree|
\dots|\endjtree|. This redefinition can be suppressed, if
desired, by redefining the token list |\jtEverytree|.  When
|\jtree| is invoked, two token lists are expanded and evaluated.
First, |\jtEverytree|\index*{+jtEverytree} and then
|\jteverytree|\index*{+jteverytree}. The first is intended for
setup use, to be rarely altered, and the second to be modified as
needed in the course of using \jTree. |\jteverytree| can be
modified by parameter setting.  The first cannot be, but can be
redefined by editing \pstjtree\/ or by resetting it in a setup
file that is loaded after {\sl pst-jtree\/} is loaded.  The
default setting is:
\medskip
\hfil|\jtEverytree={\let\!\adjoinop}|
\medskip
The |\!| will be suppressed if this is changed to:
\medskip
\hfil|\jtEverytree={}|
\medskip
With the special use of |\!| suppressed,
the syntax (\lastx b) must be used.\medskip
In my personal file, I have:
\medskip
\hfil|\jtEverytree={\everymath={\rm}\let\!\adjoinop}|
\medskip
In typesetting linguistic trees, I more often prefer roman
to math italic type in math mode.
\medskip
3.\enspace
Earlier versions of \jTree\ were incompatible with certain
packages which changed the character code of |"|, |<|, |>|, or
|^|.  Starting with version 2.4 (dated 2007/04/15), \jTree\ no
longer makes any assumptions about the character codes of these
characters.  The tree description parser does assume that
characters in the set $\{|.|,|:|,|(|,|)|,|@|,|!|\}$ are not
active and have the same category code that they had when
\pstjtree\/ was loaded.
Users with unsolved compatibility problems are encouraged
to report them to me at {\sl j.frampton@neu.edu\/}.




