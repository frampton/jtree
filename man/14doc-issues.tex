
\section Compatibility issues

1.\enspace Earlier versions of \jTree\ had the syntax in (\nextx b), rather
than the syntax (\nextx a) which has been explained and used in
this documentation.

\pex[interpartskip=!.5ex]
\a \vtop{\hsize=4.4in  \leftskip=0pt
|\jftree|\par
{\sl preliminary definitions. parameter settings\/}\par
|\! = |{\sl simple tree description}|.|\par
{\sl definitions, parameter settings, dimensionless graphics}\par
|\!a = |{\sl simple tree description}|.|\par
{\sl definitions, parameter settings, dimensionless graphics}\par
|\!b = |{\sl simple tree description}|.|\par
{\sl dimensionless graphics}\par
|\endjtree|\par}
\a \vtop{\hsize=4.4in  \leftskip=0pt
|\jftree|\par
{\sl preliminary definitions. parameter settings\/}\par
|\start |{\sl simple tree description}|.|\par
{\sl definitions, parameter settings, dimensionless graphics}\par
|\adjoin at !a |{\sl simple tree description}|.|\par
{\sl definitions, parameter settings, dimensionless graphics}\par
|\adjoin at !b |{\sl simple tree description}|.|\par
{\sl dimensionless graphics}\par
|\endjtree|\par}
\xe

{\it The old syntax can still be used, if desired.}  It is even
possible to mix the old and new syntax.

\medskip
2.\enspace Tex uses the control sequence |\!| for negative
spacing in math mode.  \jTree\ redefines |\!| inside |\jtree|
\dots|\endjtree|. This redefinition can be suppressed, if
desired, by redefining the token list |\jtEverytree|.  When
|\jtree| is invoked, two token lists are expanded and evaluated.
First, |\jtEverytree|\index*{+jtEverytree} and then
|\jteverytree|\index*{+jteverytree}. The first is intended for
setup use, to be rarely altered, and the second to be modified as
needed in the course of using \jTree. |\jteverytree| can be
modified by parameter setting.  The first cannot be, but can be
redefined by editing \pstjtree\/ or by resetting it in a setup
file that is loaded after {\sl pst-jtree\/} is loaded.  The
default setting is:
\medskip
\hfil|\jtEverytree={\let\!\adjoinop}|
\medskip
The |\!| will be suppressed if this is changed to:
\medskip
\hfil|\jtEverytree={}|
\medskip
With the special use of |\!| suppressed,
the syntax (\lastx b) must be used.\medskip
In my personal file, I have:
\medskip
\hfil|\jtEverytree={\everymath={\rm}\let\!\adjoinop}|
\medskip
In typesetting linguistic trees, I more often prefer roman
to math italic type in math mode.
\medskip
3.\enspace
Early versions of jTree assumed that the characters |^|, |>|,
|<|, and |"|, which all have special meaning to the jTree parser,
had their normal category codes when the macro |\jtree| was
invoked.  This made jTree incompatible with Babel, gb4e, and any
other macro packages which alter the category code of these
characters.  This incompatibility has now been removed.  Invoking
|\jtree| changes the category codes of these characters to the
normal values.  The change does not apply within labels, so the
Babel shortcuts can be used in labels. Users with unsolved
compatibility problems are encouraged to report them to me at
{\sl j.frampton@neu.edu\/}.




