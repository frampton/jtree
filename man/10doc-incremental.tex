
\section How to build complex trees

\ftag{\the\secno}[incremental]%
\jTree\ has several features which make it easy to build trees
incrementally.  In complex trees, this is a big advantage,
because the software does not make it easy to pinpoint the source
of errors in the code.  Consider, for example, the tree that is
used to illustrate the features of |qtree|, a popular
tree-drawing macro package (available on CTAN).

\ex
\def\GO#1#2#3{$\rm GO\bigl({#1},[_{Path}\,
   FROM({#2}),TO({#3})]\bigr)$}%
\jtree[xunit=3em,yunit=1em]
\defstuff[a]{\multiline
   \it The cup slid from John to Mary\cr
   \GO{\bf cup}{\bf John}{\bf Mary}\cr
   IP\endmultiline}
\defstuff[b]{\multiline
   \it The cup\cr
   \bf cup\cr
   NP\endmultiline}
\defstuff[c]{\multiline
   \it slid from John to Mary\cr
   \GO{\mit x}{\bf John}{\bf Mary}\cr
   IP\endmultiline}
\defstuff[d]{\multiline
   \it The\cr
   $\perp$\cr
  \quad N$\rm _{SPEC}$\endmultiline}
\defstuff[e]{\multiline
   \it cup\cr
   \bf cup\cr
   N\endmultiline}
\def\Fracture{<vert>{\omit\psframebox[framesep=.4ex]
   {Fracture$\vphantom j$}}}%
\! = {\stuff[a]}
   \Fracture
   :\jtbig{\stuff[b]}!a \jtbig{\stuff[c]}
   \Fracture
   :{$\vdots$} {$\vdots$}.
\!a = \Fracture
   :{\stuff[d]} {\stuff[e]}.
\endjtree
\xe
\ftagex[qtree]

We proceed, as usual, by starting with the right branching spine
of the tree, putting in adjunction points for complex material on
left branches that must be added.  Similar to adjunction points,
\jTree\ allows for tags to be inserted for ``stuff'' that can be
filled in later.  Macros |\stuff| and |\defstuff| are defined in
\pstjtree\/ and used as shown
below.\index*{+stuff}\index*{+defstuff}

\exh \tac|
\jtree
\def\fracture{\psframebox{Fracture}}
\! = {\stuff[a]}
   <vert>{\fracture}
   :\jtbig{\stuff[b]}!a
      \jtbig{\stuff[c]}
   <vert>{\fracture}
   :{$\vdots$} {$\vdots$}.
\endjtree
|endtac \hfil
\jtree
\def\fracture{\psframebox{Fracture}}
\! = {\stuff[a]}
   <vert>{\fracture}
   :\jtbig{\stuff[b]}!a \jtbig{\stuff[c]}
   <vert>{\fracture}
   :{$\vdots$} {$\vdots$}.
\endjtree
\xe

We then proceed to complete the structure at the tagged
positions, fill in the missing stuff, and to make whatever other
changes are needed to get a good looking tree.  Some things are
already apparent: there is too much whitespace under ``Fracture''
inside the box; the tree branches do not meet the ``fracture
box'', and the tree needs to be stretched in the x-direction.  We
delay filling in most of the missing stuff until the end, and
concentrate on fixing the problems noted above and
getting the tree structure right. This produces:

\exh \tac|
\jtree[xunit=3em,yunit=1em]
\def\Fracture{<vert>{\omit\psframebox[framesep=.4ex]
   {Fracture$\vphantom j$}}}%
\! = {\stuff[a]}
   \Fracture
   :\jtbig{\stuff[b]}!a
      \jtbig{\stuff[c]}
   \Fracture
   :{$\vdots$}
      {$\vdots$}.
\!a = \Fracture
   :{\stuff[d]}
      {\stuff[e]}.
\endjtree
|endtac \hfill \kern-30em
\jtree[xunit=3em,yunit=1em,treevshift=-3.5em]
\def\Fracture{<vert>{\omit\psframebox[framesep=.4ex]
   {Fracture$\vphantom j$}}}%
\! = {\stuff[a]}
   \Fracture
   :\jtbig{\stuff[b]}!a
      \jtbig{\stuff[c]}
   \Fracture
   :{$\vdots$}
      {$\vdots$}.
\!a = \Fracture
   :{\stuff[d]}
      {\stuff[e]}.
\endjtree
\kern1ex
\xe

We now are ready to fill in the stuff.  Since the go-path
construction is complicated, appears twice, and is probably
useful for this kind of work, it has been generalized to a macro.
\pstjtree\/ contains the |\multiline| \index*{+multiline}and
|\endmultiline| \index*{+endmultiline}macros to help construct
complex multiline labels.  |\multiline| starts a vbox and begins
|\halign{\hfil#\hfil\cr|. |\endmultiline| closes up the
construction.  Some care is taken with the vertical spacing. The
full code for (\lastx) is:
\medskip

\goodbreak
\codelines
|\def\GO#1#2#3{$\rm GO\bigl({#1},[_{Path}\,
   FROM({#2}),TO({#3})]\bigr)$}
|endcodelines

\def\Goodbreak#1{\goodbreak}
\medskip
\codelines
|\jtree[xunit=3em,yunit=1em]
\defstuff[a]{\multiline
   \it The cup slid from John to Mary\cr
   \GO{\bf cup}{\bf John}{\bf Mary}\cr
   IP\endmultiline}|Goodbreak|
\defstuff[b]{\multiline
   \it The cup\cr
   \bf cup\cr
   NP\endmultiline}
\defstuff[c]{\multiline
   \it slid from John to Mary\cr
   \GO{\mit x}{\bf John}{\bf Mary}\cr
   IP\endmultiline}
\defstuff[d]{\multiline
   \it The\cr
   $\perp$\cr
  \quad N$\rm _{SPEC}$\endmultiline}|Goodbreak|
\defstuff[e]{\multiline
   \it cup\cr
   \bf cup\cr
   N\endmultiline}
\def\Fracture{<vert>{\omit\psframebox[framesep=.4ex]
   {Fracture$\vphantom j$}}}%
\! = {\stuff[a]}
   \Fracture
   :\jtbig{\stuff[b]}!a \jtbig{\stuff[c]}
   \Fracture
   :{$\vdots$} {$\vdots$}.
\!a = \Fracture
   :{\stuff[d]} {\stuff[e]}.
\endjtree
|endcodelines

