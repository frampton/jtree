\global\pageno=1
\font\namefont=t1xr at 16pt
\font\namefontsl=t1xsl at 16pt

\leavevmode \vfil
\hfil\psscalebox{8}{\jTree}\par
\vskip-2ex
\hfil\psscalebox{3}{\quad for linguists}
\vskip4ex
\psscalebox{1.9}{Tex macros for typesetting complex trees}
\vskip4ex
\vfil\vfil
\hfil
\jtree[xunit=2.4em,yunit=1.2em,style=arrows2,nodesep=0,
   bbadjust=left .8em height 2ex]
\def\broken{[branch=\brokenbranch,scaleby=1.6]}%
\! = {\elc{\it a}}@A1
   <right>{\elc{\it b}}@A1a
   :{C$_2$}()  \broken @A2
   <right>@A3
   <right>@A3a
   :{C$_1$}()  \broken
   :{ubil}  {\elc{\it c}}@A4
   :{kolko}  {\elc{\it d}}
   :{studenti}  {\elc{\it e}}@A5
   :{ot} :{koi} {strani}.
\psLNode(A1)(A1a){.5}{K1}
\psLNode(A2)(A3){.5}{K2}
\psLNode(A3)(A3a){.5}{K3}
\psset{angleA={(1,1)},angleB=90,ncurvA=.6,ncurvB=1}
\nccurve{K1}{A5}
\nccurve{-}{K2}{A5}
\nccurve{K3}{A4}
\psset{angleA={(-1,-1)},angleB={(-1,-1)},ncurv=4,arrows=-}
\nccurve{A1}{K1}
\nccurve{A2}{K2}
\nccurve{A3}{K3}
\endjtree
%
%\jtree[xunit=2.4em,yunit=1.2em,nodesep=0,
%   arrowlength=1.4,arrowsize=2pt 3,arrowinset=.4]
%\def\broken{[branch=\brokenbranch,scaleby=1.6]}%
%\def\stub{<right>[scaleby=.5,arrows=-]}%
%\def\\#1{\rput[bl](.6ex,.4ex){\it #1}}%
%\def\\#1{}%
%\! = {\omit\\a}@A1
%   \stub @K1  ^<right>{\omit\\b}
%   :{C$_2$}[]  \broken @A2
%   \stub @K2  ^<right>@A3
%   \stub @K3  ^<right>
%   :{C$_1$}[]  \broken
%   :{ubil} {\omit\\c}@A4
%   :{kolko} {\omit\\d}
%   :{studenti} {\omit\\e}@A5
%   :{ot} :{koi} {strani}.
%\psset{dirA=(1:1),angleB=90,ncurvA=.6,ncurvB=1}
%\nccurve{->}{K1}{A5}
%\nccurve{K2}{A5}
%\nccurve{->}{K3}{A4}
%\psset{dirA=(-1:-1),dirB=(-1:-1),ncurv=4,arrows=-}
%\nccurve{A1}{K1}
%\nccurve{A2}{K2}
%\nccurve{A3}{K3}
%\endjtree
\vfil
\hfil\psscalebox{3}{User's Guide}
\vskip1ex

{\leftskip=0pt plus 1fil \parfillskip=0pt
\rightskip=0pt plus 1fil \openup4pt
\parindent=0pt \namefont
John Frampton\par
{\namefontsl j.frampton@neu.edu}\vskip14pt
20 November 2012\par
Version 2.7\par}


\vfil\break

