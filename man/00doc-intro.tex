\global\pageno=1
\parindent=2em

\section Introduction

Complex trees that linguists often need to display are difficult
to represent linearly in a fashion that allows a human to readily
grasp the intended hierarchical structure. This makes it
difficult to write Tex code in such a way that it can be easily
debugged or modified a year after it was first written, or that
portions of it can be copied for use in a different environment.
\jTree\ was designed to overcome these problems.

As usual, it is much easier to design a good tool for a task if
the task is fairly narrow. So my aim was to write a tree
formatter that would be particularly good for the kind of trees
that I most often encounter.  In the syntax that most interests
me, trees are often fairly deep, with a depth of embedding at
least~5. Example~\gettag[Richards] in \exsec\ has depth~19. Using
parentheses or brackets to encode embedding is not a viable
option if the Tex code is to have the desired characteristics. In
spite of being deep, however, the trees I usually want to typeset
are simple in some other ways; they are generally binary
branching and tend strongly to have their complex branching to
the right. \jTree\ is designed to be particularly good at
typesetting such trees.  Obviously, this represents a particular
point of view about syntactic theory.

\jTree\ is based on the widely used {\sl pstricks\/} macro
package. Many of the more advanced features of \jTree\ require a
willingness to learn a certain amount of PSTricks.  Linguists
will find much in PSTricks that is useful, even outside the
context of displaying and annotating trees. Most of the commands
take parameters which are defined and explained in the PSTricks
documentation.  For information about PSTricks, see {\sl
http://www.tug.org/applications/PSTricks/\/}. \jTree\
%http://www.tug.org/applications/PSTricks/\/}\wiggle[3em]. \jTree\
also requires the {\sl xkeyval\/} package, which has become the
standard mechanism for handling parameters in {\sl pstricks\/}
based packages.  See Appendix A for information on installing the
PSTricks, XKeyVal, and \jTree\ packages.

Some tree formatters attempt to make many decisions about
placement automatically, in order to make the formater easy and
foolproof to use.  The downside is that complex trees are
generally quite difficult to typeset because some effort is
required to undo the automatic help provided by the formatter.
The approach of \jTree\  is to do very little automatically, but
to make what is done very transparent and very easy to adjust.
Many features of tree construction in \jTree\ are controlled by
parameters, whose default settings usually suffice, but which are
available for fine-grained control when necessary. The core of
\jTree\ is a tree description language.  We begin by discussing
that language.

