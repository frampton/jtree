
\font\bigtt=txtt at 18pt
\def\\#1{\psframebox{\hbox to .75em{\hfil#1\hfil}%
   \vrule width0pt depth.3ex}}%
\def\Hat{\leavevmode\lower4pt\hbox{\bigtt\char`\^}}
\def\Period{\leavevmode\hbox{\bigtt.}}
\def\Langle{$\langle$}
\def\Rangle{$\rangle$}
\def\branch#1{$\langle #1\rangle$}
{\catcode`<=\active \catcode`>=\active
\gdef\markupmode{\catcode`<=\active
   \catcode`>=\active \let<=\Langle \let>=\Rangle
   \spaceskip=1.6ex}}

\section The \jTree\  description language (JTDL)

There are two components: 1.~a simple way to describe right
branching trees (which I take to be trees whose left branches do
not branch); and 2.~a mechanism for adjoining one tree onto
another.  Consider the following tree, for example:
\bigskip
\hfil
\jtree[xunit=2.2em,yunit=1em]
\def\:{:[scaleby=2.3]}%
\! = \:!a :{is} {rotten}.
\!a = :{the} :{cheese} :{that } \:!b :{ate} {\it t}.
\!b = :{the} :{rat} :{that } :{John \ } :{killed} {\it t}.
\endjtree
\bigskip

This tree can be described by a sequence of three right branching
trees and instructions for how to combine them.

\ex \vtop{\hsize=4in
\def\Bigskip{\vskip1.5em minus .2em}
\def\:{:[scaleby=2.3]}

\settabs\+ 2.\enspace adjoin at $P_1$:\qquad&\cr
\+1. & \jtree[baretopadjust=0]
\! = \:{\tenpoint$\mit P_1$}[labelgapt=-1pt] :{is} {rotten}.
\endjtree\cr

\Bigskip
\+ 2.\enspace adjoin at $P_1$:&
\jtree
\! = :{the} :{cheese} :{that }
   \:{\tenpoint$\mit P_2$}[labelgapt=-1pt]!b
   :{ate} {\it t}.
\endjtree\cr

\Bigskip
\+ 3.\enspace adjoin at $P_2$:&
\jtree
\! = :{the} :{rat} :{that }
   :{John \ } :{killed} {\it t}.
\endjtree\cr }\xe

We first address the question of describing right branching
trees, then a mechanism for combining them.

\subsection The description of right branching trees

Some trees are essentially a linear sequence of branches and
nodes.\par\nobreak

\ex \quad
a.\quad \jtree[labelgap=0,everylabel={}]
\! = {\\A}
<left>{\\B} <right>{\\C}
<right>{\\D} <right>{\\E}.
\endjtree
\hfil
b.\quad \jtree[labelgap=0,everylabel={}]
\! = {\\A}
<left>{\\B}{\\C}
<right>{\\D}.
\endjtree
\hfil
c.\quad
\jtree[labelgap=0,everylabel={}]
\! = {\\A} <left> <right> <left>{\\B}<right>{\\C}.
\endjtree
\xe

Such a tree can be described as a sequence of branches and
labels, each one simply attached to the one before it in the
obvious way.
\bigskip

\begingroup\leftskip=1.5em
a.\quad{\markupmode \\A <left> \\B
   <right> \\C <right> \\D  <right> \\E .}\medskip
b.\quad {\markupmode \\A <left> \\B \\C <right> \\D  .}\medskip
c.\quad {\markupmode \\A <left> <right> <left>
\\B <right> \\C .}\par\endgroup
\bigskip

If this language is extended by adding an operator
\Hat\ which shifts the attachment point to
the start of the previous branch, then right branching trees can
be described.  The trees (\nextx) are described by the sequences
(\anextx).
\bigskip

\ex a.\quad \jtree[labelgap=0,everylabel={}]
\! = {\\A}
:{\\B} {\\C}
:{\\D} {\\E}.
\endjtree
\hfil
b.\quad \jtree[labelgap=0,everylabel={}]
\! = {\\A}
<left>{\\B} ^<vert>{\\C} ^<right>{\\D}
<left>{\\E} ^<vert>{\\F} ^<right>{\\G}
<left>{\\H} ^<vert>{\\I} ^<right>{\\J}.
\endjtree
\xe

\def\A#1{\leavevmode\llap{\hbox to 1.5em{#1.\hfil}}\ignorespaces}

\vskip-1.5em
\pex
\a \vtopmarkup
\noindent \\A
<left> \\B \Hat\ <right> \\C
<left> \\D \Hat\ <right> \\E .
\endvtopmarkup
\a \vtopmarkup
\noindent \\A
<left> \\B \Hat\ <vert> \\C \Hat\ <right> \\D
<left> \\E \Hat\ <vert> \\F \Hat\ <right> \\G
<left> \\H \Hat\ <vert> \\I \Hat\ <right> \\J .
\endvtopmarkup
\xe

A parser for such descriptions needs to keep track of just two
positions.  At the start, \Pcurr\ and \Pprev\ are both
set to some default starting point.  After a label or branch
is parsed, \Pcurr\ is updated to be the current point.  When a
branch is parsed, \Pprev\ is updated to be its starting point.
When \Hat\ is parsed, \Pcurr\  is set to \Pprev.

\subsection Tree adjunction

This language can be extended by adding the notion of an adjunction
point, and the syntax for adjoining one tree to another at a
specified adjunction point. The decomposition (\nextx b) of the
tree (\nextx a) is described by (\nextx c).

\pex[interpartskip=1em]
\a \quad
\jtree[labelgap=0]
\! = {\\A}
   <wideleft>{\\B}(:{\\D}{\\E})
   ^<wideright>{\\C} :{\\F}{\\G}.
\endjtree
\medskip
\a \quad
\jtree[labelgap=0]
\! = {\\A}
   <wideleft>{\\B}@A1
      {\omit\qdisk(0,0){.4ex}}
      {\omit \rput(-1,-1.5){\rnode{A2}{!a}}}
   ^<wideright>{\\C} :{\\F}{\\G}.
\nccurve[angleA=0,angleB=-100,linewidth=.3pt,ncurvB=1]{->}{A2}{A1:b}
\endjtree
\qquad
\jtree[labelgap=0,everylabel={}]
\! = {adjoin at !a}[labelgapb=1ex]:{\\D}{\\E}.\endjtree
\a \vtopmarkup
\noindent \\A
   <wideleft> \\B !a \Hat\ <wideright> \\C
   <left> \\F \Hat\ <right> \\G .
\noindent !a $=$ <left> \\D \Hat\ <right> \\E .
\endvtopmarkup
\xe

When the parser encounters an adjunction point, it records
\Pcurr\ and names it so that it can be accessed in the future. In
the \jTree\ implementation of this tree description language, the
names of adjunction points, called !tags, are character strings
beginning with |!| and followed by a space.  The tree that is
described is a physical tree, not an abstract tree.  Branches are
objects which have dimensions and are identified by a name.  The
branch denoted by |<wideleft>| above, for example, has different
dimensions than the branch denoted by |<left>|.

\medskip
\jTree\ is based on this tree description language. Facilities
are provided for defining and naming branches of various kinds
and for drawing trees specified in JTDL format.  How a tree is
rendered will depend not only on its description, but on numerous
parameter settings, font choices, etc.  The Tex code for
rendering a tree has the form:
\medskip
{\leftskip=1.5em
|\jtree|\index*{+jtree}\par
{\sl preliminary definitions. parameter settings\/}\par
|\! = | {\sl simple tree description}|.|\par
{\sl definitions, parameter settings, dimensionless graphics}\par
|\!a = | {\sl simple tree description}|.|\par
{\sl definitions, parameter settings, dimensionless graphics}\par
|\!b = | {\sl simple tree description}|.|\par
{\sl dimensionless graphics}\par
|\endjtree|\index*{+endjtree}\par}
\medskip

When |\jtree| is executed, it establishes an adjunction point
named |!|.  The macro |\!|\index*{+!} is defined so that
executing |\!xxx = |sets \Pcurr\ and \Pprev\ to the point named
|!xxx| (|xxx| can be empty, so the point can be named |!|) and
parsing the tree description which follows is initiated,
terminating when a period is encountered. For example:

\exh \tac|
\jtree
\! = <left>{A}!a ^<right>{B}.
\!a = <left>{C}!b ^<right>{D}.
\!b = <left>{E} ^<right>{F}.
\endjtree
|endtac \hfil
\jtree
\! = <left>{A}!a ^<right>{B}.
\!a = <left>{C}!b ^<right>{D}.
\!b = <left>{E} ^<right>{F}.
\endjtree
\xe

The following code produces identical results.

\exh
\tac|
\jtree
\! = <left>{A}!a ^<right>{B}.
\!a = <left>{C}!a ^<right>{D}.
\!a = <left>{E} ^<right>{F}.
\endjtree
|endtac \hfil
\jtree
\! = <left>{A}!a ^<right>{B}.
\!a = <left>{C}!a ^<right>{D}.
\!a = <left>{E} ^<right>{F}.
\endjtree
\xe

The name |!a| created in the main tree, is no longer needed (in
this example) after the first subtree is initiated and can
therefore be reassigned.  It is even possible to say:

\exh
\tac|
\jtree
\! = <left>{A}! ^<right>{B}.
\! = <left>{C}! ^<right>{D}.
\! = <left>{E} ^<right>{F}.
\endjtree
|endtac \hfil
\jtree
\! = <left>{A}! ^<right>{B}.
\! = <left>{C}! ^<right>{D}.
\! = <left>{E} ^<right>{F}.
\endjtree
\xe

There is nothing special about the name |!| that |\jtree|
assigns other than it is assigned before tree  construction
commences.


\subsection The colon construction

The combination\enspace |<left>|{\sl label\/}|^<right>| occurs
frequently in the descriptions of binary trees. JTDL provides a
shortcut for this.  |:|{\sl label\/} is interpreted as
|<left>�label\/�^<right>|.  Using this:

\exh \tac|
\jtree
\! = {a}
   :{b} {c}
   :{d} {e}
   :{f} {g}
   :{h} {i}.
\endjtree
|endtac \hfil
\jtree
\! = {a}
   :{b} {c}
   :{d} {e}
   :{f} {g}
   :{h} {i}.
\endjtree
\xe

The formatting is for readability.  The following
code produces the same tree:
\medskip
\quad |\jtree\! ={a}:{b}{c}:{d}{e}:{f}{g}:{h}{i}.\endjtree|
\medskip
One more example:

\exh \tac|
\jtree
\! = :{a} :{b} :{c} :{d} {e}.
\endjtree
|endtac \hfil
\jtree
\! = :{a} :{b} :{c} :{d} {e}.
\endjtree
\xe

For expository reasons, the description above was oversimplified.
Actually, |:�\sl label\/�| is shorthand for
|<colonA>�label\/�^<colonB>|. \pstjtree\/ defines the
branches |<colonA>| and |<left>| identically, and the branches
|<colonB>| and |<right>| identically.  The user, however, is free
to redefine any of these branches.  For example.


\exh \tac|
\jtree
\defbranch<colonB>(1)(-.5)
\! = {a}
   :{b} {c}
   :{d} {e}
   :{f} {g}.
\endjtree
|endtac \quad \hfil
\jtree
\defbranch<colonB>(1)(-.5)
\! = {a}
   :{b} {c}
   :{d} {e}
   :{f} {g}.
\endjtree
\xe

A later section will give the details of how branches are defined.
Suffice it to say here that a branch is defined by giving its
height and slope.  Labels are recognized by the \jTree\ parser
as |{�stuff\/�}|, where the ``stuff'' is anything that can go
in a Tex hbox.  We also defer to a later section automatic
space that is inserted over and under labels and how this is
specified and adjusted.

\medskip
The example below uses the colon shortcut
extensively.\par\nobreak

\ex
\psset{xunit=2.2em,yunit=1em}
\tac|
\jtree
\defbranch<Left>(2.3)(1)
\! = <Left>!a ^<right> :{is} {rotten}.
\!a = :{the} :{cheese} :{that}
   <Left>!b ^<right> :{ate} {\it t}.
\!b = :{the} :{rat} :{that}
   <Left>!c ^<right> :{killed} {\it t}.
\!c = :{the} :{cat} :{that} :{John} :{owned} {\it t}.
\endjtree
|endtac
\bigskip
\hfil\jtree
\defbranch<Left>(2.3)(1)
\! = <Left>!a ^<right> :{is} {rotten}.
\!a = :{the} :{cheese} :{that}
   <Left>!b ^<right> :{ate} {\it t}.
\!b = :{the} :{rat} :{that}
   <Left>!c ^<right> :{killed} {\it t}.
\!c = :{the} :{cat} :{that} :{John} :{owned} {\it t}.
\endjtree

\xe
\ftagex[cheese]\ftagpage[cheesepage]

\subsection Inline adjunction

In addition to the |:| operator, \jTree\ provides one other
shortcut.  It provides an alternative to adjunction in the code
below.

\exh \tac|
\jtree[xunit=2.8em,yunit=1em]
\! = {S}
   :{NP}!a {VP}
   :{V}!b {NP}
   <vert>{Tom}.
\!a = <vert>{Mary}.
\!b = <vert>{saw}.
\endjtree
|endtac \hfil
\jtree[xunit=2.8em,yunit=1em]
\! = {S}
   :{NP}!a {VP}
   :{V}!b {NP}
   <vert>{Tom}.
\!a = <vert>{Mary}.
\!b = <vert>{saw}.
\endjtree
\xe

\goodbreak
This can be written:
\bigskip
\qquad \tac|
\jtree[xunit=2.8em,yunit=1em]
\! = {S}
   :{NP}(<vert>{Mary})  {VP}
   :{V}(<vert>{saw})  {NP}<vert>{Tom}.
\endjtree
|endtac
\bigskip
Effectively, interpreting |(�$\,\ldots\,$�)| carries out {\it
inline adjunction}, without the need to explicitly name the
adjunction point.  In practice, description by inline adjunction
should be avoided for all but very simple nonterminals (as in
this example) since it can quickly lead to opaque code and the
hornet's nest of parentheses that JTDL was designed to avoid.




