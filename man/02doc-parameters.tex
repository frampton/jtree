
\section Parameters

Many \jTree\ items take which control various aspects of
typesetting.  \pstjtree\/ contains the definitions:
\cframe|\defbranch<left>(1)(1)
\defbranch<right>(1)(-1)
|endcframe
(Quotations from {\sl pst-jtree\/} will be displayed in a box,
as above.)
This defines |<left>| to be a branch with height 1 psyunit and
slope is 1, and |<right>| to be a branch with height 1 psyunit
and slope $-1$.  The specification of branches will be discussed
in more detail in Section~4, but for now it is sufficient to know
that |<left>| and |<right>| are drawn as shown below:

\ex\quad
\psset{unit=2cm}
\pspicture[shift=-1.5](0,-.55)(1.3,1.25)%\psgrid
\psline[style=dotted](0,0)(1,0)(1,1)
\psline(1,1)(0,0)
\rput(.5,-.8em){\tenpoint 1 psxunit}
\rput(1,.5){\rput{90}(.8em,0){\tenpoint 1 psyunit}}
\rput(.5,-2.2em){|<left>|}
\endpspicture
\hskip4em
\pspicture[shift=-1.5](0,-.55)(1.3,1.25)%\psgrid
\psline[style=dotted](0,1)(1,1)(1,0)
\psline(0,1)(1,0)
\rput(.5,1){\rput(0,.8em){\tenpoint 1 psxunit}}
\rput(1,.5){\rput{90}(.8em,0){\tenpoint 1 psyunit}}
\rput(.5,-2.2em){|<right>|}
\endpspicture
\xe

The fact that the dimensions of branches can be specified in
psunits, with the xunits independent of the yunits, coupled with
the fact that branches are rendered in physical (Tex) units,
produces a great deal of flexibility.  The three trees below were
produced with exactly the same \jTree\ code, but with
|\psset{unit=1em}|, |psset{unit=2em}|, and
|psset{xunit=2em,yunit=1em}|.

\ex
\psset{unit=1em}
a.\quad \jtree
\! = {VP} <left>{Bill} ^<right>{V$'$}
<left>{saw} ^<right>{Mary}.
\endjtree
\psset{unit=2em}
\hfil b. \jtree
\! = {VP} <left>{Bill} ^<right>{V$'$}
<left>{saw} ^<right>{Mary}.
\endjtree
\psset{xunit=2em,yunit=1em}
\hfil c. \jtree
\! = {VP} <left>{Bill} ^<right>{V$'$}
<left>{saw} ^<right>{Mary}.
\endjtree
\xe

|\jtree| accepts parameters directly, so you can say:

\exh \tac|
\jtree[xunit=1.5em,yunit=1em]
\! = {VP}
   <left>{Bill} ^<right>{V$'$}
   <left>{saw} ^<right>{Mary}.
\endjtree
|endtac
\hfil
\jtree[xunit=1.5em,yunit=1em]
\! = {VP}
   <left>{Bill} ^<right>{V$'$}
   <left>{saw} ^<right>{Mary}.
\endjtree
\xe

Individual branches also take parameters, so you can say:

\exh \tac|
\jtree[xunit=1.5em,yunit=1em]
\! = {VP}
   <left>{Bill} ^<right>[xunit=3em]{V$'$}
   <left>{saw} ^<right>{Mary}.
\endjtree
|endtac \hfill
\jtree[xunit=1.5em,yunit=1em]
\! = {VP}
   <left>{Bill} ^<right>[xunit=3em]{V$'$}
   <left>{saw} ^<right>{Mary}.
\endjtree\kern2em
\xe

The parameters |unit|, |xunit|, and |yunit| are defined by
PSTricks.  \jTree\ defines some of its own parameters.  A
parameter |scaleby| \index*{scaleby}is provided.
|\psset{scaleby=�x\/� �y�}| causes branches to be drawn
as if
\medskip\hfil
|\psset{xunit=�x\/�\psxunit,yunit=�y\/�\psyunit}|
\medskip
had been executed (but the psunits are not actually changed).

\medskip
|\psset{scaleby=�x\/�}| is equivalent to
|\psset{scaleby=�x\/� �x\/�}|.
\medskip
The following are then possible:

\exh
\tac|
\jtree[xunit=1.5em,yunit=1em]
\! = {VP}
   :[scaleby=2]{Bill} {V$'$}
   :{saw} {Mary}.
\endjtree
|endtac \hfill
\jtree[xunit=1.5em,yunit=1em]
\! = {VP}
   :[scaleby=2]{Bill} {V$'$}
   :{saw} {Mary}.
\endjtree
\kern1ex
\bigskip
\tac|
\jtree[xunit=1.5em,yunit=1em]
\! = {VP}
   <left>{Bill} ^<right>[scaleby=2 1]{V$'$}
   :{saw} {Mary}.
\endjtree
|endtac \hfill
\jtree[xunit=1.5em,yunit=1em]
\! = {VP}
   <left>{Bill} ^<right>[scaleby=2 1]{V$'$}
   :{saw} {Mary}.
\endjtree
\kern1ex
\xe

\jTree\ defines a dozen or so parameters in all.  They can be
set by |\psset|, but more importantly, can be attached to
particular instances of |\jtree| and the branches and labels that
appear inside |\jtree| environments.

\medskip
Modifying branches by standard PSTricks parameters is often
useful.\par\nobreak
\exh
\tac|
\jtree[xunit=3em,yunit=3em]
\! = <left>[scaleby=2 1,
            arrows=->]{a}
   ^<left>[linewidth=2pt]{b}
   ^<right>[scaleby=1 2,
      linestyle=dashed]{c}
   ^<right>[scaleby=1.5 1,
      linestyle=dotted,linewidth=1pt]{d}
   ^<right>[scaleby=2 .5,doubleline=true]{e}.
\endjtree
|endtac \kern-4em\hfill
\jtree[xunit=3em,yunit=3em]
\! = <left>[scaleby=2 1,
            arrows=->]{a}
   ^<left>[linewidth=2pt]{b}
   ^<right>[scaleby=1 2,
      linestyle=dashed]{c}
   ^<right>[scaleby=1.5 1,
      linestyle=dotted,linewidth=1pt]{d}
   ^<right>[scaleby=2 .5,doubleline=true]{e}.
\endjtree
\kern1em
\xe

The PSTricks documentation can be consulted for many other line
and arrow parameters which offer further control over how
branches are drawn.

