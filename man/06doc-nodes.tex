
\section @tags

\ftag{\the\secno}[nodesec]%
\jTree\ includes direct support for PSTricks' powerful \pstnode\/
package. Even in linear structures, \pstnode\/ is very useful to
linguists. The ability to define nodes and to connect them with
arrowed curves of various kinds allows the user to easily code
things like:

\exh
\tac|
\rnode{A1}{Who} \rnode{B1}{did}
   Jack \rnode{B2}{\sl t\/}
   see \rnode{A2}{\sl t\/}?}
\psset{linearc=2pt}
\ncbar[angle=-90]{->}{A2}{A1}
\ncbar[angle=90]{->}{B2}{B1}
|endtac \hfil
\lower1em\hbox{\rnode{A1}{Who} \rnode{B1}{did}
   Jack \rnode{B2}{\sl t\/}
   see \rnode{A2}{\sl t\/}?}%
\psset{linearc=2pt}
\ncbar[angle=-90]{->}{A2}{A1}
\ncbar[angle=90]{->}{B2}{B1}
\xe

In tree structures, the \pstnode\/ macro |\nccurve| is
particularly useful. {\bf Careful}: Nodes refer to named
structures defined by \pstnode\/ commands, not to tree nodes.

\exh
\tac|
\jtree
\! = :{\rnode{A1}{who}}
   :{\rnode{B1}{did}} :{Jack}
   :{\rnode{B2}{\sl t}} :{see}
   {\rnode{A2}{\sl t}}.
\psset{arrows=->,angleA=-150,
   angleB=-90}
\nccurve{A2}{A1}
\nccurve{B2}{B1}
\endjtree
|endtac \hfil
\jtree
\! = :{\rnode{A1}{who}}
   :{\rnode{B1}{did}} :{Jack}
   :{\rnode{B2}{\sl t}} :{see}
   {\rnode{A2}{\sl t}}.
\psset{arrows=->,angleA=-150,
   angleB=-90}
\nccurve{A2}{A1}
\nccurve{B2}{B1}
\endjtree
\xe

\jTree\ facilitates the use of \pstnode\/ by means of what are
called @tags, as illustrated below.

\exh
\tac|
\jtree
\! = :{who}@A1 :{did}@B1 :{Jack}
   :{\sl t}@B2 :{see} {\sl t}@A2 .
\psset{arrows=->,angleA=-150,
   angleB=-90}
\nccurve{A2}{A1}
\nccurve{B2}{B1}
\endjtree
|endtac  \hfil
\jtree
\! = :{who}@A1 :{did}@B1 :{Jack}
   :{\sl t}@B2 :{see} {\sl t}@A2 .
\psset{arrows=->,angleA=-150,
   angleB=-90}
\nccurve{A2}{A1}
\nccurve{B2}{B1}
\endjtree
\xe

A \jTree\ @tag consists of |@| followed by any sequence of
characters that is a valid PSTricks node name, followed by a
space.  If a @tag follows a label (parameters are part of the
label), then three nodes are created. If the @tag is |@P2|, for
example, point nodes named |P2:t|, |P2|, and |P2:b| are created.
|P2:t| is at the point that was current before the label was
added to the structure, and |P2:b| is at the point that becomes
the current point after the label is added to the structure. |P2|
is a box node consisting of the label box, with the reference
point at its center.  With respect to |P2|, |{�stuff\/�}@P2 |is
equivalent to |{\rnode{P2}{�\sl stuff\/�}}|. If a @tag does not
follow a label, then a single point node is created at the
current position~\Pcurr.

\medskip
These ideas are illustrated below:\par\nobreak

\exh \tac|
\jtree
\! = @AA
   <vert>{A}
   <vert>{B}[labelgapt=12pt]@P2
   <vert>{C}.
\psset{arrows=->,angleA=0,angleB=0,
   ncurv=1.4}
\nccurve[nodesep=0]{AA}{P2:t}
\nccurve[nodesepA=0]{AA}{P2}
\nccurve[nodesep=0,ncurv=1.6]{AA}{P2:b}
\endjtree
|endtac \hfil
\jtree
\! = @AA
   <vert>{A}
   <vert>{B}[labelgapt=12pt]@P2
   <vert>{C}.
\psset{arrows=->,angleA=0,angleB=0,
   ncurv=1.4}
\nccurve[nodesep=0]{AA}{P2:t}
\nccurve[nodesepA=0]{AA}{P2}
\nccurve[nodesep=0,ncurv=1.6]{AA}{P2:b}
\endjtree

\medskip
\hrule

\bigskip
\tac|
\jtree[scaleby=1 2,labelgap=1.2ex]
\! = :{and}@A
   {whatever}[labeloffset=1em]@AA .
\psset{linestyle=dashed,arrows=->}
\ncline{A}{AA:t}
\ncline{A}{AA:b}
\nccurve[angleA=-90,angleB=-90,ncurv=1]{A}{AA}
\endjtree
|endtac \kern-2em\hfil
\jtree[scaleby=1 2,labelgap=1.2ex]
\! = :{and}@A
   {whatever}[labeloffset=1em]@AA .
\psset{linestyle=dashed,arrows=->}
\ncline[nodesepB=0]{A}{AA:t}
\ncline[nodesepB=0]{A}{AA:b}
\nccurve[angleA=-90,angleB=-90,ncurv=1]{A}{AA}
\endjtree
\xe

There is some further information about nodes and the use of
|\nccurve| in Section~\gettag[nodesandhow].

