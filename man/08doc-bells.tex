\section Bells and whistles

\vskip-1ex
\subsection {\sl baretopadjust}

The baseline of the box created by |\jtree...\endjtree| is
normally the baseline of the root label.  If, however, the root
label is empty, that puts the baseline too low, at least for my
taste.  Contrast the trees below:
\bigskip
\qquad a.\quad \jtree \start {A} :{B} {C}.\endjtree
\hfil b.\quad \jtree[baretopadjust=0] \start :{B} {C}.\endjtree
\bigskip
\jTree\  takes corrective action by raising the tree diagram if
the root label is empty by the amount specified by the parameter
\index*{baretopadjust}|baretopadjust|.  The default setting is
1.4ex, but you can set it to whatever you want.  With the default
setting:
\bigskip
\qquad a.\quad \jtree \start {A} :{B} {C}.\endjtree
\hfil b.\quad \jtree[baretopadjust=1.4ex] \start :{B} {C}.\endjtree

\subsection {\sl treevshift}

\jTree\ also provides the parameter
\index*{treevshift}|treevshift|, which is set to 0 by default. It
is independent of |baretopadjust| and can be used to move the
tree diagram up and down, if desired.

\cd \leavevmode\raise1.2em\hbox{\tac|
$\to$\qquad
\jtree[treevshift=1.2em]
\! = {A} :{B} {C}.
\endjtree
|endtac}\hfil
$\to$\qquad
\jtree[treevshift=1.2em]
\! = {A} :{B} {C}.\endjtree
\dc

\subsection {\sl \index{everytree}}

If you say, for example,
|\psset{everytree={\psset{style=treestyle}}}|, and if you have
defined |treestyle| using PSTricks style definition, then
your trees will done in that style.  Setting |everytree| sets a
token list |\jteverytree| which is inserted at the beginning of
every |\jtree|\dots|\endjtree| construction.  It is grouped, so
that its scope is limited to the tree construction.
\smallskip
If you want, you can set |\jteverytree| directly.  The same is
true, by the way, for |everylabel|, which uses the token list
|\jteverylabel|.

\subsection \index{Custom branches}

\jTree\ ordinarily draws branches using the PSTricks macro
|\psline|.  It does this in an indirect way.  It actually calls
the macro |\branch@type| to do the drawing; and \jTree\ contains
the line |\let\branch@type=\psline|.  If the user says
|\psset{branch=\customline}|, then |\branch@type| is made
|\customline| instead of |\psline| with the usual locality. If
|\customline| is suitably defined then it will be
used to draw branches.  A few alternative macros for drawing
branches are included in \jTree.  The enterprising user might
want to define appropriate zigzag or coil macros, following the
PSTricks model.  \jTree\ provides |\blank|, |\brokenbranch|, and
|\etcbranch|.\index*{branch}\index*{+blank}\index*{+brokenbranch}%,
\index*{+etcbranch}

\cd \tac|
\jtree[xunit=3em,yunit=2em]
\! = {root}
<left>[branch=\blank]{A}
^<right>[branch=\brokenbranch]{B}
<right>[branch=\etcbranch].
\endjtree
|endtac \qquad\hfil
\jtree[xunit=3em,yunit=2em] \start {root}
<left>[branch=\blank]{A} ^<right>[branch=\brokenbranch]{B}
<right>[branch=\etcbranch].
\endjtree
\dc

The proportion of the ``etc branch'' that is dotted is controlled
by a parameter index*{etcratio}|etcratio|.  \jTree\ says |\psset{etcratio=.75}|,
but the user is free to change this if desired.  The style for
the dotted portion is defined by
\medskip
\quad|\newpsstyle{etc}{nodesepB=0,nodesepA=1pt,linestyle=dotted,|\par
\qquad |linewidth=1.2pt,dotsep=2pt}|
\medskip
The user can change (i.e. overwrite) this, if desired.

\jTree\ has the macro definition:
\medskip
\quad |\def\etc{[branch=\etcbranch,scaleby=.7]}|
\medskip
So you can write:

\cd \tac|
\jtree \! = :{A} :{B}
   <left>{C} ^<right>\etc.
\endjtree
|endtac \hfil
\jtree \! = :{A} :{B} <left>{C} ^<right>\etc.
\endjtree
\dc


\subsection The pseudo-parameters {\sl dirA\/} and {\sl dirB}

\everymath={}%
%
\index*{dirA, dirB}%
Suppose, for example, that you want to use |\nccurve| to draw a
curve which leaves the point A in a direction towards the point
which is displaced from A by $(1,-2)$ (ps-units).  You need to
set |angleA| to the appropriate value.  This is complicated,
because the x-scale and the y-scale may be different.  The
pseudo-parameter |dirA| is provided to accomplish this.
\hbox{|\psset{dirA=(1:-2)}|} will cause |angleA| to be set to the
appropriate value.  It is called a pseudo-parameter (my
terminology) because
\hbox{|\psset{dirA=(1:-2)}|} is executed for its effect on the parameter
|angleA|, not |dirA|.  Note the use of the colon, which keeps
|\psset| from getting confused.
\smallskip
Use of |dirA| below ensures that the ``curved branch'' lines up
with the straight ones.  Note that the curve is made fairly stiff
(a high value of |ncurv|) so that it bows out sufficiently.

\cd \tac|
\jtree
\! = @A1 <right> :{B} :{C} {D}@A2 .
\nccurve[dirA=(-1:-1),angleB=200,
   ncurv=2,nodesepB=.5ex]{A1}{A2}
\endjtree
|endtac \hfill
\jtree \! = @A1 <right> :{B} :{C} {D}@A2 .
\nccurve[dirA=(-1:-1),angleB=200,
   ncurv=2,nodesepB=.5ex]{A1}{A2}
\endjtree\kern1em
\dc

Of course, |dirB| works the same way for |angleB|.

\def\andthen{\kern1pt:\enspace}

\endinput

\subsection Complete list of \jTree\ parameters

|labelgapt|, |labelgapb|, |labelgap|\andthen The center of the
top of the label box is placed a distance |labelgapt| below the
previous termination point.  The termination point of the label
box is placed a distance |labelgapb| below the bottom edge of the
label box. |labelgap| is a pseudo-parameter, used for its effect.
|\psset{labelgap=�x\/�}| has the effect
|\psset{labelgapt=�x\/�,labelgapb=�x\/�}|.

\medskip
|scaleby|\andthen |\psset{scaleby=�x\enspace y\/�}| has the effect
of scaling all branches, triangles, and vartriangles by {\sl x\/}
horizontally and {\sl y\/} vertically.  If {\sl y\/} is omitted,
scaling by {\sl x\/} in both directions is carried out.

\medskip
|everytree|\andthen
The effect of
|\psset{everytree=�$\alpha$�}| is
|\jteverytree={�$\alpha$�}|.  This token list is inserted every
time that |\jtree| is invoked.

\medskip
|everylabel|\andthen
The effect of
|\psset{everylabel=�$\alpha$�}| is
|\jteverylabel={�$\alpha$�}|.  This token list is inserted every
time that a label box is created, unless cancelled by |\omit|.

\medskip
|treevshift|, |baretopadjust|\thinspace:\enspace These parameters
control the relation between the baseline of the assembled tree
and the baseline of the root label.  The baseline of the root
label is a distance $\sl d=v+b\mskip2mu$ above the baseline of
the tree, where $\sl v$ is |treevshift|; and $\sl b$ is
|baretopadjust| if the root label has 0 height, otherwise is~0.

\medskip
|triratio|\andthen This determines the termination point of
triangles, and the skew of vartriangles.  See Chapter~5 for
discussion.

\medskip
|branch|\andthen The control sequence |\branch@type| is used to
draw the connections in branches.  {\sl pst-jtree\/} contains
|\let\branch@type=\psline|.  The effect of
|\psset{branch=\usermacro}| is |\let\branch@type=\usermacro|. It
is assumed that |\usermacro| has the argument structure\smallskip
\hfil|\def\usermacro(#1,#2)(#3,#4){|\dots
\smallskip

\medskip
|etcratio|\andthen {\sl pst-jtree\/} provides the macro
|\etcbranch| as an alternative to |\psline|.  It draws a line
which begins as a solid line but ends as a dotted line.  Its use
is to indicate an elided tree tail.  |etcratio| determines what
fraction of the line is dotted.

\medskip
|dirA|, |dirB|\andthen These are pseudo-parameters, used to set
the parameters |angleA| and |angleB|, respectively.
|\psset{dirA=(�a\/�:�b\/�)}| sets angleA so that
the direction corresponding to angleA is the same as the
direction corresponding to the vector $\sl\langle a,b\rangle$.
Setting |dirB| has a similar effect on |angleB|.




