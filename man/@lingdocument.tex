
\def\fileversion{03.4}
\def\filedate{2004/06/26}
\message{ v\fileversion, \filedate}

\lingmkletter @

%%%%%%%%%
%% \ex, \pex
%%%%%%%%%
\newcount\exno \exno=0
\def\smallpenaltypar{\endgraf
   \vskip \ling@exsplitvfil
   \expandafter\penalty\ling@exgoodparpenalty
   \vskip-\ling@exsplitvfil}
\def\beginex{\par\penalty50 \prevdepth30pt
   \vskip\ling@beforeexskip
   \vskip\ling@exsplitvfil
      \expandafter\penalty\ling@exbeginsplitpenalty
      \vskip-\ling@exsplitvfil
   \global\advance\exno by 1
   \hsize=\ling@exhsize
   \examplebaselines
   \let\goodpar=\smallpenaltypar
   \interlinepenalty=\ling@exparpenalty
   \parindent=0pt}

\edef\ling@letterwd{.8em}

\newtoks\everyex \everyex={}
\newcount\pexpartcount
\newif\ifling@textadjust
\newif\ifling@firstpart

\def\ex{\ling@getpars\ex@A}
\def\ex@A{\bgroup
   \expandafter\Lingsetkeys\expandafter{\pst@pars}
   \beginex
   \the\everyex
   \setbox0=\hbox{\hskip\ling@numoffset
      (\rm\the\exno)\hskip\ling@exoffset}%
   \leftskip=\wd0
   \leavevmode\llap{\unhbox0}\ignorespaces}
\def\pex{\@ifnextchar*\pex@a\pex@b}
\def\pex@a#1{\ling@textadjusttrue\pex@c}
\def\pex@b{\ling@textadjustfalse\pex@c}
\def\pex@c{\ling@getpars\pex@d}
\def\pex@d{\bgroup
   \expandafter\Lingsetkeys\expandafter{\pst@pars}%
   \beginex
   \ifling@pexnumbers \pexpartcount=0 \let\a=\ling@pexnumber
      \else \pexpartcount=96 \let\a=\ling@pexchar \fi
   \ling@firstparttrue
   \the\everyex
   \setbox0=\hbox{\hskip\ling@numoffset
      (\rm\the\exno)\hskip\ling@letteroffset\hskip\ling@letterwd
         \hskip\ling@exoffset}%
   \leftskip=\wd0
   \leavevmode\llap{\unhbox0}%
   \ifling@textadjust\hskip-\ling@pextextadjust\fi \ignorespaces}
\def\xe{\vskip\ling@afterexskip
   \nointerlineskip\allowbreak \egroup }
\def\ling@pexchar{\advance\pexpartcount by 1
   \ifling@firstpart\ling@firstpartfalse\else
      \par\vskip\ling@interpartskip
      \fi
   \leavevmode
   \llap{\char\the\pexpartcount.\hskip\ling@exoffset}}
\def\ling@pexnumber{\advance\pexpartcount by 1
   \ifling@firstpart\ling@firstpartfalse\else
      \par\vskip\ling@interpartskip
      \fi
   \leavevmode
   \llap{\the\pexpartcount.\hskip\ling@exoffset}}
% -- glosses -----------
\newskip\glossspaceskip  \glossspaceskip=1.5ex
\newtoks\everyentry
\def\glossit{\noalign{\global\everyentry={\it}}}
\def\gloss{\vtop\bgroup \everyentry={}%
   \halign\bgroup \the\everyentry ##\hfil &&
      \the\everyentry\kern\glossspaceskip ##\hfil\cr}
\def\endgloss{\strut\crcr\egroup\egroup\par\prevdepth=.5ex}
\catcode`|=\active %confine|
\def\glossstrut{\vrule height1.9ex depth.7ex width0pt}
\def\@glossline@AA{\ling@getpars\@glossline@BB}
\def\@glossline@BB{%
   \ifx\pst@pars\@empty
      \global\everyentry={}%
   \else
      \expandafter\global\expandafter
         \everyentry\expandafter{\pst@pars}%
   \fi
   \@glossline@a
}
\def\@glossline@a#1 {\the\everyentry #1\@glossline@b}
\def\@glossline@b{%
   \@ifnextchar| %confine|
      {\glossstrut\gloss@gobble}%
      {\@glossline@c}%
}
\def\@glossline@c#1 {& \the\everyentry #1\@glossline@b}
\def\gloss@gobble#1#2{%
   \ifx#2^%
      \def\next{\cr\egroup\egroup\ignorespaces}%
   \else
      \def\next{\cr#2}%
   \fi
   \next
}
\def\igloss@a{%
   \let|=\@glossline@AA %confine|
   \halign\bgroup ##\hfil && \kern\glossspaceskip ##\hfil\cr
}
\gdef\igloss{%
   \leavevmode
   \vtop\bgroup
      \everyentry={}%
      \offinterlineskip
      \catcode`\|=\active % confine|
      \igloss@a
}
\def\|{\igloss \@glossline@AA} %confine|
\catcode`|=12  %confine|
%%%%%%%%
%% headline
%%%%%%%%
%%%%
\newbox\HLbox
% headlines
\def\mkheadline{\vtop to 0pt{\vss
   \hbox to\hsize{\spaceskip=0pt \tenrm \ling@headline\hfil}%
   \vskip\ling@headlinelevel}}
%   \hbox to\hsize{\ling@headline}\vskip\ling@headlinelevel}}

\def\mkheadline{\rput[Bl](0,\ling@headlinelevel)%
   {\hbox to \ling@normalhsize{\spaceskip=0pt \ling@headline}}}
%% redefine \plainoutput to include PSTricks commands and
%%   marginal notes
\newtoks\psteverypage   % pst commands only
\def\jfoutput{\shipout\vbox{%
   \hbox to0pt{\the\psteverypage\hss}\pagebody}
   \advancepageno
   \ifnum\outputpenalty>-20000 \else\dosupereject\fi}
\let\plainoutput=\jfoutput
\psteverypage={\mkheadline}

\def\dump{\ifling@printnotes \dumpnotes \fi
   \ifling@printrefs \dumprefs \fi}

%%%%%%%%%%%
%% sections
%%%%%%%%%%%

\newcount\secno            \secno=0
\newcount\subsecno
\newcount\subsubsecno
\newcount\subsubsubsecno

\outer\def\section #1\par{%
   \global\advance\secno by 1  \global\subsecno=0
      \global\subsubsecno=0
   \vskip0pt plus.1\vsize\penalty0
   \vskip0pt plus-.1\vsize\bigskip\vskip\parskip
   {\noindent \titlefont{sec}
       \the\secno.\enskip #1\par}\nobreak
   \medskip\nobreak\noindent}

\outer\def\subsection #1 \par{%
   \global\advance \subsecno by 1 \global\subsubsecno=0
   \vskip0pt plus.1\vsize\penalty0
   \vskip0pt plus-.1\vsize\bigskip\vskip\parskip
   {\noindent \titlefont{sub}
      \the\secno .\the\subsecno. \enskip #1}\par\nobreak
   \medskip\nobreak\noindent}

\outer\def\subsubsection #1 \par{%
   \global\advance \subsubsecno by 1 \global\subsubsubsecno=0
   \vskip0pt plus.1\vsize\penalty0
   \vskip0pt plus-.1\vsize\bigskip\vskip\parskip
   {\noindent \titlefont{subsub}
      \the\secno .\the\subsecno.\the\subsubsecno \enskip #1}\par\nobreak
   \medskip\nobreak\noindent}

\def\subsubsubsection #1 \par{\advance\subsubsubsecno by 1
   \vskip0pt plus.1\vsize\penalty0
   \vskip0pt plus-.1\vsize\medskip
   \noindent {\it \the\secno.\the\subsecno.\the\subsubsecno
      .\the\subsubsubsecno\enspace #1}:\hskip.5em}

\def\bulletsection #1\par{%
   \vskip0pt plus.1\vsize\penalty0
   \vskip0pt plus-.1\vsize\medskip
   \noindent $\bullet$\enspace
      {\it #1}\par\nobreak\medskip\nobreak\noindent}

%%%%%%%%%%%%%%%%%%%%%%%%%%%%%%%%%%%%%%%%%%%%%%%%%
% ----- disclaimer, title, name ------

\def\title{%
   \bgroup
   \leftskip=0pt plus 1fil \rightskip=0pt plus 1fil
   \parindent=0pt \parfillskip=0pt
   \baselineskip=\ifling@doublespace 22\else 13\fi pt
   \titlefont{main}}
\def\author{\vskip.7em
   \baselineskip=\ifling@doublespace 14\else 13\fi pt
   \titlefont{subsub}}
\def\endtitle{\vskip1.6em \egroup}
\def\disclaimer{\leavevmode
   \bgroup \parindent=0pt
   \baselineskip=13pt
   \vskip-.8truein
   \hrule
   \medskip}
\def\enddisclaimer{\medskip\egroup
   \hrule
   \vskip2.5em plus 2em minus .2em}

% ------  footnotes ------

\newcount\fnno
\fnno=0

\def\raisedasterisk{\raise3pt\rlap{\twelverm *}}
\def\ackmark{$\,{}^1$ \global\fnno=1}
\def\putfnno{\hbox{\global\advance\fnno by 1 $^{\the\fnno}$}}

\long\def\acknowledgement#1{\global\advance\fnno by 1
  {\edef\next{\immediate\write\notesfile{\noexpand\note\the\fnno}}\next}%
  \immediate\write\notesfile{#1}%
  \immediate\write\notesfile{}}

%\gdef\endfn{"endfootnote}
%\gdef\putfnno{\global\advance\fnno by 1
%   ${}^{\the\fnno}$}
%\gdef\footsetup{\edef\fntempsf{\the\spacefactor}\putfnno
%   \spacefactor=\fntempsf{} }
%\gdef\footbody{\ifling@gathernotes\immediate\write\notesfile{}\fi%
%   \edef\n@xt{\noexpand\noexpand\noexpand \note\the\fnno}%
%   \ifling@gathernotes\immediate\write\notesfile{\n@xt}\fi\fnverbatim}
%\gdef\footnote{\footsetup \footbody}       %"endfootnote
%\gdef\fnverbatim{\begingroup\setupcopyfn\copyfnfirst}
%\gdef\setupcopyfn{\def\do##1{\catcode`##1=12 }\dospecials
%   \obeylines}
%{\obeylines
%\gdef\copyfn#1
%   {\def\n@xt{#1}%
%    \ifx\n@xt\endfn%
%         \ifling@gathernotes\immediate\write\notesfile{}\fi%
%         \let\n@xt=\endgroup%
%      \else%
%         \ifling@gathernotes\immediate\write\notesfile{\n@xt}\fi%
%         \let\n@xt=\copyfn \fi%
%         \n@xt}
%\gdef\copyfnfirst#1
%   {\def\n@xt{#1}%
%   \ifx\n@xt\empty%
%   \else\ifling@gathernotes\immediate\write\notesfile{\n@xt}\fi%
%   \fi \copyfn}}

%% printing notes and references
\def\dumpnotes{\immediate\closeout\notesfile
   \dumpnotesinit
   \def\note ##1 {%
      \smallskip\penalty10\noindent ##1.\quad}%
   \smalltext
   \input !notes.tex}
\def\dumpnotesinit{\vskip30pt
   \vskip.3\vsize\penalty-30\vskip-.3\vsize
   \centerline{Notes}
   \medskip}
\def\dumprefs{\vskip30pt
   \vskip.3\vsize\penalty-30\vskip-.3\vsize
   \smalltext
   \parindent=0pt
   \parskip=0pt plus 1pt
   \rightskip=0pt
   \tolerance=350 \interlinepenalty=400
   \centerline{\normaltext References}
   \bigskip
   \input @@refs.tex }


% -- numbering and reference to numbers

\newcount\exno \exno=0
\newcount\tracinglabels \tracinglabels=0

\def\nextx{{\advance\exno by 1 \number\exno}}
\def\anextx{{\advance\exno by 2 \number\exno}}
\def\lastx{\number\exno}
\def\blastx{{\advance\exno by -1 \number\exno}}
\def\bblastx{{\advance\exno by -2 \number\exno}}

%% TAGS
\newread\ftagsin
\newwrite\ftagsout
%  input !ftags.tex, first testing to see that it exists
\def\fdef#1 #2 {\expandafter\xdef\csname TAG#1\endcsname{#2}}
\immediate\openin\ftagsin=!ftags.tex
\ifeof\ftagsin \else \closein\ftagsin \input !ftags \fi
\ifling@gatherftags \immediate\openout\ftagsout=!ftags.tex \fi
%  local tags for example numbers
\def\tagex[#1]{%
   \expandafter\xdef\csname TAG#1\endcsname{\the\exno}\ignorespaces}
% f(ar)tags
\def\ftagpage[#1]{\ifling@gatherftags
   \write\ftagsout{\noexpand\fdef #1 {\the\pageno}}\fi \ignorespaces}
\def\ftag#1[#2]{\ifling@gatherftags
   \immediate\write\ftagsout{\noexpand\fdef #2 {#1}}\fi\ignorespaces}
\def\currsec{\ifnum\chapno>0 \the\chapno
   \ifnum\secno>0 .\the\secno
   \ifnum\subsecno>0 .\the\subsecno
   \ifnum\subsubsecno>0 .\the\subsubsecno \fi\fi\fi\fi}
\def\currex{\the\exno}
\def\ftagex[#1]{\ftag\the\exno[#1]}
\def\ftagsec[#1]{\ftag\currsec[#1]}
%  retrieving tags
\def\gettag[#1]{\expandafter\ifx\csname TAG#1\endcsname\relax
   \ifling@gatherftags \else \write16{****TAG NOT DEFINED**** [#1]}\fi
    $\bullet${\tt #1}\else
    \ifnum\tracinglabels=1
       {\tt !#1}\else \csname TAG#1\endcsname\fi\fi}
%  Use: \gettagrel[Malay:-2]  the Malay number minus 2
%  Caution: the tag retrieved must be numeric
\def\gettagrel[#1:#2]{{\expandafter\ifx\csname TAG#1\endcsname\relax
   \tracinglabels=2 $\bullet$\fi
   \ifnum\tracinglabels=0 \count0=\csname TAG#1\endcsname
      \advance\count0 by #2 \the\count0
   \else \tt {\ifnum\tracinglabels=1 !\fi}#1:#2\fi}}

% -- right arrows ---------------------

\newskip\beforetoskip
\newskip\aftertoskip
\newskip\normaltoskipamount
\normaltoskipamount=1em minus .5em
\def\normaltoskips{\beforetoskip=\normaltoskipamount
   \aftertoskip=\normaltoskipamount}
\normaltoskips
\newdimen\torise
\torise=1ex
\def\to{\ifmmode \rightarrow \else
   \hskip\beforetoskip$\rightarrow$\hskip\aftertoskip \fi}
\def\tooby #1 {\hskip\beforetoskip\hbox{$\buildrel {\rm #1}
   \vrule width0pt depth\torise height0pt
   \over\longrightarrow$}\hskip\aftertoskip}

% -- alignment ---------------

\newskip\Tskip \Tskip=2.4em
\def\Tspace{\hskip\Tskip}
\def\hwit#1{\hidewidth \it #1\hidewidth}

% numbering and lettering in \halign
\let\numbers=\letters
\newcount\lettercharno \lettercharno=1
\newdimen\letterswd \letterswd=2em
\def\omitletter{\omit\hskip\letterswd}
\def\iniletters{\global\lettercharno=96 }
\def\ininumbers{\lettercharno=48}
\newtoks\everyletter \everyletter={}
\def\letters{\global\advance\lettercharno by 1
   \hbox to\letterswd{\rm \the\everyletter\char\the\lettercharno.\hfil}}
\let\numbers=\letters
\def\lettering{\iniletters
   \def\\{\global\advance\lettercharno by 1 \char\the\lettercharno}}
%% the following should be placed on the line before the
%% line in which the letter is set
\def\tagletterA{\xdef\letterA{\char\the\lettercharno}}
\def\tagletterB{\xdef\letterB{\char\the\lettercharno}}
\def\tagletterC{\xdef\letterC{\char\the\lettercharno}}
\def\tagletterD{\xdef\letterD{\char\the\lettercharno}}
\def\tagletterE{\xdef\letterE{\char\the\lettercharno}}
%%%%% \crskip matters
\def\crs{\cr\noalign{\vskip\crskip}}
\newskip\crskip
\newskip\normalcrskipamount  \normalcrskipamount=.6em
\def\normalcrskip{\crskip=\normalcrskipamount}
\normalcrskip
%%%%% utilities
\def\clap#1{\hbox to 0pt{\hss#1\hss}}

\def\ling@doublespacestyle{%
   exhsize=6truein,
   beforeexskip=1.4em plus.1em,
   afterexskip=1.4em plus.1em,
   interpartskip=0pt}

%%%%%%%%%%%%%%%%%%%%%%%%%%%%%%%%%%%%%%%%%%%%%%%%%%%%%%%%

\lingrestore @

\everymath={\rm}



