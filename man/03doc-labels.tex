
\section Labels

In tree descriptions, material in |{|\dots|}| is typeset in an
hbox, called a {\it label box}.  After the label box is added to
a structure with current point $\cal P$, a new point $\cal P'$
becomes the current point. The issues addressed in this section
are the relative positions of $\cal P$, the label box, and $\cal
P'$.  The position of the label box is specified by giving the
position of the center of its baseline, called $\cal Q$ here. The
relative positions depend upon the height $h$ and depth $d$ of
the label box, and five different parameters: {\tt
\index{labelgapt}}, {\tt \index{labelgapb}}, {\tt
\index{labelstrutt}}, {\tt \index{labelstrutb}}, and {\tt
\index{labeloffset}}. The beginning user should not be
discouraged by the apparent complexity.  Various defaults are set
in \pstjtree\/ and the issue can be largely ignored unless some
special effect is desired.  When the time comes that a complex
positioning problem arises, the flexibility will be both
understandable and welcome. The positioning rules are given
below, with \value{parameter} representing the value of the
parameter.

\ex \psset{unit=1em}
\quad
\pspicture[shift=-9](0,1.8)(10.2,12)%\psgrid
\pspolygon(0,6)(8,6)(8,9)(0,9)
\psline[linestyle=dashed](0,7)(8,7)
\qdisk(4,7){2pt}
\rput[lb](4.3,7.3){$\cal Q$}
\qdisk(2,11){2pt}
\rput[lb](2.3,11.3){$\cal P$}
\qdisk(2,4){2pt}
\rput[lb](2.3,4.3){$\cal P'$}
\psset{nodesep=0}
%
\pcline{|<-|}(9,7)(9,4)
\lput(.5){\rput(2ex,0){$y_2$}}
\pcline{|<-}(9,11)(9,7)   %|
\lput(.5){\rput(2ex,0){$y_1$}}
%
\pcline{|->|}(2,3)(4,3)
\lput(.5){\rput(0,-2ex){$x_1$}}
\endpspicture
\hfil
\vtop{\hsize=3in \openup.5ex
$x_1=\vert|labeloffset|\vert$\smallskip
$y_1=\mathop{\rm
Max}(h,\vert|labelstrutt|\vert)$\par
\hfill $+\vert|labelgapt|\vert$\smallskip
$y_2=\mathop{\rm
Max}(d,\vert|labelstrutb|\vert)$\par
\hfill $+\vert|labelgapb|\vert$\smallskip
}
\xe

A consequence of these rules is that the top of a label box will
be at a distance of at least $\vert|labelgapt|\vert$ below the
terminus of the branch or label that it follows, and the baseline
of a label box will be at a distance
$\vert|labelstrutt|\vert+\vert|labelgapt|\vert$ below that
terminus unless the label box is unusually high (i.e.
$h>\vert|labelstrutt|\vert$).  This means that label boxes never
get too close to the branches or labels they follow, and the
baselines of label boxes on different branches of the same height
will be aligned, unless the label boxes are exceptionally high.
The same considerations apply to the relative positions of the
label box and $\cal P'$.

\medskip
There is also a parameter
|normallabelstrut|\index*{normallabelstrut} which can be set to
{\sl true\/} or {\sl false}.  If set to {\sl true\/}, every time
that |\jtree| is executed the dimensions of the label strut are
set to the dimensions of the current Tex strut. Specifically,
\medskip
\hfil |\psset{labelstrut={\the\ht\strutbox} {\the\dp\strutbox}}|
\medskip
is executed when |\jtree| is invoked. |\psset{labelstrut=�x\/�
�y�}|\index*{labelstrut} is equivalent to
|\psset{labelstrutt=�x\/�,|\allowbreak|labelstrutb=�y\/�}|. Users
do not have to be concerned with setting label strut unless they
desire some special effect since \pstjtree\/ sets
|normallabelstrut| to {\sl true}.  The default settings for
the top and bottom label gaps are $\rm .35\,ex$.
|\psset{labelgap=�x\/�}|\index*{labelgap} is equivalent to
|\psset{labelgapt=�x\/�,labelgapb=�x\/�}|.

\medskip
The effect of this scheme is shown below with |normallabelstrut|
set to {\sl true\/} and |labelgapt| and
|labelgapb| set to $\rm .6\,ex$.
The size of the labels is as shown, with the baselines shown. If
the height of the label does not exceed \value{labelstrutt}, the
baselines of the labels are aligned, otherwise the top of the
label box is at a distance \value{labelgapt} below the terminus
of the branch it follows.

\bigskip
\dimpuba=.8ex
\dimpubb=.5ex
\dimpubc=.5ex
\def\foop{\vrule width0pt height\dimpubb depth\dimpubc%
   \pspolygon(-\dimpuba,-\dimpubc)(\dimpuba,-\dimpubc)%
      (\dimpuba,\dimpubb)(-\dimpuba,\dimpubb)%
   \psline(-\dimpuba,0)(\dimpuba,0)%
   \global\advance\dimpubb by .68ex}%
\hfil\jtree[xunit=2cm,yunit=1cm,labelgap=.6ex]
\! =    <wideleft>{\foop}@A
   ^<left>{\foop}
   ^<vert>{\foop}
   ^<right>{\foop}
   ^<wideright>{\foop}.
\SpecialCoor
\pcline[style=dotted](-2.2,0|A)(2.2,0|A)
\endjtree
\bigskip

Sometimes it is useful to set |labelstrutt| to 0.  In that case,
the result is:

\bigskip
\hfil\jtree[xunit=2cm,yunit=1cm,labelgap=.6ex,normallabelstrut=false,
   labelstrut=0pt 0pt]
\dimpubb=.5ex
\! =    <wideleft>{\foop}@A
   ^<left>{\foop}
   ^<vert>{\foop}
   ^<right>{\foop}
   ^<wideright>{\foop}.
\SpecialCoor
\endjtree


\bigskip
Many users will probably leave |normallabelstrut| set to {\sl
true\/} and forget about it.  But most users will want to change
the label gaps from time to time.  Contrast the following, for
example.  In some applications,  the second might be preferable.

\exh
\tac|
\jtree
\! = {A}
   <vert>{\psframebox{B}}
   :{C}{D}.
\endjtree
|endtac \hfill
\jtree
\! = {A}
   <vert>{\psframebox{B}}
   :{C}{D}.
\endjtree
\kern3em

\bigskip
\tac|
\jtree
\! = {A}
   <vert>{\psframebox{B}}[labelgap=0]
   :{C}{D}.
\endjtree
|endtac \hfill
\jtree
\! = {A}
   <vert>{\psframebox{B}}[labelgap=0]
   :{C}{D}.
\endjtree
\kern3em
\xe

This is used in typesetting (\gettag[qtree]) in
Section~\gettag[incremental].

\medskip
The following contrast is also interesting, in part because it
uses a negative label gap.\par

\exh
\tac|
\jtree
\! = :({the}{(article)}) {dog}{(noun)}.
\endjtree
|endtac \hfill
\jtree
\! = :({the}{(article)}) {dog}{(noun)}.
\endjtree
\kern3em
\bigskip
\tac|
\jtree
\! = :({the}{(article)}[labelgapt=-3pt])
   {dog}{(noun)}[labelgapt=-3pt].
\endjtree
|endtac \hfill
\jtree
\! = :({the}{(article)}[labelgapt=-3pt])
   {dog}{(noun)}[labelgapt=-3pt].
\endjtree
\kern3em
\xe

See examples \gettag[ExAndrews] and \gettag[ExMerchant] in \exsec\
for applications of this trick, which often eliminates the need
for complex multiline labels.

\medskip
The label box can be positioned horizontally using |labeloffset|.

\exh
\tac|
\jtree
\! = {musketeer}
   <vert>{musketeer}[labeloffset=1ex]
   <vert>{musketeer}[labeloffset=2ex].
\endjtree
|endtac \hfil
\jtree
\! = {musketeer}
   <vert>{musketeer}[labeloffset=1ex]
   <vert>{musketeer}[labeloffset=2ex].
\endjtree
\xe

See example \gettag[Dowty] in \exsec\ for an illustration of how
changing |labeloffset| can solve certain spacing problems between
labels.

\medskip
There is one other parameter that is relevant to labels. The
material specified by |everylabel|\index*{everylabel} is put into
a token list and the token list is inserted at the beginning of
every label.

\exh \tac|
\jtree[everylabel=\sl]
\! = {a}
   :{a} {p}
   :{a} {(e)}.
\endjtree
|endtac \hfil
\jtree[everylabel=\sl]
\! = {a}
   :{a} {p}
   :{a} {(e)}.
\endjtree
\xe

\medskip
If a label is |{\omit|\dots|}|\index*{+omit} or
|{\pnode|\dots|}|, then the vertical positioning algorithm
discussed above is bypassed and the label is positioned with its
top edge at the level of $\cal P$ and $\cal P'$ is positioned
directly under $\cal P$ at the level of the bottom edge of the
label box.  |labeloffset| still operates. Contrast the following:

\exh \tac|
\jtree[everylabel=\strut,labelgap=3pt]
\! = :{a} {\omit\psframebox{dog\strut}}
   :{a} {\pnode{A1}}
   :{a} {a}.
\endjtree
|endtac \hfil
\jtree[everylabel=\strut,labelgap=3pt]
\! = :{a} {\omit\psframebox{dog\strut}}
   :{a} {\pnode{A1}}
   :{a} {a}.
\endjtree
\bigskip
\tac|
\jtree[everylabel=\strut,labelgap=3pt]
\! = :{a} {\psframebox{dog\strut}}
   :{a} {}
   :{a} {a}.
\endjtree
|endtac \hfil
\jtree[everylabel=\strut,labelgap=3pt]
\! = :{a} {\psframebox{dog\strut}}
   :{a} {}
   :{a} {a}.
\endjtree
\xe

\subsection 0-dimensional labels

Consider the trees below.

\exdisplay \labels
\tl\quad \jtree \! = {CP} :{Spec} :{C} {TP}.\endjtree
\hfil
\tl\quad\jtree \! = {CP} :{Spec} {C$'$} :{C} {TP}.\endjtree
\hfil
\tl\quad \jtree \! = {CP} :{Spec} {\elc{C$'$}} :{C} {TP}.\endjtree
\xe

The various codings are:

\exdisplay \labels
\tl\quad |\jtree \! = {CP} :{Spec} :{C} {TP}.\endjtree|
\medskip
\tl\quad |\jtree \! = {CP} :{Spec} {C$'$} :{C} {TP}.\endjtree|
\medskip
\tl\quad |\jtree \! = {CP} :{Spec} {\elc{C$'$}} :{C} {TP}.\endjtree|
\xe

The first has no label over the $\fam0 {C,TP}$ subtree.  (b), of
course does.  The last has a label, but it is 0-dimensional.
The label itself occupies no space, but |\elc{C$'$}|\index*{+elc} uses PSTricks to
typset C$'$ a specified displacement from the 0-dimensional
label.  There are a few commands (|\elc|, |\pnode|, and |\omit|)
that cause jTree to make a 0-dimensional label, with no gaps, provided they
appear first in the label.
\medskip

Note the following.

\exh
\tac|
\jtree
\! = {CP} :{Spec} {} :{C} {TP}.
\endjtree
|endtac \hfil
\jtree \! = {CP} :{Spec} {} :{C} {TP}.\endjtree
\bigskip
\tac|
\jtree
\! = {CP} :{Spec} {\omit} :{C} {TP}.
\endjtree
|endtac \hfil
\jtree \! = {CP} :{Spec} {\omit} :{C} {TP}.\endjtree
\xe
An empty node, |{}|, is not interpreted as 0-dimensional since a
strut is automatically inserted.
\medskip

The jTree macro |\elc| (empty label comment) positions the
comment (it is not the label) according to the setting of the
parameters |elcxoffset|\index*{elcxoffset},
|elcyoffset|\index*{elcyoffset}, and |elcref|\index*{elcref}.  The latter
parameter determines the reference point in the label which is
placed at the designated position.  The default setting of
|elcref| is |bl| (bottom left).  The possible settings of
|elcref| are the same as the possible reference points for the
PSTricks macro |\rput|, which is used to implement |\elc|.  Note,
by the way, that since |\rput| is used, the comment positioned by
|\elc| does not contribute to the bounding box of the tree which
is constructed. See Section \gettag[BBSec] for discussion of the
bounding box. The default settings of the positioning dimensions
are |elcxoffset=.4em| and |elcyoffset=0|.




