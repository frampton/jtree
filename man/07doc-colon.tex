

\section The colon construction in more detail

There are some details of the syntax of the colon construction
that have been left vague to this point.  The idea of the colon
construction is simple enough; the colon is replaced by the
branch |<colonA>| and then, after some material has been
processed, |^<colonB>| is inserted.  But some precision is needed
in specifying exactly where this later insertion takes place. For
example, where is |^<colonB>| inserted in the expression below?

\medskip
\qquad |:{a} [labelgap=0] @AA !a @BB {c}|

\medskip
The rule is this. The template below is filled out a fully as
possible, then |:| is replaced by |<colonA>| and |^<colonB>| is
inserted after the target.

\def\\#1{\psframebox[framesep=1pt]{\vrule
   height8.4pt depth2.5pt width0pt #1}}%
\setbox\tmpbox=\hbox{$|{�stuff\/�}|\quad
|[|{\sl pars\/}|]|\quad @tag\quad
\left(\matrix{|(|\;\ldots\;|)|\cr
\noalign{\smallskip}\hbox{\sl !tag}\cr}\right)$}
%\psframebox[framesep=1pt]{\vrule
%   height8.4pt depth2.5pt width0pt ;}$}
%
\ex \quad $\sl
|:|\quad |[|{\sl pars\/}|]|\quad
\underbrace{\vrule width0pt depth1.5em\box\tmpbox}_{\textstyle\rm target}$
\xe

There are two restrictions.  First, the target cannot be
completely empty.  |:<left>|\dots\ or |::|\dots, for example, is
impossible.  Second, since parameters must be parameters of
something, the |[�pars\/�]| term in the target can only be
present if there is a |{�stuff\/�}| term.

\medskip
Consider the following examples:

\pex
\a
\tac|
\jtree
\! = :[scaleby=2]{a} {b}.
\endjtree
|endtac \hfill
\jtree
\! = :[scaleby=2]{a} {b}.
\endjtree
\kern1em
%
\a
\tac|
\jtree
\! = :{a} [scaleby=2]{b}.
\endjtree
|endtac \hfill
\jtree
\! = :{a} [scaleby=2]{b}.
\endjtree
\kern1em
\xe

You might expect the right branch to be scaled in (\lastx b). The
reason that it is not is that |[scaleby=2]| goes into the target.
It is parsed, according to (\blastx), as label parameters and
|<colonB>| is inserted after the target.  As label parameters, it
has no effect.  In Section~\gettag[ExamplesSec], the following
idiom is frequently used to ensure that parameters are not
interpreted as label parameters.

\pex
\tac|
\jtree
\! = :{a}() [scaleby=2]{b}.
\endjtree
|endtac \hfil
\jtree
\! = :{a}() [scaleby=2]{b}.
\endjtree
\xe

The template (\bblastx) is filled out with target |{a}()|, so
|<colonB>| is inserted before |[scaleby=2]|.

\medskip
The following contrast is interesting.  In (\nextx a), the target
is |{a}(:{b}{c})|, so inline adjunction occurs on the left
branch.  In (\nextx b), on the other hand, |!a| fills the
adjunction slot in the template, so the target is |{a}!a| and the
adjunction is on the right branch.

\pex[interpartskip=1em]
\a \tac|
\jtree
\! = :{a} (:{b}{c}) {d}.
\endjtree
|endtac \hfill
\jtree
\! = :{a}(:{b}{c}){d}.
\endjtree\kern2em
%
\a \tac|
\jtree
\! = :{a}!a (:{b}{c}){d}.
\endjtree
|endtac \hfill
\jtree
\! = :{a}!a (:{b}{c}){d}.
\endjtree
\kern2em
%
\xe

Note in (\lastx b) that the inline adjunction leaves the current
point \Pcurr\ unchanged, so it is overwritten by the label which
follows.

\medskip
A target can consist entirely of a @tag.

\ex
%
\tac|
\jtree
\! = :@A @B .
\nccurve[angleA=-60,
   angleB=240]{->}{A}{B}
\endjtree
|endtac \hfill
\jtree
\! = :@A @B .
\nccurve[angleA=-60,angleB=240]{->}{A}{B}
\endjtree\kern3em
\xe

For the sake of completeness, a full account of the tree
description grammar follows.

\subsection The syntax of the tree description language

\ftagpage[syntaxpage]%
The well formed tree descriptions are specified by a context free
grammar and some assumptions about how the parser operates.  We
first consider the grammar.  The term {\sl string\/} which occurs
in a number of expansions is a string of characters which can be
put into a Tex control sequence.  Various conditions on the
strings which can appear in different contexts are discussed
below.

\begingroup
\def\ttbox{\ttverbatim \spaceskip=.5em \ttboxA}
\def\ttboxA#1{\psframebox[framesep=1pt]{\vrule
   height8.4pt depth2.5pt width0pt #1}\endgroup}
\def\plus{+}
\catcode`\+=\active
\let+=\ttbox
\def\TTbox{\setbox\tmpbox\hbox\bgroup}
\def\endTTbox{\egroup \psframebox[framesep=1pt]{\vrule
   height8.4pt depth2.5pt width0pt \box\tmpbox}}
\leftskip=2em
\edef\normalspaceskip{\noindent\spaceskip=\the\spaceskip}
\spaceskip=1.6ex \jot=.7ex \openup\jot
\catcode`_=\active
\def\TO{ $\to$ }%
\def\opt#1{\hbox{$(\,\sl #1\,)$}}%
\def\optpars{\opt{+[\,pars\,+]}}%
\def\dagnote{\hskip1ex$^{\dag}$}
\sl

\medskip
tree\_description\TO
   simple\_tree\_description \opt{tree\_description}

{\sl simple\_tree\_description\/}\TO \TTbox|\!|\kern-1pt\endTTbox
   {\sl string\/}\TTbox\kern.5em\endTTbox\kern1pt +=
   \opt{tree\_body} +.

tree\_body\TO tree\_item \opt{tree\_body}

tree\_item\TO  sprout, target, vartri\_group,
colon\_group, operator\kern1em

sprout\TO +<{\sl string\/}+> \optpars

target\TO $\sl \pmatrix{label\cr \hbox{\sl @tag}\cr}$
$\sl \pmatrix{\hbox{\sl !tag}\cr \hbox{\sl inline\_adjunction}\cr}$

\smallskip
label\TO $\sl +{\,stuff\,+}$ \optpars\ \opt{@tag}

inline\_adjunction\TO $\sl +(\,tree\_body\,+)$

vartri\_group\TO +<{\sl string\/}+> \opt{parameters}
   {\sl label}

colon\_group\TO +: \optpars\ target \optpars

@tag\TO +@{\sl string\/}+ \relax

!tag\TO +!{\sl string\/}+ \relax

operator\TO +^

pars\TO {\rm\normalspaceskip valid PSTricks parameter settings},
$\emptyset$

\bigskip
\endgroup

The boxed characters above are symbols in the tree description
language, not the grammar description language. ``Stuff'' is
anything that will go in a Tex |\hbox|. Note that an empty
parameter specification |[]| is permitted, and is actually useful
at times.  This is an extension of the PSTricks parameter system,
which disallows empty parameters.

\medskip
The string which follows |\!| must be such that {\sl !string\/}
is the name of an adjunction point that has already been defined,
either by |\jtree| or by a !tag in a previously parsed target.
The |<�\sl string\/�>| which occurs in a sprout must be the name
of a branch or triangle, and the |<�\sl string\/�>| which occurs
in a vartri group must be the name of a vartriangle. A name with
an empty string or a string containing spaces is permitted, but
not advised. The string which occurs in a @tag must be a valid
PSTricks node name, with the additional restriction that it
cannot contain a space.  The string which occurs in a !tag,
preceded by |!|, becomes the name of an adjunction point. It
cannot contain a space. Note that !tags and @tags must be
followed by a space. Otherwise, \jTree\ is tolerant of spaces or
their absence between items.

\medskip
Parsing is unique under the assumption that targets and labels
are always maximized in left to right parsing and are never
empty. Labels are never empty in any event since they contain |{|
and |}|.



\endinput


%
%\a \tac|
%\jtree
%\! = :@A !a @B !b .
%\nccurve[angleA=90,angleB=90,
%   ncurv=1]{->}{B}{A}
%\!a = {a}(@AA ).
%\!b = {b}@BB .
%\ncline[linestyle=dashed,
%   nodesep=0]{->}{AA}{BB}
%\endjtree
%|endtac \hskip-2em\hfill
%\jtree[treevshift=-1em]
%\! = :@A !a @B !b .
%\nccurve[angleA=90,angleB=90,
%   ncurv=1]{->}{B}{A}
%\!a = {a}(@AA ).
%\!b = {b}@BB .
%\ncline[linestyle=dashed,nodesep=0]{->}{AA}{BB}
%\endjtree\kern2em
%
%\a \tac|
%\jtree
%\! = :({a}<vert>{b}) {c}.
%\endjtree
%|endtac \hfill
%\jtree
%\! = :({a}<vert>{b}) {c}.
%\endjtree
%\kern2em
%
%\kern2em
%\xe

Note: It is likely that there are some errors or inconsistencies
in the description of the grammar and/or its implementation. If
you find any, I would greatly appreciate it if you let me know at
{\sl j.frampton@neu.edu}.

