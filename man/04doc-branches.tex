
\section Branches

Most tree formatting schemes format the array of labels, then
connect them with branches.  The dimensions of the branches are
determined by the location of the labels they must connect. A
\jTree\ branch, on the other hand, has dimensions specified in
its definition and determines the location of the material it
points to. Branches are defined by giving their height and
slope. The syntax is:\index*{+defbranch}

\medskip
\quad |\defbranch<�name\/�>(�height\/�)(�slope\/�)|
\medskip

The height can have Tex dimensions (pt, ex, em, in, cm, etc.), or
it can be purely numerical. If numeric, it is taken to be the
height measured in psyunits. The slope is the ratio of the
vertical dimension measured in psyunits to the horizontal
dimension measured in psxunits. Lines going from the lower left
to the upper right have positive slopes and lines going from the
upper left to the bottom right have negative slopes.  Usually,
the slope can be expressed as a decimal number.  If the
horizontal dimension is zero, as it is with a vertical line, a
decimal slope is impossible (because division by zero is) and the
slope must be expressed as a ratio.

\exh
\tac|
\jtree[unit=2.5em]
\defbranch<1;3/10>(1)(3/10)
\defbranch<1;.5>(1)(.5)
\defbranch<1;1/1>(1)(1/1)
\defbranch<1;1/0>(1)(1/0)
\defbranch<1;-1>(1)(-1)
\defbranch<1;-1/2>(1)(-1/2)
\! = <1;3/10>{$3/10$}
   ^<1;.5>{$.5$}
   ^<1;1/1>{$1/1$}
   ^<1;1/0>{$1/0$}
   ^<1;-1>{$-1$}
   ^<1;-1/2>{$-1/2$}.
\endjtree
|endtac \hfil
\jtree[unit=2.5em]
\defbranch<1;3/10>(1)(3/10)
\defbranch<1;.5>(1)(.5)
\defbranch<1;1/1>(1)(1/1)
\defbranch<1;1/0>(1)(1/0)
\defbranch<1;-1>(1)(-1)
\defbranch<1;-1/2>(1)(-1/2)
\! = <1;3/10>{$3/10$}
   ^<1;.5>{$.5$}
   ^<1;1/1>{$1/1$}
   ^<1;1/0>{$1/0$}
   ^<1;-1>{$-1$}
   ^<1;-1/2>{$-1/2$}.
\endjtree
\xe

Note that the characters appearing in a branch name are not
restricted to alphabetical characters.

\medskip
I have never found a use for defining branches with the height
specified in physical units, but someone might.

\exh
\tac|
\jtree[xunit=2em,yunit=1em]
\defbranch<Right>(2em)(-1)
\! = <left>{a} ^<Right>{b}
   <left>{c} ^<right>{d}.
\endjtree
|endtac \hfil
\jtree[xunit=2em,yunit=1em]
\defbranch<Right>(2em)(-1)
\! = <left>{a} ^<Right>{b}
   <left>{c} ^<right>{d}.
\endjtree
\xe

Finally, note that the slope specification of a branch will not
match the slope of the line which is drawn on paper unless the
xunits and yunits are equal.  One is the ratio of psyunits to
psxunits, the other is the ratio of physical units.  The branch
|<right>| is specified to have slope $-1$, but:

\exh
\tac|
\jtree[xunit=.5em,
   yunit=2em]
\! = {a}<right>{b}.
\endjtree
|endtac \hfil
\jtree[xunit=.5em,
   yunit=2em]
\! = {a}<right>{b}.
\endjtree
\bigskip
\tac|
\jtree[xunit=4em,
   yunit=2em]
\! = {a}<right>{b}.
\endjtree
|endtac \hfil
\jtree[xunit=4em,
   yunit=2em]
\! = {a}<right>{b}.
\endjtree
\xe

\jTree\ specializes in binary trees, but \pstjtree\/ predefines
some branches which make it relatively easy to make bushier
trees. The entire inventory of predefined branches is:\medskip

\cframe
|\defbranch<left>(1)(1)
\defbranch<right>(1)(-1)
|medskip
\defbranch<4wideleft>(1)(2/3)
\defbranch<4left>(1)(2)
\defbranch<4right>(1)(-2)
\defbranch<4wideright>(1)(-2/3)
|medskip
\defbranch<wideleft>(1)(1/2)
\defbranch<wideright>(1)(-1/2)
|medskip
\defbranch<bigleft>(2)(1)
\defbranch<bigright>(2)(-1)
|medskip
\defbranch<vert>(1)(1/0)
\defbranch<shortvert>(.5)(1/0)
|medskip
\defbranch<colonA>(1)(1)
\defbranch<colonB>(1)(-1)
|endcframe

Of course, users can define whatever branches they find need for.
|<colonA>| and |<colonB>| are the branches used by the colon
macro.  They are defined to be the same as |<left>| and
|<right>|, but can be redefined to be whatever is convenient.
\medskip

\exh \tac|
\jtree
\! = {X}
   <left>{a} ^<vert>{b} ^<right>{c}.
\endjtree
|endtac \hfil
\jtree
\! = {X}
   <left>{a} ^<vert>{b} ^<right>{c}.
\endjtree

\bigskip
\tac|
\jtree
\! = {X}
   <4wideleft>{a} ^<4left>{b}
      ^<4right>{c} ^<4wideright>{d}.
\endjtree
|endtac \hfil
\jtree
\! = {X}
   <4wideleft>{a} ^<4left>{b}
      ^<4right>{c} ^<4wideright>{d}.
\endjtree

\bigskip
\tac|
\jtree
\! = {X}
   <wideleft>{a} ^<left>{b} ^<vert>{c}
      ^<right>{c} ^<wideright>{d}.
\endjtree
|endtac \hfil
\jtree
\! = {X}
   <wideleft>{a} ^<left>{b} ^<vert>{c}
      ^<right>{c} ^<wideright>{d}.
\endjtree
\dc






