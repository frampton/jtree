\section Bells and whistles

\vskip-1ex
\subsection {\sl baretopadjust}

The baseline of the box created by |\jtree...\endjtree| is
normally the baseline of the root label.  If, however, the root
label is empty, that puts the baseline too low, at least for my
taste.  Contrast the trees below:
\bigskip
\qquad a.\quad
\jtree \! = {A} :{B} {C}.\endjtree
\hfil b.\quad
\jtree[baretopadjust=0] \! = :{B} {C}.\endjtree
\bigskip
\jTree\  takes corrective action by raising the tree diagram if
the root label is empty by the amount specified by the parameter
\index*{baretopadjust}|baretopadjust|.  The default setting is
1.4ex, but you can set it to whatever you want.  With the default
setting:
\bigskip
\qquad a.\quad
\jtree \! = {A} :{B} {C}.\endjtree
\hfil b.\quad \jtree[baretopadjust=1.4ex] \! = :{B} {C}.\endjtree

\subsection {\sl treevshift}

\jTree\ also provides the parameter
\index*{treevshift}|treevshift|, which is set to 0 by default. It
is independent of |baretopadjust| and can be used to move the
tree diagram up and down, if desired.

\exh \leavevmode\raise1.2em\hbox{\tac|
$\to$\qquad
\jtree[treevshift=1.2em]
\! = {A} :{B} {C}.
\endjtree
|endtac}\hfil
$\to$\qquad
\jtree[treevshift=1.2em]
\! = {A} :{B} {C}.\endjtree
\xe

\subsection {\sl everytree}

\index*{everytree}If you say, for example,
|\psset{everytree=\psset{style=treestyle}}|, and if you have
defined |treestyle| using a PSTricks style definition, then your
trees will done in that style.  |\psset{everytree=�x\/�}| puts
{\sl x\/} into the token list |\jteverytree| which is inserted at
the beginning of every |\jtree|\dots|\endjtree| construction.  It
is grouped, so that its scope is limited to the tree
construction.  It is inserted before the |\jtree| parameters take
effect, so that it will be overridden by parameter settings.

\medskip
If you want, you can set |\jteverytree| directly.  The same is
true, by the way, for |everylabel|, which uses the token list
|\jteverylabel|.

\subsection Custom branches

\ftag{\the\secno.\the\subsecno}[CustomSec]

\jTree\ ordinarily draws branches using the PSTricks macro
|\psline|, but it does this in an indirect way.  It actually
calls the macro |\branch@type| to do the drawing; and \jTree\
contains the line |\let\branch@type=\psline|.  If the user says
|\psset{branch=\customline}|, then |\branch@type| is made
|\customline| instead of |\psline| with the usual locality. If
|\customline| is suitably defined then it will be used to draw
branches.  A few alternative macros for drawing branches are
included in \jTree.  The enterprising user might want to define
appropriate zigzag or coil macros, following the PSTricks model.
\jTree\ provides |\blank| (see example \gettag[Zubi] in \exsec\
for an illustration of its use), |\brokenbranch| (see example
\gettag[Frampton2] in \exsec), and |\etcbranch| (see example
\gettag[Frampton1] in \exsec).%
\index*{branch}\index*{+blank}\index*{+brokenbranch}%,
\index*{+etcbranch}

\exh \tac|
\jtree[xunit=3em,yunit=2em]
\! = :{normal}()
        [branch=\brokenbranch]{broken}
   :[branch=\blank]{blank}()
      [branch=\etcbranch]{etc}.
\endjtree
|endtac \hfill
\jtree[xunit=3em,yunit=2em]
\! = :{normal}()
        [branch=\brokenbranch]{broken}
   :[branch=\blank]{blank}()
      [branch=\etcbranch]{etc}.
\endjtree
\kern1ex
\xe

The proportion of the ``etc branch'' that is dotted is controlled
by a parameter \index*{etcratio}|etcratio|.  \pstjtree\ contains
|\psset{etcratio=.75}|, but the user is free to change this if
desired.  The style for the dotted portion is defined by

\cframe
|\newpsstyle{etc}{nodesepB=0,nodesepA=1pt,linestyle=dotted,
   linewidth=1.2pt,dotsep=2pt}
|endcframe

The user can also overwrite the specification of this style
with a new specification, if desired.

\medskip
\pstjtree\/ contains the macro definition:

\cframe|\def\etc{[branch=\etcbranch,scaleby=.7]}|endcframe
\index*{+etc}\ftagpage[etcpage]

So you can write:

\exh \tac|
\jtree
\! = :{A} :{B} :{C}() \etc.
\endjtree
|endtac \hfil
\jtree
\! = :{A} :{B} :{C}() \etc.
\endjtree
\xe

\subsection The pseudo-parameters {\sl dirA\/} and {\sl dirB}

\index*{dirA}\index*{dirB}%
%
Suppose that you wish to use |\nccurve| to draw a curve which
leaves a node A in the same direction that a standard left branch
would leave A. You need to set |angleA| to the appropriate value.
The appropriate angle can be calculated using some simple
trigonometry, but the calculation depends on the ratio of the
psxunits to the psyunits.  It is desirable both to avoid having
to do a trig calculation on the side and to code a tree so that
unit changes do not alter the geometry in an undesirable fashion.
|dirA| and |dirB| were introduced to solve this problem.
\hbox{|\psset{dirA=(-1:-1)}|} will cause |angleA| to be set so
that a path drawn by |\nccurve| will leave the starting node in
the direction of the vector $(-1,-1)$. It is called a
pseudo-parameter (my terminology) because |\psset{dirA=�x\/�}| is
executed for its effect on the parameter |angleA|, not |dirA|.
Note that a colon is used, so that the |\psset| parser is not
confused. |dirB| works almost the same way for |angleB|, but the
vector which sets |dirB| should point backwards along the curve,
which is the direction which |angleB| measures.


\medskip
Use of |dirA| below ensures that the ``curved branch'' lines up
with the straight ones.  Note that the curve is made fairly stiff
(a high value of |ncurv|) so that it bows out sufficiently.

\exh \tac|
\jtree
\! = @A1
   <right>
   :{B}
   :{C} {D}@A2 .
\nccurve[dirA=(-1:-1),angleB=200,
   ncurv=2,nodesepA=0]{A1}{A2}
\endjtree\kern1em
|endtac \hfill
\jtree \! = @A1
   <right>
   :{B}
   :{C} {D}@A2 .
\nccurve[dirA=(-1:-1),angleB=200,
   ncurv=2,nodesepA=0]{A1}{A2}
\endjtree\kern1em
\xe

The example in \gettag[Frampton2] is a good illustration of the
use of this parameter.


\endinput
\def\andthen{\kern1pt:\enspace}

\subsection Complete list of \jTree\ parameters

|labelgapt|, |labelgapb|, |labelgap|\andthen The center of the
top of the label box is placed a distance |labelgapt| below the
previous termination point.  The termination point of the label
box is placed a distance |labelgapb| below the bottom edge of the
label box. |labelgap| is a pseudo-parameter, used for its effect.
|\psset{labelgap=�\sl x\/�}| has the effect
|\psset{labelgapt=�\sl x\/�,labelgapb=�\sl x\/�}|.

\medskip
|scaleby|\andthen |\psset{scaleby=�\sl x\enspace y\/�}| has the effect
of scaling all branches, triangles, and vartriangles by {\sl x\/}
horizontally and {\sl y\/} vertically.  If {\sl y\/} is omitted,
scaling by {\sl x\/} in both directions is carried out.

\medskip
|everytree|\andthen
The effect of
|\psset{everytree=�$\alpha$�}| is
|\jteverytree={�$\alpha$�}|.  This token list is inserted every
time that |\jtree| is invoked.

\medskip
|everylabel|\andthen
The effect of
|\psset{everylabel=�$\alpha$�}| is
|\jteverylabel={�$\alpha$�}|.  This token list is inserted every
time that a label box is created, unless cancelled by |\omit|.

\medskip
|treevshift|, |baretopadjust|\thinspace:\enspace These parameters
control the relation between the baseline of the assembled tree
and the baseline of the root label.  The baseline of the root
label is a distance $\sl d=v+b\mskip2mu$ above the baseline of
the tree, where $\sl v$ is |treevshift|; and $\sl b$ is
|baretopadjust| if the root label has 0 height, otherwise is~0.

\medskip
|triratio|\andthen This determines the termination point of
triangles, and the skew of vartriangles.  See Chapter~5 for
discussion.

\medskip
|branch|\andthen The control sequence |\branch@type| is used to
draw the connections in branches.  \pstjtree\/ contains
|\let\branch@type=\psline|.  The effect of
|\psset{branch=\usermacro}| is |\let\branch@type=\usermacro|. It
is assumed that |\usermacro| has the argument structure\smallskip
\hfil|\def\usermacro(#1,#2)(#3,#4){|\dots
\smallskip

\medskip
|etcratio|\andthen \pstjtree\/ provides the macro
|\etcbranch| as an alternative to |\psline|.  It draws a line
which begins as a solid line but ends as a dotted line.  Its use
is to indicate an elided tree tail.  |etcratio| determines what
fraction of the line is dotted.

\medskip
|dirA|, |dirB|\andthen These are pseudo-parameters, used to set
the parameters |angleA| and |angleB|, respectively.
|\psset{dirA=(�\sl a\/�:�\sl b\/�)}| sets angleA so that
the direction corresponding to angleA is the same as the
direction corresponding to the vector $\sl\langle a,b\rangle$.
Setting |dirB| has a similar effect on |angleB|.




