\parindent=2em

\example

\hrule
\medskip
\noindent This example, and the following four, illustrate some
techniques for displaying multidominance structures.  I have
argued that movement should be seen as creating multidominance
trees with shared structure.  My suspicion is that the difficulty
in portraying the resulting structures in a typeset diagram makes
the idea less appealing than it might otherwise be.  It would be
unfortunate for deficiencies in typesetting technology to have a
significant influence in shaping theoretical work.  What made
writing \pstjtree\/ worth the trouble was that I see it as a
possible contribution to the development of theories of mental
computation.

The influence of typesetting technology on linguistic theory
should not be underestimated.  The widespread inattention to
autosegmental structure in phonology is at least partially
a reflection of the state of typesetting technology.

Since McCawley (1982) seems to be the first linguist to attempt
to use multidominance to solve a complex syntax problem, we begin
with one of his examples (right node raising).
\medskip
\hrule

\bigskip
\noindent\hfil
\jtree[xunit=2.45em,yunit=1.4em,dirA=(1:-1),nodesep=0]
\def\\{[labelgapb=-4pt]}%
\def\V{$\rm \overline V$}%
\! = {S}
   <wideleft>{S}!a ^<vert>{and} ^<wideright>{S}
   :({NP}<shortvert>{Fred}) {\V}
   :({V}\\{knows}) {NP}
   <tri>{a man} ^<right>
   <right>[scaleby=3.5 1,branch=\blank]{NP}@A3 !b ^{S}
   :({NP}<shortvert>{who}) {\V}@A2
   <left>({V}\\{repairs}).
\!a = :({NP}<shortvert>{John}) {\V}@A1
   <left>{V}\\{sells}.
\!b = <vartri>{washing machines}.
\nccurve[angleB=150,ncurvB=1.4]{A2:b}{A3:t}
\nccurve[angleB=135,ncurvA=.5,ncurvB=2.6]{A1:b}{A3:t}
\endjtree

\vfil\break

\codelines|
\jtree[xunit=2.45em,yunit=1.4em,dirA=(1:-1),nodesep=0]
\def\\{[labelgapb=-4pt]}%
\def\V{$\rm \overline V$}%
\! = {S}
   <wideleft>{S}!a ^<vert>{and} ^<wideright>{S}
   :({NP}<shortvert>{Fred}) {\V}
   :({V}\\{knows}) {NP}
   <tri>{a man} ^<right>
   <right>[scaleby=3.5 1,branch=\blank]{NP}@A3 !b ^{S}
   :({NP}<shortvert>{who}) {\V}@A2
   <left>({V}\\{repairs}).
\!a = :({NP}<shortvert>{John}) {\V}@A1
   <left>{V}\\{sells}.
\!b = <vartri>{washing machines}.
\nccurve[angleB=150,ncurvB=1.4]{A2:b}{A3:t}
\nccurve[angleB=135,ncurvA=.5,ncurvB=2.6]{A1:b}{A3:t}
\endjtree
|endcodelines

\medskip
1. A blank branch is used to position the node which is shared
between the two conjuncts.

\smallskip
2. Note the high value of |ncurvB| that is used to properly
locate the curve.
