
\example

\ftagEx[ExSprout]

The following display is from Christos Vlachos, modeled on Chung,
Ladusaw \& McCloskey 1995: 247, (16).  (His code was slightly
modified in order to illustrate a few points of general
interest.)  It uses ideas from Section \gettag[CircleSec].

\exdisplay
\quad \jtree[unit=3.5em,yunit=1em,elcxoffset=.7ex,
   elcyoffset=.7ex,bbadjust=height 1ex depth 4ex]
\defbranch<wideright>(1)(-2/3)
\defbranch<shortright>(.5)(-2/3)
\defbranch<steepright>(.7)(-3/2)
\defbranch<steepleft>(.7)(3/2)
\! = {CP}
<left>{PP$_i$}@PPtop !a ^<wideright>{C$'$}
<left>{C$^0$}!b ^<wideright>{\elc{IP}}
<left>{DP}(<vartri>{\rnode{joan}{Joan}})
   ^<shortright>{\elc{I$'$}}
<left>{I$^0$} ^<shortright>{VP}@vp
<left>[scaleby=1.3]{VP}!c
   ^<wideright>[scaleby=.7]{PP$_i$}@PPbot .
\!a = <steepleft>{P}(<vert>{with})
   ^<steepright>{DP} <vartri>{whom$\fam0 ^x$}.
\!b = {[+Q]}[labelgapt=-.6ex]<vert>{e\rlap{$\fam0 ^x$}}.
\!c = <steepleft>{DP}
   ^<steepright>{\elc{V$'$}}
   <steepleft>{V$^0$}(<shortvert>{ate})
   ^<steepright>{DP}<vartri>{dinner}.
\nccurve[angleA=-120,angleB=-95,ncurvA=.7,ncurvB=1.2,
   nodesepA=0,nodesepB=.5ex,arrows=<->]{PPtop}{PPbot}
\ncput*[npos=.4]{\it merger}
\psLNode(joan)(PPbot){.5}{center1}
\psLNode(vp)(PPbot){.5}{center2}
\rput(center1){\pscircle{2.5}
   \rput[l](2.5;30){$\,\longleftarrow$ copying}}
\rput(center2){\pscircle{1}
   \rput[l](1;40){$\,\longleftarrow$ sprouting}}
\endjtree
\xe

\CLnum
\jtree[unit=3.5em,yunit=1em,elcxoffset=.7ex,
   elcyoffset=.7ex]
\defbranch<wideright>(1)(-2/3)
\defbranch<shortright>(.5)(-2/3)
\defbranch<steepright>(.7)(-3/2)
\defbranch<steepleft>(.7)(3/2)
\! = {CP}
<left>{PP$_i$}@PPtop !a ^<wideright>{C$'$}
<left>{C$^0$}!b ^<wideright>{\elc{IP}}
<left>{DP}(<vartri>{\rnode{joan}{Joan}})
   ^<shortright>{\elc{I$'$}}
<left>{I$^0$} ^<shortright>{VP}@vp
<left>[scaleby=1.3]{VP}!c
   ^<wideright>[scaleby=.7]{PP$_i$}@PPbot .
\!a = <steepleft>{P}(<vert>{with})
   ^<steepright>{DP} <vartri>{whom$\fam0 ^x$}.
\!b = {[+Q]}[labelgapt=-.6ex]<vert>{e\rlap{$\fam0 ^x$}}.
\!c = <steepleft>{DP}
   ^<steepright>{\elc{V$'$}}
   <steepleft>{V$^0$}(<shortvert>{ate})
   ^<steepright>{DP}<vartri>{dinner}.
\nccurve[angleA=-120,angleB=-95,ncurvA=.7,ncurvB=1.2,
   nodesepA=0,nodesepB=.5ex,arrows=<->]{PPtop}{PPbot}
\ncput*[npos=.4]{\it merger}
\psLNode(joan)(PPbot){.5}{center1}
\psLNode(vp)(PPbot){.5}{center2}
\rput(center1){\pscircle{2.5}
   \rput[l](2.5;30){$\,\longleftarrow$ copying}}
\rput(center2){\pscircle{1}
   \rput[l](1;40){$\,\longleftarrow$ sprouting}}
\endjtree|endCLnum  %|
\medskip

|\fam0| is used on lines 16 and 17 so that letters are taken
from the standard font set in math mode rather than the math
italic font set.
\medskip

Another version of Vlachos' display is given below.

\exdisplay
\ \jtree[xunit=3.5em,yunit=1em,elcxoffset=.7ex,
   elcyoffset=.7ex,bbadjust=height .5ex depth 5.5ex]
\defbranch<wideright>(1)(-2/3)
\defbranch<shortright>(.5)(-2/3)
\! = {CP}
<left>{PP$_i$}@PPtop !a ^<wideright>{C$'$}
<left>{C$^0$}!b ^<wideright>{\elc{IP}}
<left>{DP}(<vartri>{\rnode{joan}{Joan}})
   ^<shortright>{\elc{I$'$}}
<left>{I$^0$} ^<shortright>{\rnode{vp}{VP}}
<left>{VP}!c
   ^<wideright>[scaleby=.7]{PP$_i$}@PPbot .
\psset{scaleby=.5 .7}
\!a = <left>{P}(<vert>{with})
   ^<right>{DP} <vartri>{whom$\fam0 ^x$}.
\!b = {[+Q]}[labelgapt=-.6ex]<vert>{e\rlap{$\fam0 ^x$}}.
\!c = <left>{DP}
   ^<right>{\elc{V$'$}}
   <left>{V$^0$}(<vert>{ate})
   ^<right>{DP}<vartri>{dinner}.
\ncangles[angleA=0,angleB=-90,armA=2.9,armB=7.7,arrows=<->,
   linearc=.4,nodesepA=.5ex,nodesepB=.2ex]{PPtop}{PPbot}
\ncput*[npos=2.3]{\it merger}
\psLNode(joan)(PPbot){.54}{center1}
\psLNode(vp)(PPbot){.5}{center2}
\rput{-20}(center1){\jtenode*(2.5,7.2){40}{A1}}
\rput[l](A1){$\leftarrow$ copying}
\rput{-20}(center2){\jtenode*(1.1,2.2){10}{A2}}
\rput[l](A2){$\longleftarrow$ sprouting}
\endjtree
\xe

\CLnum
\jtree[xunit=3.5em,yunit=1em,elcxoffset=.7ex,
   elcyoffset=.7ex,bbadjust=height .5ex depth 5.5ex]
\defbranch<wideright>(1)(-2/3)
\defbranch<shortright>(.5)(-2/3)
\! = {CP}
<left>{PP$_i$}@PPtop !a ^<wideright>{C$'$}
<left>{C$^0$}!b ^<wideright>{\elc{IP}}
<left>{DP}(<vartri>{\rnode{joan}{Joan}})
   ^<shortright>{\elc{I$'$}}
<left>{I$^0$} ^<shortright>{\rnode{vp}{VP}}
<left>{VP}!c
   ^<wideright>[scaleby=.7]{PP$_i$}@PPbot .
\psset{scaleby=.5 .7}
\!a = <left>{P}(<vert>{with})
   ^<right>{DP} <vartri>{whom$\fam0 ^x$}.
\!b = {[+Q]}[labelgapt=-.6ex]<vert>{e\rlap{$\fam0 ^x$}}.
\!c = <left>{DP}
   ^<right>{\elc{V$'$}}
   <left>{V$^0$}(<vert>{ate})
   ^<right>{DP}<vartri>{dinner}.
\ncangles[angleA=0,angleB=-90,armA=2.9,armB=7.7,arrows=<->,
   linearc=.4,nodesepA=.5ex,nodesepB=.2ex]{PPtop}{PPbot}
\ncput*[npos=2.3]{\it merger}
\psLNode(joan)(PPbot){.54}{center1}
\psLNode(vp)(PPbot){.5}{center2}
\rput{-20}(center1){\jtenode*(2.5,7.2){40}{A1}}
\rput[l](A1){$\leftarrow$ copying}
\rput{-20}(center2){\jtenode*(1.1,2.2){10}{A2}}
\rput[l](A2){$\longleftarrow$ sprouting}
\endjtree|endCLnum

1. Rather than defining new branches |<steepleft>| and
|<steepright>|, the branches used to construct the |\!a|, |\!b|,
and |\!c| subtrees are preceded by a change of scale executed by
|\psset{scaleby=.5 .7}|.
\medskip
2. The merger connection is made using |\ncangles| rather than
|\nccurve|.  Note how |npos| works with |\ncangles|.  0 to 1
gives positions along the first segment, 1 to 2 along the second
segment, etc.
\medskip
3.  Ellipses, with nodes on their boundaries, are drawn  using the jTree
command |\jtenode|.  See Section \gettag[EllipseSec] for the
details.




