
\example



\hfil
%% ACTIVE CODE
\jtree[xunit=3em,yunit=1.4em]
\def\what{[which claim]$_k$}%
\! = {CP$_1$}
   <left>[scaleby=1.7]{CP$_2$}!a ^<right>[scaleby=2.5]{C$'$}
   :{$\rm was+Q$} {TP}
   <vartri>[scaleby=1.6,triratio=.4]
      {he$_i$ willing to discuss \what}.
\!a = :{\what} {CP$_2$}
   <vartri>[scaleby=1.6,triratio=.6]
      {that John$_i$ made \what}.
\endjtree
%% END ACTIVE CODE

\bigskip
(Nunes, Jairo. 2004.  {\sl Linearization of Chains
and Sideward Movement}, p.~149.)
\medskip

\codelines
|\jtree[xunit=3em,yunit=1.4em]
\def\what{[which claim]$_k$}%
\! = {CP$_1$}
   :[scaleby=1.7]{CP$_2$}!a [scaleby=2.5]{C$'$}||1
   :{$\rm was+Q$} {TP}
   <vartri>[scaleby=1.6,triratio=.4]||2
      {he$_i$ willing to discuss \what}.
\!a = :{\what} {CP$_2$}
   <vartri>[scaleby=1.6,triratio=.6]||2
      {that John$_i$ made \what}.
\endjtree
|endcodelines

\medskip
1.\ The two legs are not scaled equally because the structure
is not (linguistically) symmetrical.  Keeping the left branch
shorter helps visually to identify the high C-specifier.

\medskip
2.\ The settings of |triratio| allow for a much more compact
display.

