
\example
\ftag{\the\Exno}[ExBilink]

\def\bilink(#1,#2)(#3,#4){{%
   \psset{offset=.1ex,arcangle=6,arrows=->,nodesep=.8ex}
   \pcarc(#1,#2)(#3,#4)
   \pcarc[linestyle=dashed](#3,#4)(#1,#2)
}}
\newpsstyle{arrowsA}{arrowsize=.7ex,arrowlength=1.8,arrowinset=.35}

\exdisplay
\def\O{\pscircle{.8ex}}%
\hfil
\jtree[xunit=4em,yunit=4ex,branch=\bilink,
   style=arrowsA,labelgapt=1ex,bbadjust=left 3em depth 5ex]
\! = {\omit\O}@A1 <right>{\omit\O}@B1
   :{\omit\O}({C}) [scaleby=1.4]{\omit\O}@B2
   :{\omit\O}({see}) {\omit\O}@A2 {who}[labeloffset=.8em].
\psset{offset=.2ex,nodesep=.6ex,arrows=->}
\nccurve[angleA={(-1,-1)},angleB=-135,ncurvA=1.5,ncurvB=1.1]
   {A1}{A2}
\nccurve[angleA=-130,angleB={(-1,-.9)},ncurvA=1.2,ncurvB=1.55,
   linestyle=dashed]{A2}{A1}
\pcline[linestyle=none,offset=0]{-}(B1)(B2)
\ncput*[framesep=1.5ex]{\dots}
\endjtree
\xe

\codelines
|\def\bilink(#1,#2)(#3,#4){{%
   \psset{offset=.1ex,arcangle=6,arrows=->,nodesep=.8ex}
   \pcarc(#1,#2)(#3,#4)
   \pcarc[linestyle=dashed](#3,#4)(#1,#2)
}}
\def\O{\pscircle{.8ex}}%
\newpsstyle{arrowsA}{arrowsize=.7ex,arrowlength=1.8,
   arrowinset=.35}

|bigskip|relax
\jtree[xunit=4em,yunit=4ex,branch=\bilink,
   style=arrowsA,labelgapt=1ex]
\! = {\omit\O}@A1 <right>{\omit\O}@B1
   :{\omit\O}({C}) [scaleby=1.4]{\omit\O}@B2
   :{\omit\O}({see}) {\omit\O}@A2 {who}[labeloffset=.8em].
\psset{offset=.2ex,nodesep=.6ex,arrows=->}
\nccurve[angleA={(-1,-1)},angleB=-135,ncurvA=1.5,ncurvB=1.1]
   {A1}{A2}
\nccurve[angleA=-130,angleB={(-1,-.9)},ncurvA=1.2,ncurvB=1.55,
   linestyle=dashed]{A2}{A1}
\pcline[linestyle=none,offset=0,arrows=-](B1)(B2)
\ncput*[framesep=1.5ex]{\dots}
\endjtree

|endcodelines

\bigskip
Needless to say, a display likes this takes quite a bit of tuning
of the various parameters to get good result.




