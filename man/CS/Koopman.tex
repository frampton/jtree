\vfil\break

\example\hfil

\vskip-1em
\hfil
\def\scaleA{[scaleby=1.6 1]}%
\def\scaleB{[scaleby=.6 1,doubleline=true,doublesep=.1ex]}%
\def\mkovalnode{\rput(-1ex,-.8)
   {\ovalnode[framesep=\psxunit]{K}{\hskip2em}}}%
\jtree[xunit=3em,yunit=1.5em]
\def\scaleA{[scaleby=1.6 1]}%
\def\scaleB{[scaleby=.6 1,doubleline=true,doublesep=.1ex]}%
\def\mkovalnode{\rput(-1ex,-.8)
   {\ovalnode[framesep=\psxunit]{K}{\hskip2em}}}%
\! = {$\rm Agr'$}
   :\scaleA{Agr}!a \scaleA{\bf VP}
   <vert>[linestyle=dashed,linewidth=1pt]{\bf V}
   :{${\rm [_V\,e]}_i$}@A1 {T}.
\!a = :{V$_i$}!b {Agr}
   <vert>{[e]$_j$}@A2 .
\!b = {\omit\mkovalnode}
   :\scaleB{${\rm [_{Agr}\,Agr]}_j$}@A3 \scaleB{$\rm [_T\,T]$}.
\psset{angleA=-90,angleB=-45,arrows=->}
\nccurve[nodesepB=0]{A1}{K}
\nccurve[ncurv=1.3]{A2}{A3}
\endjtree

\vskip5em

(Koopman, Hilda. 1995.  On Verbs That Fail to Undergo V-Second,
{\sl Linguistic Inquiry\/} 26.1:150.)

\medskip
\codelines
|\jtree[xunit=3em,yunit=1.5em]
\def\scaleA{[scaleby=1.6 1]}%
\def\scaleB{[scaleby=.6 1,doubleline=true,doublesep=.1ex]}%
\def\mkovalnode{\rput(-1ex,-.8)
   {\ovalnode[framesep=\psxunit]{K}{\hskip2em}}}%
\! = {$\rm Agr'$}
   :\scaleA{Agr}!a \scaleA{\bf VP}
   <vert>[linestyle=dashed,linewidth=1pt]{\bf V}
   :{${\rm [_V\,e]}_i$}@A1 {T}.
\!a = :{V$_i$}!b {Agr}
   <vert>{[e]$_j$}@A2 .
\!b = {\omit\mkovalnode}
   :\scaleB{${\rm [_{Agr}\,Agr]}_j$}@A3 \scaleB{$\rm [_T\,T]$}.
\psset{angleA=-90,angleB=-45,arrows=->}
\nccurve[nodesepB=0]{A1}{K}
\nccurve[ncurv=1.3]{A2}{A3}
\endjtree
|endcodelines

\bigskip
The difficulty in this problem is drawing the oval and making it
a node, so that a node connection can point to the oval.
|\mkovalnode|, which is evaluated at the apex of the |!b|
subtree, does all the work.  |\rput(-1ex,-.8)| positions the
center of the oval slightly to the left of the apex and 80\% of
the way to the bottom of the two branches from the apex (since
they have height 1).  The size of the oval is determined by
\value{framesep} and box |\hbox{\hskip2em}|.  It required some
trial and error to get the various settings in |\mkovalnode|
adjusted to values which produced a suitable oval.

\bigskip
It is worth noting that this tree formatting can be adjusted to
different sizes and fonts without difficulty. This makes it
fairly easy to resize the tree to make the transition from a
draft paper to a book manuscript, or to move the tree from the
text to a footnote with a smaller font.

\smallskip
|\twelvepoint\jtree[xunit=4.8em,yunit=2em]|, without altering any
other code, produces\medskip
\hfill
\twelvepoint
\jtree[xunit=4.8em,yunit=2em]
\def\mkovalnode{\rput(-1ex,-.8){\ovalnode[framesep=\psxunit]{K}
   {\hskip2em}}}%
\! = {$\rm Agr'$}
   :\scaleA{Agr}!a \scaleA{\bf VP}
   <vert>[linestyle=dashed,linewidth=1pt]{\bf V}
   :{${\rm [_V\,e]}_i$}@A1 {T}.
\!a = :{V$_i$}!b {Agr}
   <vert>{[e]$_j$}@A2 .
\!b = {\omit\mkovalnode}
   :\scaleB{${\rm [_{Agr}\,Agr]}_j$}@A3 \scaleB{$\rm [_T\,T]$}.
\psset{angleA=-90,angleB=-45,arrows=->}
\nccurve[nodesepB=0]{A1}{K}
\nccurve[ncurv=1.3]{A2}{A3}
\endjtree

\vskip6em
|\tenpoint\jtree[xunit=2.6em,yunit=1em]|, again
without altering any other code, produces:\bigskip

\hfil \begingroup\tenpoint
\jtree[xunit=2.6em,yunit=1em]
\def\mkovalnode{\rput(-1ex,-.8){\ovalnode[framesep=\psxunit]{K}
   {\hskip2em}}}%
\! = {$\rm Agr'$}
   :\scaleA{Agr}!a \scaleA{\bf VP}
   <vert>[linestyle=dashed,linewidth=1pt]{\bf V}
   :{${\rm [_V\,e]}_i$}@A1 {T}.
\!a = :{V$_i$}!b {Agr}
   <vert>{[e]$_j$}@A2 .
\!b = {\omit\mkovalnode}
   :\scaleB{${\rm [_{Agr}\,Agr]}_j$}@A3 \scaleB{$\rm [_T\,T]$}.
\psset{angleA=-90,angleB=-45,arrows=->}
\nccurve[nodesepB=0]{A1}{K}
\nccurve[ncurv=1.3]{A2}{A3}
\endjtree
\vskip4em
\endgroup

The oval is somewhat too small, but that is easily fixed by
increasing |framesep|.



