
\example
\ftag{\the\Exno}[Dowty]

\hfil
\jtree[xunit=4em,yunit=4.4ex,everylabel=\it]
\! = {John walks from Boston to Detroit\rm, t, 4}
   :{John\rm, T} {walk from Boston
                     to Detroit\rm, IV,7}[labeloffset=1.5em]
   :\jtlong{walk from Boston\rm, IV,7}!a
      {to Detroit\rm, IV/IV, 5}
   :[scaleby=.8]{to\rm, IAV/T}() [scaleby=.8]{Detroit\rm, T}.
\!a = :{walk\rm, IV} {from Boston\rm, IV/IV, 5}
   :{from\rm, IAV/T} {Boston\rm, T}.
\endjtree
\vskip1.5em

(Dowty, David.  1979.  {\sl Word Meaning and Montague Grammar},
p.~215.)

\medskip
\codelines
|\jtree[xunit=4em,yunit=4.4ex,everylabel=\it]
\! = {John walks from Boston to Detroit\rm, t, 4}
   :{John\rm, T} {walk from Boston
                     to Detroit\rm, IV,7}[labeloffset=1.5em]||1
   :\jtlong{walk from Boston\rm, IV,7}!a
      {to Detroit\rm, IV/IV, 5}
   :[scaleby=.8]{to\rm, IAV/T}() [scaleby=.8]{Detroit\rm, T}.||2
\!a = :{walk\rm, IV} {from Boston\rm, IV/IV, 5}
   :{from\rm, IAV/T} {Boston\rm, T}.
\endjtree
|endcodelines

\vfil
\bigskip
1. The label is offset to improve the spacing.  The
label is off center, but not so much as to be a
distraction.
\medskip

2. The null inline adjunction |()| keeps
|[scaleby=.8]| from being interpreted as a
parameter associated with the preceding label.

