\hfil
\jtree[xunit=2.2em,yunit=.7em]
\def\\{[labelgapb=-.5ex]}%
\! = {S}
   :({NP}\\{that book}@A1 ) \jtwide{S/NP}
   :({NP}\\{I}) {VP/NP}
   :({V}\\{want}) {VP/NP}
   :{\it to} {VP/NP}
   <left>({V}\\{ask}) ^<vert>({NP}\\{Mary}) ^<wideright>{VP/NP}
   :{\it to} {VP/NP}
   <left>({V}\\{tell}) ^<vert>({NP}\\{Tom}) ^<wideright>{VP/NP}
   :{\it to} {VP/NP}
   :({V}\\{read}) {NP/NP}\\{e}@A2 .
\nccurve[angleA=210,angleB=-90]{->}{A2}{A1}
\endjtree

\vskip3em
(This is an adaptation of an example in the documentation to
Avery Andrew's tree formatting preprocessor.)

Some may prefer the following formatting style.  The code is
almost identical to the code for the style above.  |\\| has a
different definition and |<vartri>| is used instead of |\\|
in one place.\bigskip

\hfil
\jtree[xunit=2.2em,yunit=.7em]
\def\\{<shortvert>}%
\! = {S}
   :({NP}<vartri>{that book}@A1 ) \jtwide{S/NP}
   :({NP}\\{I}) {VP/NP}
   :({V}\\{want}) {VP/NP}
   :{\it to} {VP/NP}
   <left>({V}\\{ask}) ^<vert>({NP}\\{Mary}) ^<wideright>{VP/NP}
   :{\it to} {VP/NP}
   <left>({V}\\{tell}) ^<vert>({NP}\\{Tom}) ^<wideright>{VP/NP}
   :{\it to} {VP/NP}
   :({V}\\{read}) {NP/NP}\\{e}@A2 .
\nccurve[angleA=210,angleB=-90]{->}{A2}{A1}
\endjtree


