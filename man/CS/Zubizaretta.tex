
\vfil\break

\example

\ftag{\the\Exno}[Zubi]
\hfil
%% ACTIVE CODE
\jtree[xunit=2.5em,yunit=2em]
\def\ovalstuff{\vtop{\hbox{arg$^3$}%
   \hbox to 4em{(\thinspace \leaders\hrule\hfil\ NP)}}}%
\! = {V}@B1
   <left>{face}({arg$^1$}@B2 )
      ^<right>{$\rm [_V\,$leggere]}({arg$^2$,})
      ^<right>[scaleby=3 1,branch=\blank]
         {}{\ovalnode[framesep=1ex,boxsep=false]{K}{\ovalstuff}}.
\psset{arrows=->,nodesepA=0}
\nccurve[angleA=150,angleB=180,ncurv=1.2]{B2}{B1}
\nccurve[angleA=90,angleB=0,ncurv=.8]{K}{B1}
\endjtree
%% END ACTIVE CODE
\kern6em

\vskip2.5em
(Zubizaretta, Maria Luisa. 1985.
Morphology and Morphosyntax: Romance Causatives,
{\sl Linguistic Inquiry\/} 16.2, p.~276.)

\medskip
\codelines
|\jtree[xunit=2.5em,yunit=2em]
\def\ovalstuff{\vtop{\hbox{arg$^3$}%
   \hbox to 4em{(\thinspace \leaders\hrule\hfil\ NP)}}}%
\! = {V}@B1
   <left>{face}({arg$^1$}@B2 )
   ^<right>{$\rm [_V\,$leggere]}({arg$^2$,})
   ^<right>[scaleby=3 1,branch=\blank]
      {}{\ovalnode[framesep=1ex,boxsep=false]{K}{\ovalstuff}}.
\psset{arrows=->,nodesepA=0}
\nccurve[angleA=150,angleB=180,ncurv=1.2]{B2}{B1}
\nccurve[angleA=90,angleB=0,ncurv=.8]{K}{B1}
\endjtree
|endcodelines
\medskip
In order to see how this tree is put together, consider first:
\medskip
\hfil\jtree[xunit=2.5em,yunit=2em]
\! = {V}
   <left>{face}({arg$^1$})
   ^<right>{$\rm [_V\,$leggere]}({arg$^2$})
   ^<right>[scaleby=3 1]
      {X}{arg$^3$}.
\endjtree
\bigskip
\codelines
|\jtree[xunit=2.5em,yunit=2em]
\! = {V}
   <left>{face}({arg$^1$})
   ^<right>{$\rm [_V\,$leggere]}({arg$^2$})
   ^<right>[scaleby=3 1]{X}
   {arg$^3$}.
\endjtree
|endcodelines
\medskip
Then:
\medskip
\hfil
\jtree[xunit=2.5em,yunit=2em]
\def\ovalstuff{\vtop{\hbox{arg$^3$}%
   \hbox to 4em{(\thinspace \leaders\hrule\hfil\ NP)}}}%
\! = {V}
   <left>{face}({arg$^1$})
   ^<right>{$\rm [_V\,$leggere]}({arg$^2$})
   ^<right>[scaleby=3 1,branch=\blank]{}
      {\ovalstuff}.
\endjtree

\medskip
\codelines
|\jtree[xunit=2.5em,yunit=2em]
\def\ovalstuff{\vtop{\hbox{arg$^3$}%
   \hbox to 4em{(\thinspace \leaders\hrule\hfil\ NP)}}}%
\! = {V}
   <left>{face}({arg$^1$})
   ^<right>{$\rm [_V\,$leggere]}({arg$^2$})
   ^<right>[scaleby=3 1,branch=\blank]{}
      {\ovalstuff}.
\endjtree
|endcodelines

\medskip
Finally, the oval node is built around |\ovalstuff| using
|\ovalnode|.  The syntax is:
\smallskip \hfil
|\ovalnode[�pars\/�]{�name\/�}{�stuff\/�}|
\smallskip
with the parameters optional.  The box is enlarged on all sides
by \value{framesep} and an oval drawn through the four corners of
the resulting box.  If \value{boxsep} is {\sl false},
the node construction is invisible to Tex, otherwise |\ovalnode|
creates a Tex box the size of the oval.

\medskip
The parameter settings in the use of |\ovalnode| in the example
are crucial.  If \value{boxsep} were not {\sl false}, the
alignment would be disrupted.
%
$\vert \sl framesep\vert=\rm 1\,ex$ means that the oval is drawn
around the box with a separation of $\rm 1\,ex$, leaving a
visually important gap between the oval and the box it surrounds.

