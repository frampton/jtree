
\example

In order to show that right node raising does not in general
apply to constituents, Chris Wilder gave the example (from
German):

\medskip
{\leftskip=1.2em \rightskip=\leftskip
Er hat einen Mann, der drei, und sie hat eine Frau,
   die vier,\par \hfill Katzen besitzt, gekannt.\medskip}

In my own work on right node raising, I have had occasion to
typeset the following.  It may provide some useful things for
\jTree\ users, so it is included here.
\bigskip

\hfil
\jtree[dirA=(1:-1),nodesepA=0,nodesepB=.8ex,style=arrows2]
\! = :!a {\rnode{K1}{knew}}.
\!a = :!b {\rnode{O1}{owned}}.
\!b = :!c {\rnode{C1}{cats}}.
\!c =
   :\jtlong !d [scaleby=1.8]
   :{and}() [scaleby=2.4]
   :{he}() @K2
   <left>\jtjot !e .
\!d =
   :{she}() @K3
   <left>\jtjot !f .
\!e =
   :{a woman}[labeloffset=-1ex]
   :{who}() @O2
   <left>@C2
   <left>{four}.
\!f =
   :{a man}
   :{who}() @O3
   <left>@C3
   <left>{three}.
\psset{linestyle=dashed,arrows=<-}
\nccurve[angleB=-10,ncurvB=2,ncurvA=1.2]{O2}{O1}
\nccurve[angleB=-90,ncurvA=1.4]{O3}{O1}
\nccurve[angleB=-10,ncurvB=1.8,ncurvA=1.6]{K2}{K1}
\nccurve[angleB=-90,ncurvA=1.4]{K3}{K1}
\nccurve[angleB=-90,ncurvA=1.4]{C3}{C1}
\nccurve[angleB=-10,ncurvB=1.8,ncurvA=1.6]{C2}{C1}
\endjtree\kern6em

\codelines|
\jtree[dirA=(1:-1),nodesepA=0,nodesepB=.8ex,arrows=->,
   arrowlength=3.6,arrowsize=2pt,arrowinset=.4]
\! = :!a {\rnode{K1}{knew}}.
\!a = :!b {\rnode{O1}{owned}}.
\!b = :!c {\rnode{C1}{cats}}.
\!c =
   :\jtlong !d [scaleby=1.8]
   :{and}() [scaleby=2.4]
   :{he}() @K2
   <left>\jtjot !e .
\!d =
   :{she}() @K3
   <left>\jtjot !f .
\!e =
   :{a woman}[labeloffset=-1ex]
   :{who}() @O2
   <left>@C2
   <left>{four}.
\!f =
   :{a man}
   :{who}() @O3
   <left>@C3
   <left>{three}.
\psset{linestyle=dashed,arrows=<-}
\nccurve[angleB=-10,ncurvB=2,ncurvA=1.2]{O2}{O1}
\nccurve[angleB=-90,ncurvA=1.4]{O3}{O1}
\nccurve[angleB=-10,ncurvB=1.8,ncurvA=1.6]{K2}{K1}
\nccurve[angleB=-90,ncurvA=1.4]{K3}{K1}
\nccurve[angleB=-90,ncurvA=1.4]{C3}{C1}
\nccurve[angleB=-10,ncurvB=1.8,ncurvA=1.6]{C2}{C1}
\endjtree\kern6em
|endcodelines
