\vfil\break
% Chung example (p. 246)

\example
\ftag{\the\Exno}[Chung2]
\vskip-1ex
\hfil
\jtree[xunit=2em,yunit=1.4em,labelgapb=0,triratio=0,
   arrowscale=1.6 1]
\deftriangle<tri>(1.8)(1)(-.5)
\defbranch<colonB>(1)(-.5)
\! = {CP}
   :{\sc WH}@A {C$'$}
   <tri>{\rlap{V}}@AA ^<tri>[triratio=.55]{CP}
   :{\it t}@B {C$'$}
   <tri>{\rlap{V}}@BB ^<tri>[triratio=.55]{CP}
   :{\it t}@C {C$'$}
   <tri>{\rlap{V}}@CC .
\psset{linewidth=1pt,ncurvB=1.1,nodesepA=1ex,
   angleA=-90,angleB=180,offsetA=.5ex}
\nccurve{A}{AA}
\nccurve{B}{BB}
\nccurve{C}{CC}
\psset{offsetA=-.5ex,arrows=->}
\nccurve{A}{AA}
\nccurve{B}{BB}
\nccurve{C}{CC}
\endjtree

\vskip1.2em
(Chung, Sandy. 1998. {\sl The Design of Agreement}, p.~246.)

\medskip
\codelines
|\jtree[xunit=2em,yunit=1.4em,labelgapb=0,triratio=0]||1
\deftriangle<tri>(1.8)(1)(-.5)
\defbranch<colonB>(1)(-.5)
\! = {CP}
   :{\sc WH}@A {C$'$}
   <tri>{\rlap{V}}@AA ^<tri>[triratio=.55]{CP}||2
   :{\it t}@B {C$'$}
   <tri>{\rlap{V}}@BB ^<tri>[triratio=.55]{CP}
   :{\it t}@C {C$'$}
   <tri>{\rlap{V}}@CC .
\psset{linewidth=1pt,ncurvB=1.1,nodesepA=1ex,
   angleA=-90,angleB=180,offsetA=.5ex}
\nccurve{A}{AA}
\nccurve{B}{BB}
\nccurve{C}{CC}
\psset{offsetA=-.5ex,arrows=->}
\nccurve{A}{AA}
\nccurve{B}{BB}
\nccurve{C}{CC}
\endjtree

|endcodelines

\bigskip
1.\ Since all the node labels are uppercase, and therefore do not
descend below the baseline, spacing is improved by |labelgapb=0|.
|triratio| is set to 0 in order to properly position the
instances of V.

\medskip
2.\ The triangle is overwritten so that both V and CP can be
positioned independently, using different values of |triratio|.

\bigskip
Here is a less elegant approach which achieves the same effect without
using an~offset.

\medskip
\codelines
|\jtree[xunit=2em,yunit=1.4em,labelgapb=0,triratio=0]
\deftriangle<tri>(1.8)(1)(-.5)
\defbranch<colonB>(1)(-.5)
\def\gap{\hskip1ex}%
\def\\{[labelstrutb=0]}%
\! = {CP}
   :{\sc WH}\\({\pnode{A1}\gap\pnode{A2}}) {C$'$}||1
   <tri>{\rlap{V}}@A ^<tri>[triratio=.55]{CP}
   :{\it t}\\({\pnode{B1}\gap\pnode{B2}}) {C$'$}
   <tri>{\rlap{V}}@B ^<tri>[triratio=.55]{CP}
   :{\it t}\\({\pnode{C1}\gap\pnode{C2}}) {C$'$}
   <tri>{\rlap{V}}@C .
\psset{linewidth=1pt,ncurvB=1.1,nodesepA=1ex,
   angleA=-90,angleB=180}
\nccurve{A1}{A}
\nccurve{B1}{B}
\nccurve{C1}{C}
\psset{arrows=->}
\nccurve{A2}{A}
\nccurve{B2}{B}
\nccurve{C2}{C}
\endjtree
|endcodelines

\medskip
1.\ It is necessary to set |labelstrutb| to 0 (via |\\|) so that
the inline adjunction is to the bottom of the label box.
|labelgapb| is 0 throughout the tree.

\endinput

\jtree[xunit=2em,yunit=1.4em,labelgapb=0,triratio=0]
\deftriangle<tri>(1.8)(1)(-.5)
\defbranch<colonB>(1)(-.5)
\def\gap{\hskip1ex}%
\def\\{[labelstrutb=0]}%
\! = {CP}
   :{\sc WH}\\({\pnode{A1}\gap\pnode{A2}}) {C$'$}
   <tri>{\rlap{V}}@A ^<tri>[triratio=.55]{CP}
   :{\it t}\\({\pnode{B1}\gap\pnode{B2}}) {C$'$}
   <tri>{\rlap{V}}@B ^<tri>[triratio=.55]{CP}
   :{\it t}\\({\pnode{C1}\gap\pnode{C2}}) {C$'$}
   <tri>{\rlap{V}}@C .
\psset{linewidth=1pt,ncurvB=1.1,nodesepA=1ex,
   angleA=-90,angleB=180}
\nccurve{A1}{A}
\nccurve{B1}{B}
\nccurve{C1}{C}
\psset{arrows=->}
\nccurve{A2}{A}
\nccurve{B2}{B}
\nccurve{C2}{C}
\endjtree



