
\example
\ftag{\the\Exno}[Frampton2]

\exdisplay
\hfil
\jtree[xunit=2.4em,yunit=3ex,style=arrows2,nodesep=0]
\def\broken{[branch=\brokenbranch,scaleby=1.6]}%
\! = {\elc{\it a}}@A1
   <right>{\elc{\it b}}@A1a
   :{C$_2$}()  \broken @A2
   <right>@A3
   <right>@A3a
   :{C$_1$}()  \broken
   :{ubil}  {\elc{\it c}}@A4
   :{kolko}  {\elc{\it d}}
   :{studenti}  {\elc{\it e}}@A5
   :{ot} :{koi} {strani}.
\psLNode(A1)(A1a){.5}{K1}
\psLNode(A2)(A3){.5}{K2}
\psLNode(A3)(A3a){.5}{K3}
\psset{angleA={(1,1)},angleB=90,ncurvA=.6,ncurvB=1}
\nccurve{K1}{A5}
\nccurve[arrows=-]{K2}{A5}
\nccurve{K3}{A4}
\psset{angleA={(-1,-1)},angleB={(-1,-1)},ncurv=4,arrows=-}
\nccurve{A1}{K1}
\nccurve{A2}{K2}
\nccurve{A3}{K3}
\endjtree
\xe

Before we tackle the display above, consider the simpler one
below.

\exh \tac|
\jtree[xunit=3em,yunit=3ex,nodesep=0]
\! = @Q1 <right>@Q2
   <left>{A} ^<right>@Q3
   <left>{B} ^<right>{C}.
\psLNode(Q1)(Q2){.5}{Q12}
\nccurve[angleA={(-1,-1)},
   angleB={(-1,-1)},ncurv=3]{Q1}{Q12}
\nccurve[angleA={(1,1)},angleB=90,
   ncurv=1]{Q12}{Q3}
\endjtree
|endtac \hfil
\qquad\jtree[xunit=3em,yunit=3ex,nodesep=0]
\! = @Q1 <right>@Q2
   <left>{A} ^<right>@Q3
   <left>{B} ^<right>{C}.
\psLNode(Q1)(Q2){.5}{Q12}
\nccurve[angleA={(-1,-1)},
   angleB={(-1,-1)},ncurv=3]{Q1}{Q12}
\nccurve[angleA={(1,1)},angleB=90,
   ncurv=1]{Q12}{Q3}
\endjtree
\xe

The curve is drawn in two segments which join at the midpoint of
the highest right branch.  That point is found by |\psLNode|,
which interpolates a point between the two ends of that branch.
The angle at which the first segment leaves its starting point is the same
as the angle at which left branches leave their parent nodes, hence
|angleA={(-1,-1)}|.  The angle at which the first segment leaves
the finish point of the segment is the same.  (One might expect
that |angleB| would be the negative of |angleA|, but that is not
the PSTricks convention.  |angleA| and |angleB| are independent
of the direction that the segment is drawn in.) |ncurv| is set to a large value in
order to get the first segment to bow out sufficiently.  The fact
that |angleA| for the second segment is the opposite of |angleB|
for the first curve means that the two segments join smoothly.
By ``opposite'' I mean that the two angles differ by $180^\circ$.

\medskip
We are now ready for the code for the complex display.

\exdisplay
\codelines
|\jtree[xunit=2.4em,yunit=3ex,style=arrows2,nodesep=0]
\def\broken{[branch=\brokenbranch,scaleby=1.6]}%
\! = {\elc{\it a}}@A1
   <right>{\elc{\it b}}@A1a
   :{C$_2$}()  \broken @A2
   <right>@A3
   <right>@A3a
   :{C$_1$}()  \broken
   :{ubil}  {\elc{\it c}}@A4
   :{kolko}  {\elc{\it d}}
   :{studenti}  {\elc{\it e}}@A5
   :{ot} :{koi} {strani}.
\psLNode(A1)(A1a){.5}{K1}
\psLNode(A2)(A3){.5}{K2}
\psLNode(A3)(A3a){.5}{K3}
\psset{angleA={(1,1)},angleB=90,ncurvA=.6,ncurvB=1}
\nccurve{K1}{A5}
\nccurve[arrows=-]{K2}{A5}
\nccurve{K3}{A4}
\psset{angleA={(-1,-1)},angleB={(-1,-1)},ncurv=4,arrows=-}
\nccurve{A1}{K1}
\nccurve{A2}{K2}
\nccurve{A3}{K3}
\endjtree|endcodelines
\xe

\medskip
1. See Section~\gettag[CustomSec] for discussion of the
|branch| parameter.

\smallskip
2. The empty adjunction |()| is used so that |\broken| is not
parsed as parameters for the preceeding label.

\endinput




\bigskip
An alternate to using |\stub| to position the crossing points is
|\psinterpolate|.

\medskip
\codelines
|\jtree[xunit=2.4em,yunit=1.2em,arrows=->,nodesep=0,
   arrowlength=3.6,arrowsize=2pt,arrowinset=.4]
\def\broken{[branch=\brokenbranch,scaleby=1.6]}%
\def\\#1{\rput[bl](.6ex,.4ex){\it #1}}%
\! = {\omit\\a}@A1
   <right>{\omit\\b}@A1a
   :{C$_2$}()  \broken @A2
   <right>@A3
   <right>@A3a
   :{C$_1$}()  \broken
   :{ubil}  {\omit\\c}@A4
   :{kolko}  {\omit\\d}
   :{studenti}  {\omit\\e}@A5
   :{ot} :{koi} {strani}.
\psinterpolate(A1)(A1a){.5}{K1}
\psinterpolate(A2)(A3){.5}{K2}
\psinterpolate(A3)(A3a){.5}{K3}
|endcodelines
\smallskip
\dots\ continues as above


