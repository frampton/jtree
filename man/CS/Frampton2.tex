
\example

\ftag{\the\Exno}[Frampton2]
\bigskip
\hfil
\jtree[xunit=2.4em,yunit=1.2em,style=arrows2,nodesep=0]
\def\broken{[branch=\brokenbranch,scaleby=1.6]}%
\def\stub{<right>[scaleby=.5,arrows=-]}%
\def\\#1{\rput[bl](.6ex,.4ex){\it #1}}%
\! = {\omit\\a}@A1
   \stub @K1  ^<right>{\omit\\b}
   :{C$_2$}()  \broken @A2
   \stub @K2  ^<right>@A3
   \stub @K3  ^<right>
   :{C$_1$}()  \broken
   :{ubil}  {\omit\\c}@A4
   :{kolko}  {\omit\\d}
   :{studenti}  {\omit\\e}@A5
   :{ot} :{koi} {strani}.
\psset{dirA=(1:1),angleB=90,ncurvA=.6,ncurvB=1}
\nccurve{K1}{A5}
\nccurve{-}{K2}{A5}
\nccurve{K3}{A4}
\psset{dirA=(-1:-1),dirB=(-1:-1),ncurv=4,arrows=-}
\nccurve{A1}{K1}
\nccurve{A2}{K2}
\nccurve{A3}{K3}
\endjtree

\vskip1.5em
\codelines
|\jtree[xunit=2.4em,yunit=1.2em,arrows=->,nodesep=0]
\def\broken{[branch=\brokenbranch,scaleby=1.6]}%||1
\def\stub{<right>[scaleby=.5,arrows=-]}%||2
\def\\#1{\rput[bl](.6ex,.4ex){\it #1}}%||3
\! = {\omit\\a}@A1
   \stub @K1  ^<right>{\omit\\b}
   :{C$_2$}()  \broken @A2
   \stub @K2  ^<right>@A3
   \stub @K3  ^<right>
   :{C$_1$}()  \broken
   :{ubil}  {\omit\\c}@A4
   :{kolko}  {\omit\\d}
   :{studenti}  {\omit\\e}@A5
   :{ot} :{koi} {strani}.
\psset{dirA=(1:1),angleB=90,ncurvA=.6,ncurvB=1}||4
\nccurve{K1}{A5}
\nccurve{-}{K2}{A5}
\nccurve{K3}{A4}
\psset{dirA=(-1:-1),dirB=(-1:-1),ncurv=4,arrows=-}||5
\nccurve{A1}{K1}
\nccurve{A2}{K2}
\nccurve{A3}{K3}
\endjtree
|endcodelines

\medskip
1.\ See Section~\gettag[customsec] for information about the
|branch| parameter.

\smallskip
2.\ |\stub| is used to position nodes halfway down certain right
branches to facilitate drawing the complex (2~segment) pointers.

\smallskip
3.\ |\\| is used to position the tags {\sl a, b, c,} \dots

\smallskip
4.\ |dirA| is used in order to ensure that the complex pointers
are parallel to left branches at the point that they cross the
right branches.

\smallskip
5.\ A very high value of |ncurv| is used in order to get the loop
to bow out sufficiently.
\bigskip
An alternate to using |\stub| to position the crossing points is
|\psinterpolate|.

\medskip
\codelines
|\jtree[xunit=2.4em,yunit=1.2em,arrows=->,nodesep=0,
   arrowlength=3.6,arrowsize=2pt,arrowinset=.4]
\def\broken{[branch=\brokenbranch,scaleby=1.6]}%
\def\\#1{\rput[bl](.6ex,.4ex){\it #1}}%
\! = {\omit\\a}@A1
   <right>{\omit\\b}@A1a
   :{C$_2$}()  \broken @A2
   <right>@A3
   <right>@A3a
   :{C$_1$}()  \broken
   :{ubil}  {\omit\\c}@A4
   :{kolko}  {\omit\\d}
   :{studenti}  {\omit\\e}@A5
   :{ot} :{koi} {strani}.
\psinterpolate(A1)(A1a){.5}{K1}
\psinterpolate(A2)(A3){.5}{K2}
\psinterpolate(A3)(A3a){.5}{K3}
|endcodelines
\smallskip
\dots\ continues as above


