

\example
\ftag{\the\Exno}[Richards]
\ftagpage[Richardspage]

\hfil
\jtree[xunit=1.5em,yunit=1.1em,labelgap=0]
\def\A#1{\pnode(0,.3){A#1}}%
\def\B#1{\pnode(0,-.3){B#1}}%
\! = :{car} {\omit\A1}
   :({\omit\A0}{Op})
   :{that}
   :{I}
   :{T}
   :{know} {\omit\B1}
   :({\omit\B0}{who})
   :{C}
   :{Pro}
   :{to}() \jtjot
   :{persuade}() \jtjot
   :{V} {\omit\A2}
   :\jtlong{\omit\A4}!a
   :{Pro}
   :{to}
   :{talk} {\omit\B2}
   :\jtlong{\omit\B3}!b
   :{V}
   :{about}() {\omit\A3}{\it t}.
\!a = :{owners}() \jtjot
   :{of} {\omit\A5}{\it t}.
\!b = :{to} {\omit\B4}{\it t}.
\psset{linestyle=dashed,linewidth=.3ex,linecolor=blue,nodesep=0}
\def\fudge{.5}%
\ncline[nodesepA=-\fudge]{A0}{A1}
\ncline[nodesepB=-\fudge]{A1}{A3}
\ncline{A2}{A4}
\ncline[nodesepB=-\fudge]{A4}{A5}
\psset{linestyle=dotted,linewidth=.5ex,linecolor=red}
\ncline[nodesepA=\fudge]{B0}{B1}
\ncline{B1}{B2}
\ncline{B2}{B3}
\ncline[nodesepB=\fudge]{B3}{B4}
\endjtree

\vskip1.5em
(Richards, Norvin.  2001. {\sl Movement in Language: Interactions
and Architectures}, p.~262.)

\medskip
\codelines
|\jtree[xunit=1.5em,yunit=1.1em,labelgap=0]
\def\A#1{\pnode(0,.3){A#1}}%
\def\B#1{\pnode(0,-.3){B#1}}%
\! = :{car} {\omit\A1}
   :({\omit\A0}{Op})
   :{that}
   :{I}
   :{T}
   :{know} {\omit\B1}
   :({\omit\B0}{who})
   :{C}
   :{Pro}
   :{to}() \jtjot
   :{persuade}() \jtjot
   :{V} {\omit\A2}
   :\jtlong{\omit\A4}!a
   :{Pro}
   :{to}
   :{talk} {\omit\B2}
   :\jtlong{\omit\B3}!b
   :{V}
   :{about}() {\omit\A3}{\it t}.
\!a = :{owners}() \jtjot
   :{of} {\omit\A5}{\it t}.
\!b = :{to} {\omit\B4}{\it t}.
\psset{linestyle=dashed,linewidth=.3ex,
   linecolor=blue,nodesep=0}
\def\fudge{.5}%
\ncline[nodesepA=-\fudge]{A0}{A1}
\ncline[nodesepB=-\fudge]{A1}{A3}
\ncline{A2}{A4}
\ncline[nodesepB=-\fudge]{A4}{A5}
\psset{linestyle=dotted,linewidth=.5ex,linecolor=red}
\ncline[nodesepA=\fudge]{B0}{B1}
\ncline{B1}{B2}
\ncline{B2}{B3}
\ncline[nodesepB=\fudge]{B3}{B4}
\endjtree
|endcodelines

\bigskip
Some adjustments were made after the first proof was examined.
|xunit| was adjusted so that the display was as wide as possible,
|\jtjot| was used to slightly lengthen a few branches in order to
improve the spacing, and |\fudge| was introduced in order to
adjust the end points of the dotted and dashed paths.

\medskip
A solution using |offset| in drawing the paths is also possible,
but it is much more difficult to ensure that the segments join
smoothly.

