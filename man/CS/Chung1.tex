
% Chung examples

\example

\exdisplay[xunit=4em,yunit=4ex]
\hfil a.\quad
\jtree
\! = {CP}
   <left>@A1 ^<tri>[triratio=.65]{CP}
   <left>@A2 ^<tri>[triratio=.65]{CP}
   <tri>{\it wh}@A3 .
\psset{angleB=-90,arrows=->}
\nccurve[angleA=190,ncurv=1.3]{A3}{A2}
\nccurve[angleA=160]{A2}{A1}
\endjtree
\hfil
b.\quad
\jtree
\! = {CP}
   <left>@A1 ^<tri>[triratio=.65]{CP}
   <tri>[triratio=.65]{CP}
   <tri>{\it wh}@A3 .
\nccurve[angleA=190,angleB=-90,ncurv=1.3]{->}{A3}{A1}
\endjtree
\xe

\vskip1.5em
(Chung, Sandy. 1998. {\sl The Design of Agreement}, p.~365.)

\medskip
\codelines
|\jtree[xunit=4em,yunit=4ex]
\! = {CP}
   <left>@A1 ^<tri>[triratio=.65]{CP}||1
   <left>@A2 ^<tri>[triratio=.65]{CP}
   <tri>{\it wh}@A3 .
\psset{angleB=-90,arrows=->}
\nccurve[angleA=190,ncurv=1.3]{A3}{A2}
\nccurve[angleA=160]{A2}{A1}
\endjtree
|bigskip
\jtree[xunit=4em,yunit=4ex]
\! = {CP}
   <left>@A1 ^<tri>[triratio=.65]{CP}
   <tri>[triratio=.65]{CP}
   <tri>{\it wh}@A3 .
\nccurve[angleA=190,angleB=-90,ncurv=1.3]{->}{A3}{A1}
\endjtree
|endcodelines

\bigskip
1.\ The |<left>...^<tri>| construction is used so that the @tag
following |<left>| and the label following |<tri>| can be
independently positioned.  The triangle simply overwrites the
left branch.  In other situations, something like
|<left>[branch=\blank]| might be appropriate.


