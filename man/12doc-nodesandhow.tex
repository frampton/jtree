
\section Nodes and connections between them

\ftag{\the\secno}[nodesandhow]%
Nodes have a shape, a reference point, and a name. \pstnode\
allows the user to define box, elliptical, circular, and point
nodes.  In addition to a shape, nodes have a reference point.  It
is at the center of an elliptical, circular, or point node.  It
can be at the center of a box node, but there are other options
for box nodes: the ends and centers of the edges and the
baseline.  \pstnode\ has a number of commands for drawing curves
of various kinds between nodes.  Various features of how these
curves are rendered (linewidth, linestyle, arrows, etc.) can be
specified.  The most useful curve drawing command for annotating
tree structures is |\nccurve|.  The main point of this section is
to explain how |\nccurve| works.

\subsection \sl nccurve

Every PSTricks node has a shape and a reference point.Suppose
there is a box node named A1 and a point node named A2 as shown
below. The dots are the reference points and the border of the
box node is indicated by a dashed line.

\newpsstyle{how}{angleA=50,angleB=110,unit=2cm,ncurv=.67,offset=0,nodesep=0}
\def\goop{\rnode[lB]{A1}{\psframebox[boxsep=false,framesep=0,
   linecolor=black,linestyle=dashed,linewidth=.4pt]
   {\vrule width0pt height1\psunit depth.5\psunit
      \hskip2\psunit}}}

\exdisplay[style=how]
\qquad
\pspicture(0,-0)(4.6,1.5)%\psgrid
\rput[bl](0,0){\goop}
\pnode(4,0){A2}
\psdot(A1)
\rput[bl](A1){\kern.5em $\fam0 \scriptstyle A1$}
\psdot(A2)
\rput[bl](A2){\kern.5em $\fam0 \scriptstyle A2$}
\endpspicture
\xe

PSTricks has a range of macros which can be used to draw
connections between nodes.  |nccurve| is the most important.  Its
use is illustrated below.

\medskip
|\nccurve[angleA=50,angleB=110]{A1}{A2}|

\exdisplay[style=how]
\qquad
\pspicture(0,-0)(4.6,2.4)%\psgrid
\rput[bl](0,0){\goop}
\pnode(4,0){A2}
\rput(A1){\psline[linewidth=.4pt,linecolor=gray](0,0)(2.4;50)
   \psline[linewidth=.4pt,linecolor=gray](0,0)(.6,0)
   \psarc[linewidth=.4pt,arrowsize=3pt]{->}(0,0){.4}{0}{50}
   \rput(.7;25){$\scriptstyle 50^\circ$}}
\rput(A2){\psline[linewidth=.4pt,linecolor=gray](0,0)(2.5;110)
   \psline[linewidth=.4pt,linecolor=gray](0,0)(.6,0)
   \psarc[linewidth=.4pt,arrowsize=3pt]{->}(0,0){.4}{0}{110}
   \rput(.7;45){$\scriptstyle 110^\circ$}}
\psdot(A1)
\psdot(A2)
\nccurve[nodesep=0,angleA=50,angleB=110]{A1}{A2}
\endpspicture
\xe

This relatively simple picture can be modified by a number of
parameters. In addition to the usual parameters like |linewidth|,
|linestyle| and |arrows| which determine how a geometrical curve
is rendered, there are six parameters which modify the geometry
of the curve itself: |ncurvA|, |ncurvB|, |nodesepA|, |nodesepB|,
|offsetA|, and |offsetB|.  The parameters can be set individually
or in pairs.  |\psset{nodesep=�x\/�}| induces
|\psset{nodesepA=�x\/�,nodesepB=�x\/�}|.  |ncurv| and |offset|
work the same way. For the display above, the settings are what
is produced by |\psset{ncurv=.67,offset=0,nodesep=0}|.  The
displays which follow in the next three sections which show how
the display above is modified by changing one or the other of
these parameters.
\medskip

\subsubsection \sl nodesep

|\psset{nodesepA=�x\/�}| causes the dimensions of the node shape
at the starting node of the connection to be to be adjusted by
{\sl x\/} before the curve is determined.  |nodesepB| affects the
finishing point.  Point nodes are adjusted to be disks.

\bigskip
|\nccurve[angleA=50,angleB=110,nodesepA=�$x$�]{A1}{A2}|
\par\nobreak\smallskip
for $x=0$ and $x=.3\,$.\par\nobreak

\exdisplay[style=how]
\qquad\pspicture(0,-.5)(4.6,2.4)%\psgrid
\rput[bl](0,0){\goop}
\pnode(4,0){A2}
\rput(A1){\psline[linewidth=.4pt,linecolor=gray](0,0)(2.4;50)
   \psline[linewidth=.4pt,linecolor=gray](0,0)(.6,0)
   \psarc[linewidth=.4pt,arrowsize=3pt]{->}(0,0){.4}{0}{50}
   \rput(.7;25){$\scriptstyle 50^\circ$}}
\rput(A2){\psline[linewidth=.4pt,linecolor=gray](0,0)(2.5;110)
   \psline[linewidth=.4pt,linecolor=gray](0,0)(.6,0)
   \psarc[linewidth=.4pt,arrowsize=3pt]{->}(0,0){.4}{0}{110}
   \rput(.7;45){$\scriptstyle 110^\circ$}}
\psdot(A1)
\psdot(A2)
\rput(A1){\psframe[boxsep=false,framesep=0,
   linecolor=black,linestyle=dotted,linewidth=.8pt](-.3,-.8)(2.3,1.3)}
\nccurve[nodesepA=0]{A1}{A2}
\naput[npos=.02,labelsep=1pt]{$\scriptstyle 0$}
\nccurve[nodesepA=.3]{A1}{A2}
\naput[npos=.02,labelsep=1pt]{$\scriptstyle .3$}
\endpspicture
\xe

The first display below uses the default {\sl nodesep\/}
parameters. The second shows the effects of different settings.
Modification of the {\sl nodesep\/} parameters is very useful in
various contexts.

\exh[xunit=2em,yunit=2ex]
\psset{xunit=2em,yunit=2ex}
\tac|
\jtree \! = :{A}@A1 :{B} :{C}@A2 {D}.
\nccurve[angleA=-180,angleB=-90]
   {A2}{A1}\endjtree
|endtac \hfil
\jtree \! = :{A}@A1 :{B} :{C}@A2 {D}.
\nccurve[angleA=-180,angleB=-90]{A2}{A1}\endjtree

\tac|
\jtree[nodesepA=1.5ex,nodesepB=.2ex]
\! = :{A}@A1 :{B} :{C}@A2 {D}.
\nccurve[angleA=-180,angleB=-90]
   {A2}{A1}\endjtree
|endtac \hfil
\jtree[nodesepA=1.5ex,nodesepB=.2ex]
\! = :{A}@A1 :{B} :{C}@A2 {D}.
\nccurve[angleA=-180,angleB=-90]{A2}{A1}\endjtree
\xe


\subsection \sl offset

|\psset{offsetA=�$x$�}| shifts the starting point of the
connection perpindicular to the curve a distance $x$.
If $x>0$, the shift is to the left looking in the direction of
the curve away from the starting point.  |offsetB| has the
corresponding effect with respect to the finishing point.

\medskip
|\nccurve[angleA=50,angleB=110,offsetA=�$x$�]{A1}{A2}|
\smallskip
for $x=0$, $x=-.3$, and $x=+.3\,$.

\exdisplay[style=how]
\qquad\pspicture(0,-.4)(4.6,2.4)%\psgrid
\rput[bl](0,0){\goop}
\pnode(4,0){A2}
\rput(A1){\psline[linewidth=.4pt,linecolor=gray](0,0)(2.4;50)
   \psline[linewidth=.4pt,linecolor=gray](0,0)(.6,0)
   \psarc[linewidth=.4pt,arrowsize=3pt]{->}(0,0){.4}{0}{50}
   \rput(.7;25){$\scriptstyle 50^\circ$}}
\rput(A2){\psline[linewidth=.4pt,linecolor=gray](0,0)(2.5;110)
   \psline[linewidth=.4pt,linecolor=gray](0,0)(.6,0)
   \psarc[linewidth=.4pt,arrowsize=3pt]{->}(0,0){.4}{0}{110}
   \rput(.7;45){$\scriptstyle 110^\circ$}}
\psdot(A1)
\psdot(A2)
\rput(.839,1.5){\psline[linewidth=.4pt,linecolor=gray](.5;-40)(.5;140)}
\nccurve[nodesep=0,angleA=50,angleB=110]{A1}{A2}
\nbput[npos=.06,labelsep=0]{$\scriptstyle 0$}
\nccurve[nodesep=0,angleA=50,angleB=110,offsetA=.3]{A1}{A2}
\nbput[npos=.06,labelsep=0]{$\scriptstyle +.3$}
\nccurve[nodesep=0,angleA=50,angleB=110,offsetA=-.3]{A1}{A2}
\nbput[npos=.06,labelsep=0]{$\scriptstyle -.3$}
\endpspicture
\xe

Here is a simple application of using the offset parameters that
might be appropriate for some special emphasis.\par\nobreak

\exh
\tac|
\jtree[xunit=3em,yunit=2em]
\! = :{A}@A {B} <vert>{C}@C .
\psset{angleA=-90,angleB=180}
\nccurve[offsetA=.5ex]{->}{A}{C}
\nccurve[offsetA=-.5ex]{A}{C}
\endjtree
|endtac \hfil
\jtree[xunit=3em,yunit=2em]
\! = :{A}@A {B} <vert>{C}@C .
\psset{angleA=-90,angleB=180}
\nccurve[offsetA=.5ex]{->}{A}{C}
\nccurve[offsetA=-.5ex]{A}{C}
\endjtree
\xe

This construction is used in example \gettag[Chung2]
in \exsec.  Example \gettag[ExBilink] in \exsec\ also uses the
offset parameters.


\subsection \sl ncurv

\medskip
|ncurv| affects the ``stiffness'' of the curve.

\medskip
|\nccurve[angleA=50,angleB=110,ncurv=�$x$�]{A1}{A2}|
\smallskip
for $x=.4$, $x=.67$, $x=1$, and $x=1.4\,$.

\exdisplay[style=how]
\qquad\pspicture(0,-.4)(4.6,2.8)%\psgrid
\rput[bl](0,0){\goop}
\pnode(4,0){A2}
\rput(A1){\psline[linewidth=.4pt,linecolor=gray](0,0)(2.4;50)
   \psline[linewidth=.4pt,linecolor=gray](0,0)(.6,0)
   \psarc[linewidth=.4pt,arrowsize=3pt]{->}(0,0){.4}{0}{50}
   \rput(.7;25){$\scriptstyle 50^\circ$}}
\rput(A2){\psline[linewidth=.4pt,linecolor=gray](0,0)(2.5;110)
   \psline[linewidth=.4pt,linecolor=gray](0,0)(.6,0)
   \psarc[linewidth=.4pt,arrowsize=3pt]{->}(0,0){.4}{0}{110}
   \rput(.7;45){$\scriptstyle 110^\circ$}}
\psdot(A1)
\psdot(A2)
%\rput(.839,1.5){\psline[linewidth=.4pt,linecolor=gray](.5;-40)(.5;140)}
\nccurve[nodesep=0,angleA=50,angleB=110,ncurv=.4]{A1}{A2}
\psset{labelsep=1pt}
\nbput{$\scriptstyle .4$}
\nccurve[nodesep=0,angleA=50,angleB=110,ncurv=.67]{A1}{A2}
\nbput{$\scriptstyle .67$}
\nccurve[nodesep=0,angleA=50,angleB=110,ncurv=1]{A1}{A2}
\nbput{$\scriptstyle 1$}
\nccurve[nodesep=0,angleA=50,angleB=110,ncurv=1.4]{A1}{A2}
\nbput{$\scriptstyle 1.4$}
%\nccurve[nodesep=0,angleA=50,angleB=110,offsetA=.3]{A1}{A2}
%\nccurve[nodesep=0,angleA=50,angleB=110,offsetA=-.3]{A1}{A2}
%\rput(.838,1.5){\rput(.3;-40)
%   {\rput[tr](0,-.5ex){$\scriptstyle
%-.3$}}}
%\rput(.838,1.5){\rput(-.3;-40)
%   {\rput[tr](0,-.5ex){$\scriptstyle
%+.3$}}}
%\psdot(.838,1.5)
\endpspicture
\xe

Here is a simple illustration of the use of |ncurv|. The
pointer below comes too close to ``B''.  Some adjustment is
needed.

\exh
\tac|
\jtree[unit=2em]
\! = :{A}@A :{B} {C}@C .
\nccurve[angleA=210,angleB=-80]{->}{C}{A}
\endjtree
|endtac \hfil
\jtree[unit=2em]
\! = :{A}@A :{B} {C}@C .
\nccurve[angleA=210,angleB=-80]{->}{C}{A}
\endjtree
\xe

One way to fix the problem is to increase the value of the
|ncurv| parameter.

\exh
\tac|
\jtree[unit=2em,nodesep=.6ex]
\! = :{A}@A :{B} {C}@C .
\nccurve[angleA=210,angleB=-80,
   ncurv=1.1]{->}{C}{A}
\endjtree
|endtac \hfil
\jtree[unit=2em,nodesep=.6ex]
\! = :{A}@A :{B} {C}@C .
\nccurve[angleA=210,angleB=-80,
   ncurv=1.1]{->}{C}{A}
\endjtree
\bigskip
\xe

\medskip
\subsection Specifying angles

Previous versions of jTree discussed parameters |dirA| and |dirB|
which could be used to set |angleA| and |angleB| indirectly, so
that they could be easily aligned with branches.  These
parameters should now be considered obsolete (although they will
still work as before).\footnote{$^1$}{The more robust handling of
``special coordinates'' in recent versions of PSTricks makes the
the {\sl dir\/} parameters unnecessary.}


\medskip
The challenge in the example below is to set |angleA| so that the
ncurve joins smoothly with the (short) left branch. One could
convert $\fam0 2\,em$ and $\fam0 3\,ex$ to the same units
(points, for example), and then use trigonometry to determine the
|angleA| setting.  This is at best tedious. Fortunately, in
contexts in which PSTricks expects an angle, it interprets
$(x,y)$ as the angle corresponding to the direction $(x,y)$.
$(-1,-1)$ is the direction of left branches in this example. Note
that in setting |angleA| in this way, braces are needed around
the direction so that the parameter parsing algorithm does not
get confused by the comma in the direction.
|\psset{angleA=(-1,-1)}| would try (unsuccessfully) to set
|angleA| to |(-1| and to find a parameter |-1)|.

\exh
\jtree[xunit=2em,yunit=3ex,bbadjust=left 1em]
\! = {A} <left>[scaleby=.4]@A1 ^<right> :{B} :{C}@A2 \etc.
\nccurve[angleA={(-1,-1)},angleB=190,nodesepA=0,
   ncurvA=1,ncurvB=1.2,arrows=->]{A1}{A2}
\endjtree
\hfill
\tac|
\jtree[xunit=2em,yunit=3ex,bbadjust=left 1em]
\! = {A} :[scaleby=.4]@A1 :{B} :{C}@A2 \etc.
\nccurve[arrows=->,angleA={(-1,-1)},angleB=190,
   nodesepA=0,ncurvA=1,ncurvB=1.2]{A1}{A2}
\endjtree
|endtac \hfil
\xe

\subsection Labeling connections between nodes

PSTricks has 3 commands for labeling connections between nodes.
|\ncput| puts the label directly on the connection, |\naput| puts
it above the connection, and |\nbput| puts it below the
connection.  There is a parameter |labelsep| which controls how
far above or below the connection the label is put; a parameter
|nrot| which can be used to rotate the label; and a parameter
|npos| which determines where on the connection the label is put.
from the starting label to the finishing label. See the PSTricks
User's Guide for all the details.
\medskip
For linguists, |\ncput| is probably the most useful of the
PSTricks connection labeling macros. Here is one example.

\exdisplay
\jtree[xunit=3em,yunit=1em,bbadjust=depth 4ex]
\! = :{who}@A1 :{did}@B1 :{Jack}
   :{\sl t}@B2 :{see} {\sl t}@A2 .
\psset{arrows=<-,angleA=-90,angleB=-150}
\nccurve{A1}{A2}
\ncput*[npos=.65]{\tenpoint\it wh-movement}
\nccurve{B1}{B2}
\endjtree
\xe

\codelines
|\jtree[xunit=3em,yunit=1em]
\! = :{who}@A1 :{did}@B1 :{Jack}
   :{\sl t}@B2 :{see} {\sl t}@A2 .
\psset{arrows=<-,angleA=-90,angleB=-150}
\nccurve{A1}{A2}
\ncput*[npos=.65]{\tenpoint\it wh-movement}
\nccurve{B1}{B2}
\endjtree
|endcodelines
\medskip

The asterisk following |\ncput| annuls the transparency of the
label.  It is surrounded by a framebox, which provides whitespace
around the label.  The size of the whitespace is controlled by
the parameter |framesep|.  The default value is used above.
\medskip

\subsection Labeling tree branches

Labeling branches in jTree trees is not entirely straightforward
because the branches (by default) are drawn using |\psline|,
which does not establish PSTricks nodes at the ends of the
branches.  Since there are no nodes, the labeling mechanism does
not apply.  It is, however, possible to change the PSTricks macro
that is used to draw branches by resetting |branch| to |\pcline|,
which draws the branch exactly as |\psline| does but also
establishes PSTricks nodes at the endpoints. (The parameter
|branch| is discussed in Section \gettag[CustomSec].) This is
illustrated below.

\exh
\tac|
\jtree[xunit=3em,yunit=3ex]
\! = <left>{A}
^<right>[branch=\pcline,nodesep=0]{B}
"{\ncput[nrot=:0]{$\vert\vert$}}.
\endjtree
|endtac \hfil
\jtree[xunit=3em,yunit=3ex]
\! = <left>{A}
^<right>[branch=\pcline,nodesep=0]{B}
"{\ncput[nrot=:0]{$\vert\vert$}}.
\endjtree
\xe
|"| is an escape from parsing the tree description, so that
|"{...}| simply executes the material inside the braces.  The |"|
escape is discussed in Section \gettag[EscapeSec].

\subsection Putting labels on circles and circling nodes in
trees

\ftag{\the\secno.\the\subsecno}[CircleSec]
We approach this by explicating the code for the display below.
\bigskip

\qquad
\jtree[unit=3em,yunit=3ex,bbadjust=depth 6ex]
\! = :{A} :{B}@C1 {C}@C2 .
\psLNode(C1)(C2){.5}{center}
\rput(center){%
   \pscircle{1.3}
   \rput[bl](1.3;40){$\leftarrow$ fronted}}
\endjtree
\bigskip

\CLnum
\jtree[unit=3em,yunit=3ex]
\! = :{A} :{B}@C1 {C}@C2 .
\psLNode(C1)(C2){.5}{center}
\rput(center){%
   \pscircle{1.3}
   \rput[bl](1.3;40){$\leftarrow$ fronted}}
\endjtree|endCLnum%|
\medskip
Line 2 draws the tree and establishes PSTricks nodes C1, and C2
(along with nodes C1:t, C1:b, C2:t, and C2:b).  Line 3
uses the PSTricks macro |\psLNode| to
interpolate a node labeled ``center'' between the nodes C1
and C2.  This will be the center of the circle which is drawn.
The circle is drawn by making a circle about the origin with
radius 1.3 (line 5) and pasting this in at the node labeled
``center'' (line 4).  The label is attached to the circle before
it is pasted in at ``center''.  The bottom left (|bl|) corner of
the label is put at |(1.3;40)|, which PStricks interprets as the
point at on the circle at angle $\rm 40^\circ$.

\medskip
Note that on line 1, the |unit| parameter (not the |xunit|
parameter) is set to $\fam0 3\,em$.  The reason is this.
|\pscircle{�$x$�}|, if $x$ is a number, draws a circle of radius
$x$ psunits. PSTricks keeps track of 3 units of measurement:
psxunits, psyunits, and psunits. {\it They can all be different.}
If the unit parameter is not set on line 1, the psunits are
simply inherited from the environment.  It is set first, because
setting the psunits to a value has the side effect of setting the
psxunits and psyunits to the same value.  Setting the psxunits or
psyunits leaves the psunits unchanged.  Finally, note that it is
relatively easy to estimate the radius of the circle in terms of
the horizontal dimension of the branches; about 1.2 or 1.3 times
as long.  That horizontal dimension is exactly 1 psunit.  It can
be fine-tuned by visual inspection.

\subsubsection Ellipses

\ftag{\the\secno.\the\subsecno.\the\subsubsecno}[EllipseSec]
There are two challenges in typesetting the display below.  One
is drawing and positioning the ellipse.  The other is positioning
``split head'' and drawing the pointer which ends on the
ellipse.

\exdisplay \qquad
\jtree[xunit=3em,yunit=2em,bbadjust=depth 6ex]
\! = :{Spec} :{A}@A1 :{B}@B1 {Comp}.
\psLNode(A1)(B1){.5}{AB1}
\rput{(1,-1)}(AB1){\jtenode*(.9,1){-40}{E1}}
\rput(E1){\rput[l](1,-1){\rnode[l]{E2}{split head}}}
\nccurve[angleA=-180,angleB=-60,nodesepB=0,arrows=->]
   {E2}{E1}
\endjtree
\xe

\CLnum
\jtree[xunit=3em,yunit=2em]
\! = :{Spec} :{A}@A1 :{B}@B1 {Comp}.
\psLNode(A1)(B1){.5}{AB1}
\rput{(1,-1)}(AB1){\jtenode*(.9,1){-40}{E1}}
\rput(E1){\rput[l](1,-1){\rnode[l]{E2}{split head}}}
\nccurve[angleA=-180,angleB=-60,nodesepB=0,arrows=->]
   {E2}{E1}
\endjtree|endCLnum %|

|\rput(BC){\psellipse(.9,1)}| draws an ellipse centered at the
node AB1 with $x$-radius .9 psxunits and $y$-radius 1 psyunit and
its $x$-axis in line with the $x$-axis of the page.  This needs
to be rotated clockwise to align with the right branches.
Putting the node named E1 on the ellipse is a challenge.  jTree
provides the macro |\jtenode|\index*{+jtenode} to solve the
problem.

\medskip
|\jtenode(�$x_1$�,�$y_1$�){�$\theta$�}{�name�}| puts a node with
the given name at the point illustrated below.

\exdisplay[unit=.5in]
\qquad
\pspicture(-2.1,-1.9)(2.1,1.7)
\rput(0,0){%
   \psset{nodesep=0}
   \psellipse(2,1.4)%
   \psline(0,-1.5)(0,1.5)%
   \psline(-2.1,0)(2.1,0)%
   \pcline[arrows=|*-|*](0,-1.7)(2,-1.7)%
   \ncput*{$x_1$}
   \pcline[arrows=|*-|*](2.3,0)(2.3,-1.7)%
   \ncput*{$y_1$}
   \jtenode(2,1.4){40}{EE}
   \psdot(EE)
   \psline(0,0)(2;{(EE)})
   \psarc{->}{1}{0}{40}
   \rput[l](1.1;15){$\theta$}
}
\endpspicture
\xe

If |\jtenode| is followed directly by an asterisk, the ellipse is
drawn, otherwise it is not.  This is used on line 6 above.
\medskip

Line 5 above is worth some attention because it illustrates the
recursive application of |\rput|.  One of the things that makes
PSTricks so useful is its ability to position objects relative to
nodes that PSTricks has already positioned.
|\rput(E1){\rput[l](1,-1){split head}}| puts ``split head'' in an
hbox and positions its left edge $(1,-1)$ from the node E1. Line
6 simply adds a node named E2 to the center of the left edge of
the hbox containing ``split head''.  E2 is used to draw the
pointer from ``split head'' to the ellipse.

\medskip
See Example \gettag[ExSprout] in Section \gettag[ExamplesSec] for a complex
example using the ideas of this section.




