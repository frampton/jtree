
\section Nodes and connections between them

\ftag{\the\secno}[nodesandhow]%
Nodes have a shape, a reference point, and a name. \pstnode\
allows the user to define box, elliptical, circular, and point
nodes.  In addition to a shape, nodes have a reference point.  It
is at the center of an elliptical, circular, or point node.  It
can be at the center of a box node, but there are other options
for box nodes: the ends and centers of the edges and the
baseline.  \pstnode\ has a number of commands for drawing curves
of various kinds between nodes.  Various features of how these
curves are rendered (linewidth, linestyle, arrows, etc.) can be
specified.  The most useful curve drawing command for annotating
tree structures is |\nccurve|.  The main point of this section is
to explain how |\nccurve| works.

\medskip
Suppose there is a box node and a point node as shown below.  The
dots are the reference points.
\bigskip

\def\templabel{\rput(0,1){\rput(0,2ex){A}}}
\def\goop{\psframebox[boxsep=false,framesep=0,linecolor=gray]
   {\vrule width0pt height1\psunit depth.5\psunit
      \hskip\psunit \pnode{QQ}\templabel\hskip\psunit}}

\psset{unit=1.2cm}
\hfil \pspicture(0,0)(9,2.8)%\psgrid
\pnode(8,0){B}
\pnode(8,0){Q}
\rput(8,0){\rput(0,02ex){B}}
\rput[B](1,1){\rnode[B]{A}{\goop}}
\SpecialCoor
\qdisk(Q){2pt}
\qdisk(QQ){2pt}
\endpspicture

\bigskip
The diagram below illustrates how
\medskip
\hfil|\nccurve[angleA=50,angleB=110]{A}{B}|
\medskip
is drawn.
\bigskip

\let\templabel=\relax

\hfil \pspicture(-.4,0)(8,3)%\psgrid
\pnode(8,0){B}
\pnode(8,0){Q}
\rput[B](1,1){\rnode[B]{A}{\goop}}
\SpecialCoor
\qdisk(Q){2pt}
\qdisk(QQ){2pt}
\psset{nodesep=0}
\nccurve[angleA=50,angleB=110]{A}{B}
\psline[style=dotted](Q)([angle=110,nodesep=2.8]Q)
\psline[style=dotted](QQ)([angle=50,nodesep=2.8]QQ)
\psline[style=dotted](QQ)([angle=0,nodesep=.6]QQ)
\psline[style=dotted](Q)([angle=0,nodesep=.6]Q)
\psarc{->}(Q){.5}{0}{110}
\psarc{->}(QQ){.5}{0}{50}
\rput[bl]([angle=25,nodesep=.6]QQ){\tenpoint$50^\circ$}
\rput[bl]([angle=55,nodesep=.6]Q){\tenpoint$110^\circ$}
\endpspicture
\bigskip




This simple picture can be modified by a number of parameters. In
addition to the usual parameters like |linewidth|, |linestyle|
and |arrows| which determine how a geometrical curve is rendered,
there are six parameters which modify the geometry of the curve
itself: |ncurvA|, |ncurvB|, |nodesepA|, |nodesepB|, |offsetA|,
and |offsetB|.  The parameters can be set individually or in
pairs.  |\psset{nodesep=�x\/�}| induces
|\psset{nodesepA=�x\/�,nodesepB=�x\/�}|.  |ncurv| and |offset|
work the same way. We will examine the parameters in turn.

\medskip |\psset{nodesepA=�x\/�}|
causes the dimensions of the node box to be adjusted by {\sl
x\/} before the curve is determined.

\def\Goop{\psframebox[boxsep=false,framesep=\dimpubb,linecolor=gray]
      {\goop}}
%\def\Goop{\psframebox[boxsep=false,framesep=\dimpubb,style=dotted]
%      {\goop}}
%\dimpuba=.5\psunit
\dimpubb=.2\psunit
\bigskip
\hfil \pspicture(-.2,0)(8,3)%\psgrid
\pnode(8,0){B}
\pnode(8,0){Q}
\rput[B](1,1){\rnode[B]{A}{\Goop}}
\nccurve[angleA=50,angleB=110,%nodesepB=\dimpuba,
   nodesepA=\dimpubb,nodesepB=0]{A}{B}
\SpecialCoor
\psline[style=dotted](Q)([angle=110,nodesep=3]Q)
\psline[style=dotted](QQ)([angle=50,nodesep=2.6]QQ)
\qdisk(Q){2pt}
\qdisk(QQ){2pt}
\endpspicture
\bigskip

The effect of |ncurv| is the most subtle.  One can imagine that
the curve is ``pulled'' in the directions of the arrows below.
You can think of |ncurv| as setting the strength of the pulls.
The default is $.67$.  The picture below shows the curves that
are drawn with various settings of |ncurv|.

\bigskip
\hfil \pspicture(-.2,0)(8,4.1)%\psgrid
\pnode(8,0){B}
\pnode(8,0){Q}
\rput[B](1,1){\rnode[B]{A}{\Goop}}
\psset{angleA=50,angleB=110,nodesepA=\dimpubb,
   nodesepB=0,labelsep=3pt}
\nccurve[ncurv=.4]{A}{B}
\bput(.35){\tenrm .4}
\nccurve[ncurv=.67]{A}{B}
\bput(.35){\tenrm .67}
\nccurve[ncurv=1]{A}{B}
\bput(.35){\tenrm 1}
\nccurve[ncurv=1.4]{A}{B}
\bput(.4){\tenrm 1.4}
\SpecialCoor
\psline[style=dotted]{->}(Q)([angle=110,nodesep=4]Q)
\psline[style=dotted]{->}(QQ)([angle=50,nodesep=4]QQ)
\qdisk(Q){2pt}
\qdisk(QQ){2pt}
\endpspicture
\bigskip

Here is a simple illustration of the use of |ncurv|. There is a
typesetting problem with the pointer below which must be fixed.\par\nobreak

\exh
\tac|
\jtree[unit=2em]
\! = :{A}@A :{B} {C}@C .
\nccurve[angleA=210,angleB=-80]{->}{C}{A}
\endjtree
|endtac \hfil
\jtree[unit=2em]
\! = :{A}@A :{B} {C}@C .
\nccurve[angleA=210,angleB=-80]{->}{C}{A}
\endjtree
\xe

One way to fix the problem is to increase the value of the
|ncurv| parameter.

\exh
\tac|
\jtree[unit=2em,nodesep=.6ex]
\! = :{A}@A :{B} {C}@C .
\nccurve[angleA=210,angleB=-80,
   ncurv=1.1]{->}{C}{A}
\endjtree
|endtac \hfil
\jtree[unit=2em,nodesep=.6ex]
\! = :{A}@A :{B} {C}@C .
\nccurve[angleA=210,angleB=-80,
   ncurv=1.1]{->}{C}{A}
\endjtree
\bigskip
\xe

|\psset{ncurv=�x\/�}| has the effect
|\psset{ncurvA=�x\/�,ncurvB=�x\/�}|.
Sometimes it is advantageous to set the parameters to different
values.  Experimentation will quickly give you a feel for how
this works.

\medskip
Finally, we come to the two |offset| parameters.
|\psset{offsetA=�x\/�}| causes the starting point to be offset a
distance {\sl x\/} perpindicular to the starting direction. The
direction is to the left (looking in the starting direction) if
{\sl x\/} is positive.

%\dimpuba=.5\psunit
%\dimpubb=.2\psunit
\bigskip
\hfil \pspicture(-.2,0)(8,3.4)%\psgrid
\pnode(8,0){B}
\pnode(8,0){Q}
\rput[B](1,1){\rnode[B]{A}{\Goop}}
\SpecialCoor
\psline[style=dotted](Q)([angle=110,nodesep=3]Q)
\pnode([angle=50,nodesep=3]QQ){L1}
\psline[style=dotted](QQ)(L1)
\qdisk(Q){2pt}
\qdisk(QQ){2pt}
\pnode(2.007,2.2){Z}
\rput([angle=140,nodesep=.3]Z){\pnode{Z2}}
\nccurve[angleA=50,angleB=110,nodesep=0]{Z2}{B}
\psline[style=dotted](Z)([angle=140,nodesep=1]Z)
\psline[style=dotted]([angle=140,nodesep=.3]L1)%
   ([angle=140,nodesep=.3]QQ)
\endpspicture
\bigskip

|\psset{offsetB=�x\/�}| causes the finishing point to be
displaced a distance {\sl x\/} perpindicular to the finishing
direction, to the left.  Remember that the finishing direction
points toward the node, not back along the curve.

\medskip
Here is a simple application that might be appropriate for
some special emphasis.\par\nobreak

\exh
\tac|
\jtree[xunit=3em,yunit=2em]
\! = :{A}@A {B} <vert>{C}@C .
\psset{angleA=-90,angleB=180}
\nccurve[offsetA=.5ex]{->}{A}{C}
\nccurve[offsetA=-.5ex]{A}{C}
\endjtree
|endtac \hfil
\jtree[xunit=3em,yunit=2em]
\! = :{A}@A {B} <vert>{C}@C .
\psset{angleA=-90,angleB=180}
\nccurve[offsetA=.5ex]{->}{A}{C}
\nccurve[offsetA=-.5ex]{A}{C}
\endjtree
\xe

This construction is used several times in example \gettag[Chung2]
in \exsec.

\subsection Labeling connections between nodes

PSTricks has 3 commands for labeling connections between nodes.
|\ncput| puts the label directly on the connection, |\naput| puts
it above the connection, and |\nbput| puts it below the
connection.  There is a parameter |labelsep| which controls how
far above or below the connection the label is put; a parameter
|nrot| which can be used to rotate the label; and a parameter
|npos| which determines where on the connection the label is put.
from the starting label to the finishing label. See the PSTricks
User's Guide for all the details.
\medskip
Here are some examples of the use of |\ncput|, the most useful of
the connection labeling macros.

\jtree[xunit=3em,yunit=1em,bbadjust=depth 4ex]
\! = :{who}@A1 :{did}@B1 :{Jack}
   :{\sl t}@B2 :{see} {\sl t}@A2 .
\psset{arrows=<-,angleA=-90,angleB=-150}
\nccurve{A1}{A2}
\ncput*[npos=.65]{\tenpoint\it wh-movement}
\nccurve{B1}{B2}
\endjtree
\bigskip

\codelines
|\jtree[xunit=3em,yunit=1em]
\! = :{who}@A1 :{did}@B1 :{Jack}
   :{\sl t}@B2 :{see} {\sl t}@A2 .
\psset{arrows=<-,angleA=-90,angleB=-150}
\nccurve{A1}{A2}
\ncput*[npos=.65]{\tenpoint\it wh-movement}
\nccurve{B1}{B2}
\endjtree
|endcodelines

\jtree[xunit=3em,yunit=1em,bbadjust=depth 4ex]
\! = :{who}@A1 :{did}@B1 :{Jack}
   :{\sl t}@B2 :{see} {\sl t}@A2 .
\psset{arrows=<-,angleA=-90,angleB=-150}
\nccurve{A1}{A2}
\ncput*[nrot=:0,npos=.35]{\tenpoint\it wh-movement}
\nccurve{B1}{B2}
\endjtree
\bigskip

The code is the same as in the previous example except for the
parameters of |\ncput|.  Here, they are |[nrot=:0,npos=.35]|.
|npos|, which must be set to a number between 0 and 1,
picks out a position along the connection; the default is .5.
The parameter |nrot| gives the rotation of the label; the default
is 0.  If the angle (measure in degrees) is preceeded by a colon,
as in this example, the angle is measure with respect to the
direction of the connection at the point at which the label is
positioned.  The asterisk following |\ncput| annuls the
transparency of the label.  It is surrounded by a framebox, which
provides an extra whitespace around the label.  The size of the
whitespace is controlled by the parameter |labelsep|.

jTree constructs trees by using node connections.  Sometimes it
is desirable to put a label on a tree branch.  This poses a
problem, because the tree branch connections are generally drawn
using |branch=\psline|, which not extablish nodes so that the
PSTricks connection labeling mechanism does not apply.  But the
branch parameter can be temporarily changed, if needed.  This is
illustrated below.

\exh
\tac|
\jtree[xunit=3em,yunit=3ex]
\! = <left>{A}
^<right>[branch=\pcline,nodesep=0]{B}
"{\ncput[nrot=:0]{$\vert\vert$}}.
\endjtree
|endtac \hfil
\jtree[xunit=3em,yunit=3ex]
\! = <left>{A}
^<right>[branch=\pcline,nodesep=0]{B}
"{\ncput[nrot=:0]{$\vert\vert$}}.
\endjtree
\xe
|"| is an escape from parsing the tree description, so that
|"{...}| simply executes the material inside the braces.

\subsection Putting labels on circles and circling nodes in
trees

We approach this by explicating the code for the display below.
\bigskip

\jtree[xunit=3em,yunit=3ex,bbadjust=depth 6ex]
\! = :{A} :{B}@C1 {C}@C2 .
\psLNode(C1)(C2){.5}{center}
\rput(center){%
   \pscircle{1.3}
   \rput[bl](1.3;40){$\leftarrow$ fronted}
}
\endjtree
\bigskip

\CLnum
\jtree[xunit=3em,yunit=3ex]
\! = :{A} :{B}@C1 {C}@C2 .
\psLNode(C1)(C2){.5}{center}
\rput(center){%
   \pscircle{1.3}
   \rput[bl](1.3;40){$\leftarrow$ fronted}
}
\endjtree|endCLnum%|
\medskip
Line 2 draws the tree and establishes ps nodes C1, and C2 (along
with nodes C1:t, C1:b, C2:t, and C2:b).  Line 3 interpolates the
a node labeled ``center'' between the nodes C1 and C2.  This will
be the center of the circle which is drawn.  The circle is drawn
by making a circle about the origin with radius 1.3 (line 5) and
pasting this in at the node labeled ``center'' (line 4).  The
label is attached to the circle before it is paster in at
``center''.  The bottom left (|bl|) corner of the label is put at
|(1.3;40)|, which PStricks recognizes as the point at
distance 1.3 which makes an angle $\rm 40\,degrees$ with the
positive horizontal direction.



