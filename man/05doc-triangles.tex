
\section Triangles and vartriangles

Triangles can be defined. The definition syntax is:

\medskip\index*{+deftriangle}
\qquad|\deftriangle<�name\/�>(�height\/�)(�%
   \sl slopeA\/�)(�slopeB\/�)|

\medskip
One triangle comes predefined.

\cframe
|\deftriangle<tri>(1)(1)(-1)
|endcframe

|<tri>| can be used like any other branch.

\exh \tac|
\jtree
\! = :{a} <tri>{b} <vert>{c}.
\endjtree
|endtac \hfil
\jtree
\! = :{a} <tri>{b} <vert>{c}.
\endjtree
\xe

\pstjtree\/ defines the parameter |triratio|\index*{triratio (for
triangles)},
which can be used to adjust where the new current point is
positioned along the bottom edge of the triangle. If $\sl width$
is the width of the triangle, the new current point is at
distance $\sl x=triratio\times width$ to the right of the left
corner of the triangle. The display above is set with |triratio=.5|,
the default, which puts the current point in the center of the
bottom edge.

\exh \tac|
\jtree
\! = :{a} <tri>[triratio=.8]{b} <vert>{c}.
\endjtree
|endtac \hfil
\jtree
\! = :{a} <tri>[triratio=.8]{b} <vert>{c}.
\endjtree
\xe

When a triangle is evaluated, its width is computed and a macro
|\triwd| \index*{+triwd}is defined which evaluates to this width.  {\sl
pst-jtree\/} contains:\par\nobreak

\medskip
\index*{+triline}%
\cframe|\def\triline#1{\hbox to\triwd{#1}}|endcframe

A construction like the following is sometimes
useful:\par\nobreak
\exh \tac|
\jtree
\! = :{a} <tri>{\triline{e\hfil f}}.
\endjtree
|endtac \hfil
\jtree
\! = :{a} <tri>{\triline{e\hfil f}}.
\endjtree
\xe

Remember as well that branches can be overwritten.
Something along the lines of the following is sometimes
useful.\par\nobreak

\exh \tac|
\jtree
\! = :{a} <tri> ^:{e} {f} <vert>{c}.
\endjtree
|endtac \hfil
\jtree
\! = :{a} <tri> ^:{e} {f} <vert>{c}.
\endjtree
\xe

\subsection Vartriangles

Vartriangles adjust their width to fit the
label that they point to. A vartriangle is specified by giving a
height. The definition syntax is:

\medskip
\qquad|\defvartriangle<�name\/�>(�height\/�)|
\medskip\index*{+defvartriangle}

\jTree\ predefines
one vartriangle.

\medskip
\quad |\defvartriangle<vartri>(1)|

\exh \tac|
\jtree
\! = :{a} <vartri>{whatever width}.
\endjtree
|endtac \hfil
\jtree
\! = :{a} <vartri>{whatever width}.
\endjtree
\xe

The parameter |triratio|\index*{triratio (for vartriangles)} has a
different meaning for vartriangles than it has for ordinary
triangles.  It determines where the center of the bottom edge of
the vartriangle is with respect to the top vertex. The center of
the bottom edge is a distance $\sl triratio\times width$ to the
right of its left edge. It is easier to illustrate (done below)
than to give the formula.

\exh
\tac|
\jtree
\! = :{a} <vartri>
   [triratio=.3]{whatever width}.
\endjtree
|endtac \hfill
\jtree
\! = :{a} <vartri>
   [triratio=.3]{whatever width}.
\endjtree
\kern1ex

\bigskip

\tac|
\jtree
\! = :{a}
   <vartri>[triratio=0]{whatever width}.
\endjtree
|endtac \hfill
\jtree
\! = :{a}
   <vartri>[triratio=0]{whatever width}.
\endjtree
\kern1ex

\bigskip

\tac|
\jtree
\! = :{a}
   <vartri>[triratio=-.1,scaleby=2]
      {whatever width}.
\endjtree
|endtac \hfill
\jtree
\! = :{a}
   <vartri>[triratio=-.1,scaleby=2]
      {whatever width}.
\endjtree
\kern1ex
\xe

Generally, there is no branching from the label following a
vartriangle, so no provision is made for adjusting the
termination point of the label that follows vartriangles.  It is
at the center of the base.\par\nobreak

\exh \tac|
\jtree
\! = :{a} <vartri>
      [triratio=.1]{whatever width}
   <vert>{b}.
\endjtree
|endtac \quad \hfil
\jtree
\! = :{a} <vartri>
      [triratio=.1]{whatever width}
   <vert>{b}.
\endjtree
\xe

Another branch can follow a branch, but since the width of a
vartriangle depends on the width of the label which follows it, a
vartriangle (with optional parameters) must be followed
immediately by a label.  An error results if it is not.



