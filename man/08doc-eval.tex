
\section Expansion and evaluation of control sequences in tree parsing


\subsection The {\tt "} escape from tree parsing

\ftag{\the\secno.\the\subsecno}[EscapeSec]
If the parser encounters |"|, it evaluates the next token or
group and continues the parse.  Evaluation should contribute no
material, since it is not parsed.  But evaluation can change
parameter settings, which affects how the remaining material is
typeset.  For example:

\exh \tac|
\jtree
\! = {a} :{b}!! {c}
   "{\psset{scaleby=.5 1}} :{f} {g}.
\!! = :{d} {e}.
\endjtree
|endtac \hfil
\jtree
\! = {a} :{b}!! {c}
   "{\psset{scaleby=.5 1}} :{f} {g}.
\!! = :{d} {e}.
\endjtree
\xe

|\!� \dots\ �.| does not establish an implicit group, so the
change in scale persists to the subsequent subtree.

\medskip
The same effect can be achieved without using |"| if rescaling is
done outside tree parsing.\par\nobreak

\exh \tac|
\jtree
\! = {a} :{b}!a {c}!b .
\psset{scaleby=.5 1}
\!a = :{d} {e}.
\!b = :{f} {g}.
\endjtree
|endtac \hfil
\jtree
\! = {a} :{b}!a {c}!b .
\psset{scaleby=.5 1}
\!a = :{d} {e}.
\!b = :{f} {g}.
\endjtree
\xe

\subsection Control sequence expansion

If the parser encounters a control sequence or active character
in a tree description, it is replaced by its expansion before the
parse continues.  The expansion is parsed without further
expansion of the initial token, which prevents an infinite loop
if the expansion yields an unexpandable control sequence.

\exh
\tac|
\jtree
\def\Colon{:[scaleby=2.3]}%
\! = \Colon !a :{c} :{d} {e}.
\!a = :{a} {b}.
\endjtree
|endtac \hfil
\jtree[treevshift=-1ex]
\def\Colon{:[scaleby=2.2]}%
\! = \Colon !a :{c} :{d} {e}.
\!a = :{a} {b}.
\endjtree
\xe

\pstjtree\/ contains the definition
|\def\jtlong{[scaleby=2.3]}|, so we could also write:\par\nobreak

\exh \tac|
\jtree
\! = :\jtlong !a :{c} :{d} {e}.
\!a = :{a} {b}.
\endjtree
|endtac \hfil
\jtree[treevshift=-1ex]
\! = :\jtlong !a :{c} :{d} {e}.
\!a = :{a} {b}.
\endjtree
\xe

This technique makes some code much more transparent.
(\gettag[cheese]) on page \gettag[cheesepage], for example,
can be written:

\bigskip
\vtop{\leftskip=1.5em \mixedenv
|\jtree[xunit=2.2em,yunit=1em]
\! =  :\jtlong !a :{is} {rotten}.
\!a = :{the} :{cheese} :{that} :\jtlong !b :{ate} {\it t}.
\!b = :{the} :{rat} :{that} :\jtlong !c :{killed} {\it t}.
\!c = :{the} :{cat} :{that} :{John} :{owned} {\it t}.
\endjtree
|\par}

\bigskip
The complete list of parameter changes that are encoded in
macros in \pstjtree\/ is:\par\nobreak

\cframe
|\def\jtlong{[scaleby=2.3]}
\def\jtshort{[scaleby=.5]}
\def\jtwide{[scaleby=2 1]}
\def\jtbig{[scaleby=2]}
\def\jtjot{[scaleby=1.3]}
|endcframe
\index*{+jtlong}\index*{+jtwide}\index*{+jtbig}\index*{+jtjot}%
\index*{+jtshort}

\vskip.7em
Of course, users should define whatever similar macros are useful
in their own work.



