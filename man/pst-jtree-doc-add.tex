%\hsize=5.5in
%\hoffset=.5in
%\input pstricks
%\input pst-node
%\input pst-xkey
%\input pst-jtree
%\input ../@lingXK
%\input ../@lingdoc
%\input ../@timesmin
%\input ../@code
%\input ../@sections
%\input ../@jtree-doc
%\gatherftags
%\gatherindex
%\gathercontents

\def\showbb{\hbox\bgroup\vrule\vtop\bgroup\vbox\bgroup\hrule\hbox\bgroup}
\def\endshowbb{\egroup\egroup\hrule\egroup\vrule\egroup}


%\centerline{\sectitlefont Pst-jTree, Version 2.4:\enspace
%   Extensions and one bug fix}
%\medskip
%\centerline{\subtitlefont (September 2008)}
%\bigskip

\noindent 1. |\brokenbranch| and |\etcbranch| now work as
advertised.
\medskip
\noindent 2. Previous versions of jTree assumed that the
characters |^|, |>|, |<|, and |"|, which all have special meaning
to the jTree parser, had their normal category codes when the
macro |\jtree| was invoked.  This made jTree incompatible with
Babel, gb4e, and any other macro packages which alter the
category code of these characters.  This incompatibility has now
been removed.  Invoking |\jtree| changes the category codes of
these characters to the normal values.  The change does not apply
within labels, so the Babel shortcuts can be used in labels.
\medskip
\noindent 3. There is a new macro |\elc| (empty label comment)
which can be used as follows.

\endinput
\excentered
\CLboxed
\jtree
\! = {\elc{TP}}
:{Bill}
:{T} {\elc{VP}}
:{see} {Mary}.
\endjtree
|endCLboxed \hfil
\jtree
\! = {\elc{TP}}
:{Bill}
:{T} {\elc{VP}}
:{see} {Mary}.
\endjtree
\xe
%
The term ``comment'' is used rather than ``label'' because the
text is not treated as label by jTree.  Like |{\pnode{|\dots|}}|,
|{\elc{|\dots|}}| is considered to have an empty label, but there
is a side effect in typesetting it.  The argument of |\elc| is
typeset in a certain position relative to the empty label. The
position is determined by the parameters |elcxoffset|,
|elcyoffset|, and |elcref|.  The default settings are established
by
\medskip
\hfil|\psset{elcxoffset=.8ex,elcyoffset=0,elcref=bl}|
\medskip
\noindent With these settings, |\elc{TP}| above (for
example) is typeset by putting |TP| in an hbox and positioning
its bottom left (|bl|) corner with respect to the empty node
using the hoffset and voffset settings.
\medskip
|\elc| takes parameters, as illustrated below.  The default
elc parameters are assumed to be in force.
|elcxoffset| and |elcyoffset| are incremental
parameters.

\excentered
\CLboxed
\jtree
\! =
{\elc[elcref=b,
   elcyoffset=.5ex,
   elcxoffset=0]{CP}}
:{C} {\elc[elcxoffset=!.5ex]{TP}}
:{Bill}
:{T} {\elc{VP}}
:{see} {Mary}.
\endjtree
|endCLboxed \kern-1em\hfil
\jtree
\! =
{\elc[elcref=b,elcyoffset=.5ex,
   elcxoffset=0]{CP}}
:{C} {\elc[elcxoffset=!.5ex]{TP}}
:{Bill}
:{T} {\elc{VP}}
:{see} {Mary}.
\endjtree
\xe
|\rput| is used to typeset the node comments, so they do not
contribute to size of the bounding box which jTree creates.
\bigskip

4. jTree adjusts the size of the hbox which
|\jtree|\dots|\endjtree| creates so that it encloses all the
labels.  But paths drawn by PSTricks and text which is positioned
by PSTricks (path annotations, for example) can fall outside this
bounding box.  For example,
\bigskip
\jtree[nodesep=.5ex]
\! = {A}@A1
:{A}@A3
:{B}@A2 {C}.
\nccurve[angleA=-30,ncurvA=9,
   angleB=90,ncurvB=4,arrows=->]{A2}{A1}
\nccurve[angleA=210,
   angleB=180,ncurv=3,arrows=->]{A2}{A3}
\endjtree
\bigskip
is produced by the code below.
\bigskip
\CLboxed
For example,
\bigskip
\jtree[nodesep=.5ex]
\! = {A}@A1
:{A}@A3
:{B}@A2 {C}.
\nccurve[angleA=-30,ncurvA=9,
   angleB=90,ncurvB=4,arrows=->]{A2}{A1}
\nccurve[angleA=210,
   angleB=180,ncurv=3,arrows=->]{A2}{A3}
\endjtree
\bigskip
is produced by the code below.
|endCLboxed
\bigskip


\noindent It is usually sufficient to add suitable kerns to
get such trees properly positioned.  Many users, however, are not
familiar with the Tex's kerning system.  So jTree has been
extended by providing a parameter |bbadjust| which can be used to
adjust the size of the bounding box in tree constructions.

\vfil\break
Contrast the tree spacing above with
\bigskip
\jtree[nodesep=.5ex,bbadjust=height 4ex depth 3ex left 2em]
\! = {A}@A1
:{A}@A3
:{B}@A2 {C}.
\nccurve[angleA=-30,ncurvA=9,
   angleB=90,ncurvB=4,arrows=->]{A2}{A1}
\nccurve[angleA=210,
   angleB=180,ncurv=3,arrows=->]{A2}{A3}
\endjtree
\bigskip
which is produced by the code below.
\bigskip
\CLboxed
Contrast the tree spacing above with
\bigskip
\jtree[nodesep=.5ex,bbadjust=height 4ex depth 3ex left 2em]
\! = {A}@A1
:{A}@A3
:{B}@A2 {C}.
\nccurve[angleA=-30,ncurvA=9,
   angleB=90,ncurvB=4,arrows=->]{A2}{A1}
\nccurve[angleA=210,
   angleB=180,ncurv=3,arrows=->]{A2}{A3}
\endjtree
\bigskip
which is produced by the code below.
|endCLboxed
\bigskip
The right edge of the bounding box above has not been adjusted
even though the path projects out of the right side, because this
is irrelevant here.  But it can be adjusted in the obvious way if
desired.  Side by side trees might need such adjustment, for
example.
\medskip
The default settings are established in {\it pst-jtree\/} by
\medskip
\hfil |\psset{bbadjust=height 0pt depth 0pt left 0pt right 0pt}|
\medskip


Dimensions furnished to |bbadjust| must be Tex dimensions, not
Postcript dimensions, and they must be absolute, not incremental.


\bye
\get Cover
\get TypesetContents
\get 00-intro
\get 01-TDL
\get 02-parameters
\get 03-labels
\get 04-branches
\get 05-triangles
\get 06-nodes
\get 07-colon
\get 08-eval
\get 09-bells
\get 10-incremental
\get 11-boundingbox
\get 12-nodesandhow
\get 13-examples
\get 14-issues
\get 15-appendix
\input TypesetIndex
\bye

