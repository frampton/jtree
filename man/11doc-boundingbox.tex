
\section The bounding box

\ftag{\the\secno}[BBSec]

|\jtree�\thinspace\dots�\endjtree| yields an hbox.  It has
dimensions and a baseline.  There are jTree parameters which can be
used to make adjustments to the dimensions and baseline if
needed.

\subsection {\sl baretopadjust}

The baseline of the box created by |\jtree...\endjtree| is
normally the baseline of the root label.  If, however, the root
label is empty, that puts the baseline too low, at least for my
taste.  Contrast the trees below:
\bigskip
\qquad a.\quad
\jtree \! = {A} :{B} {C}.\endjtree
\hfil b.\quad
\jtree[baretopadjust=0] \! = :{B} {C}.\endjtree
\bigskip
\jTree\  takes corrective action by raising the tree diagram if
the root label is empty by the amount specified by the parameter
\index*{baretopadjust}|baretopadjust|.  The default setting is
$\fam0 1.4\,ex$.  With the default
setting:
\bigskip
\qquad a.\quad
\jtree \! = {A} :{B} {C}.\endjtree
\hfil b.\quad \jtree[baretopadjust=1.4ex] \! = :{B} {C}.\endjtree

\subsection {\sl treevshift}

\jTree\ also provides the parameter
\index*{treevshift}|treevshift|, which is set to 0 by default. It
is independent of |baretopadjust| and can be used to move the
tree diagram up and down, if desired.

\exh \leavevmode\raise1.2em\hbox{\tac|
$\to$\qquad
\jtree[treevshift=1.2em]
\! = {A} :{B} {C}.
\endjtree
|endtac}\hfil
$\to$\qquad
\jtree[treevshift=1.2em]
\! = {A} :{B} {C}.\endjtree
\xe

\subsection {\sl bbadjust\/} and {\sl bbshow}

PSTricks creates dimensionless graphics, but \jTree\ goes to a
lot of trouble to figure out the sizes of the trees that it
generates and to put them in appropriately sized boxes.  For
example (with the box dimensions indicated by the dashed box):

\exh \tac|
\jtree
\! = {X} :{a} :{a} :{a}
   {\multiline
      this and\cr
      that\endmultiline}.
\endjtree
|endtac \hfil
\psframebox[framesep=0,linestyle=dashed]{\jtree
\! = {X} :{a} :{a} :{a}
   {\multiline
      this and\cr
      that\endmultiline}.
\endjtree}
\xe

The sizing is not perfect.  \jTree\ is not clever enough to
recognize the white space due to |labelgapt|, |labelgapb|,
|labelstrutt|, and |labelstrutb|.  But it is not bad.

\medskip
If PSTricks is used to draw node connections, they often extend outside the
\jTree\ bounding box.

\exh \tac|
\jtree
\! = {X}@A1
   <right>
   :{a}
   :{a}
      {\multiline
         this and\cr
         that\endmultiline}@A2 .
\nccurve[angleA=210,angleB=200,
   ncurv=2,nodesepA=0]{->}{A1:b}{A2}
\endjtree
|endtac \hfil
\psframebox[framesep=0,linestyle=dashed]{\jtree
\! = {X}@A1
   <right>
   :{a}
   :{a}
      {\multiline
         this and\cr
         that\endmultiline}@A2 .
\nccurve[angleA=210,angleB=200,
   ncurv=2,nodesepA=0]{->}{A1:b}{A2}
\endjtree}
\xe

There are two ways to fix this.  The first is elementary, simply
inserting kerning as appropriate.

\exh \tac|
\hbox{\kern2.4em\jtree
\! = {X}@A1
   <right>
   :{a}
   :{a}
      {\multiline
         this and\cr
         that\endmultiline}@A2 .
\nccurve[angleA=210,angleB=200,
   ncurv=2,nodesepA=0]{->}{A1:b}{A2}
\endjtree}
|endtac \hfil
\psframebox[framesep=0,linestyle=dashed]{\kern2.4em\jtree
\! = {X}@A1
   <right>
   :{a}
   :{a}
      {\multiline
         this and\cr
         that\endmultiline}@A2 .
\nccurve[angleA=210,angleB=200,
   ncurv=2,nodesepA=0]{->}{A1:b}{A2}
\endjtree}
\xe

jTree also provides the two parameters which can be used to
fine-tune the extent of the bounding box, if such adjustment is
needed in some application.  First, the boolean parameter
|showbb|\index*{showbb} can be set to |true| to show the bounding
box. It has the XKV default value |true|, so |\psset{showbb}|
with no value is sufficient to set it to |true|.  It is always
set to |false| when |\endjtree| is executed, so the scope of a
|true| setting does not extend past the first |\endjtree|
encountered. The parameter |bbadjust|\index*{bbadjust} can be
used to modify the size of the bounding box from the size which
|\jtree| naturally computes.  Setting |bbadjust=height 1ex depth
2ex left 1em right .5em|, for example, alters the size of the box
as expected.  The specifications of the height, depth, left
(side), right (side) adjustments are optional, provided that
there is at least one specified.  |showbb| and |bbajust| are
illustrated below.

\exh
\tac|
\jtree[showbb]
\! = :{A}@A1 ({[Q]}[labelgap=-.5ex])
   :{B} {C}@C1 .
\nccurve[angleA=200,angleB=-140,
   ncurv=1.4,arrows=->]{A1}{C1}
\endjtree
|endtac \hfil
\jtree[showbb]
\! = :{A}@A1 ({[Q]}[labelgap=-.5ex]) :{B} {C}@C1 .
\nccurve[angleA=200,angleB=-140,ncurv=1.4]{->}{A1}{C1}
\endjtree
\bigskip
\tac|
\jtree[showbb,
   bbadjust=left 1.2em depth 2.3ex]
\! = :{A}@A1 ({[Q]}[labelgap=-.5ex])
   :{B} {C}@C1 .
\nccurve[angleA=200,angleB=-140,
   ncurv=1.4,arrows=->]{A1}{C1}
\endjtree
|endtac \hfil
\jtree[showbb,
   bbadjust=left 1.2em depth 2.3ex]
\! = :{A}@A1 ({[Q]}[labelgap=-.5ex])
   :{B} {C}@C1 .
\nccurve[angleA=200,angleB=-140,
   ncurv=1.4,arrows=->]{A1}{C1}
\endjtree
\xe

In addition to the boolean parameter {\sl showbb}, there is a
command macro |\showbb|\index*{+showbb} which turns on display of
the bounding box.  The macro is more convenient than the
parameter in testing and adjustment because it is more easily
commented out.


\subsection {\sl everytree}

Suppose you would like to see the outlines of all the bounding
boxes of your jTree trees in some document. You could go to to
each one and set the |showbb| parameter, but that would be
tedious.  jTree provides a parameter which solves the problem.
\index*{everytree}If |\psset{everytree=\showbb}| is executed,
then all bounding boxes in the scope of the setting will be
outlined. |\psset{everytree=�x\/�}| puts {\sl x\/} into a token
list (|\jteverytree|) whose tokens are inserted at the beginning
of every |\jtree�\thinspace\dots�\endjtree| construction. It is
inserted before the local |\jtree| parameter settings take
effect, so that it will be overridden by local parameter
settings.

