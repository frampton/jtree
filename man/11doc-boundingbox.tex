
\section The bounding box

PSTricks creates dimensionless graphics, but \jTree\ goes to a
lot of trouble to figure out the sizes of the trees that it
generates and to put them in appropriately sized boxes.  For
example:

\exh \tac|
\psframebox[framesep=0]{\jtree
\! = {X} :{a} :{a} :{a}
   {\multiline
      this and\cr
      that\endmultiline}.
\endjtree}
|endtac \hfil
\psframebox[framesep=0]{\jtree
\! = {X} :{a} :{a} :{a}
   {\multiline
      this and\cr
      that\endmultiline}.
\endjtree}
\xe

The sizing is not perfect.  \jTree\ is not clever enough to
recognize the white space due to |labelgapt|, |labelgapb|,
|labelstrutt|, and |labelstrutb|.  But it is not bad.

\medskip
If PSTricks is used to draw arrows, they often extend outside the
\jTree\ bounding box.

\exh \tac|
\psframebox[framesep=0]{\jtree
\! = {X}@A1
   <right>
   :{a}
   :{a}
      {\multiline
         this and\cr
         that\endmultiline}@A2 .
\nccurve[angleA=210,angleB=200,
   ncurv=2,nodesepA=0]{->}{A1:b}{A2}
\endjtree}
|endtac \hfil
\psframebox[framesep=0]{\jtree
\! = {X}@A1
   <right>
   :{a}
   :{a}
      {\multiline
         this and\cr
         that\endmultiline}@A2 .
\nccurve[angleA=210,angleB=200,
   ncurv=2,nodesepA=0]{->}{A1:b}{A2}
\endjtree}
\xe

This has to be fixed by hand by inserting appropriate kerning.

\exh \tac|
\psframebox[framesep=0]{\kern2.4em
\jtree
\! = {X}@A1
   <right>
   :{a}
   :{a}
      {\multiline
         this and\cr
         that\endmultiline}@A2 .
\nccurve[angleA=210,angleB=200,
   ncurv=2,nodesepA=0]{->}{A1:b}{A2}
\endjtree}
|endtac \hfil
\psframebox[framesep=0]{\kern2.4em\jtree
\! = {X}@A1
   <right>
   :{a}
   :{a}
      {\multiline
         this and\cr
         that\endmultiline}@A2 .
\nccurve[angleA=210,angleB=200,
   ncurv=2,nodesepA=0]{->}{A1:b}{A2}
\endjtree}
\xe





