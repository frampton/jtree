
\section The bounding box

\ftag{\the\secno}[BBSec]

PSTricks creates dimensionless graphics, but \jTree\ goes to a
lot of trouble to figure out the sizes of the trees that it
generates and to put them in appropriately sized boxes.  For
example:

\exh \tac|
\jtree
\! = {X} :{a} :{a} :{a}
   {\multiline
      this and\cr
      that\endmultiline}.
\endjtree
|endtac \hfil
\psframebox[framesep=0,linestyle=dashed]{\jtree
\! = {X} :{a} :{a} :{a}
   {\multiline
      this and\cr
      that\endmultiline}.
\endjtree}
\xe

The sizing is not perfect.  \jTree\ is not clever enough to
recognize the white space due to |labelgapt|, |labelgapb|,
|labelstrutt|, and |labelstrutb|.  But it is not bad.

\medskip
If PSTricks is used to draw node connections, they often extend outside the
\jTree\ bounding box.

\exh \tac|
\jtree
\! = {X}@A1
   <right>
   :{a}
   :{a}
      {\multiline
         this and\cr
         that\endmultiline}@A2 .
\nccurve[angleA=210,angleB=200,
   ncurv=2,nodesepA=0]{->}{A1:b}{A2}
\endjtree
|endtac \hfil
\psframebox[framesep=0,linestyle=dashed]{\jtree
\! = {X}@A1
   <right>
   :{a}
   :{a}
      {\multiline
         this and\cr
         that\endmultiline}@A2 .
\nccurve[angleA=210,angleB=200,
   ncurv=2,nodesepA=0]{->}{A1:b}{A2}
\endjtree}
\xe

There are two ways to fix this.  The first is elementary, simply
inserting kerning as appropriate.

\exh \tac|
\hbox{\kern2.4em\jtree
\! = {X}@A1
   <right>
   :{a}
   :{a}
      {\multiline
         this and\cr
         that\endmultiline}@A2 .
\nccurve[angleA=210,angleB=200,
   ncurv=2,nodesepA=0]{->}{A1:b}{A2}
\endjtree}
|endtac \hfil
\psframebox[framesep=0,linestyle=dashed]{\kern2.4em\jtree
\! = {X}@A1
   <right>
   :{a}
   :{a}
      {\multiline
         this and\cr
         that\endmultiline}@A2 .
\nccurve[angleA=210,angleB=200,
   ncurv=2,nodesepA=0]{->}{A1:b}{A2}
\endjtree}
\xe

jTree also provides the two parameters which can be used to
fine-tune the extent of the bounding box, if such adjustment is
needed in some application.  First, the boolean parameter
|showbb|\index*{showbb} can be set to |true| to show the bounding
box. It has the XKV default value |true|, so |\psset{showbb}|
with no value is sufficient to set it to |true|.  It is always
set to |false| when |\endjtree| is executed, so the scope of a
|true| setting does not extend past the first |\endjtree|
encountered. The parameter |bbadjust|\index*{bbadjust} can be
used to modify the size of the bounding box from the size which
|\jtree| naturally computes.  Setting |bbadjust=height 1ex depth
2ex left 1em right .5em|, for example, alters the size of the box
as expected.  The specifications of the height, depth, left
(side), right (side) adjustments are optional, provided that
there is at least one specified.  |showbb| and |bbajust| are
illustrated below.

\exh
\tac|
\jtree[showbb]
\! = :{A}@A1 ({[Q]}[labelgap=-.5ex])
   :{B} {C}@C1 .
\nccurve[angleA=200,angleB=-140,
   ncurv=1.4,arrows=->]{A1}{C1}
\endjtree
|endtac \hfil
\jtree[showbb]
\! = :{A}@A1 ({[Q]}[labelgap=-.5ex]) :{B} {C}@C1 .
\nccurve[angleA=200,angleB=-140,ncurv=1.4]{->}{A1}{C1}
\endjtree
\bigskip
\tac|
\jtree[showbb,
   bbadjust=left 1.2em depth 2.3ex]
\! = :{A}@A1 ({[Q]}[labelgap=-.5ex])
   :{B} {C}@C1 .
\nccurve[angleA=200,angleB=-140,
   ncurv=1.4,arrows=->]{A1}{C1}
\endjtree
|endtac \hfil
\jtree[showbb,
   bbadjust=left 1.2em depth 2.3ex]
\! = :{A}@A1 ({[Q]}[labelgap=-.5ex])
   :{B} {C}@C1 .
\nccurve[angleA=200,angleB=-140,
   ncurv=1.4,arrows=->]{A1}{C1}
\endjtree
\xe

