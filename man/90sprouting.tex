
\example

\ftag{\the\Exno}[sprout]

The following display is from Christos Vlachos, modeled
on Chung, Ladusaw \& McCloskey 1995: 247,
(16).  (His code was slightly modified in order to illustrate a few
points of general interest.)
\bigskip

%\
%\jtree[xunit=3.5em,yunit=1em,elcxoffset=.5ex,
%   elcyoffset=.5ex,bbadjust=height 1ex depth 4ex]
%\defbranch<wideright>(1)(-2/3)
%\defbranch<shortright>(.5)(-2/3)
%\defbranch<steepright>(.7)(-3/2)
%\defbranch<steepleft>(.7)(3/2)
%\! = {CP}
%<left>{\rnode{A}{PP$_i$}}!a ^<wideright>{C$'$}
%<left>{C$^0$}!b ^<wideright>{\elc{IP}}
%<left>{DP}(<vartri>{\rnode{joan}{Joan}})
%   ^<shortright>{\elc{I$'$}}
%<left>{I$^0$} ^<shortright>{\rnode{vp}{VP}}
%<left>[scaleby=1.3]{VP}!c
%   ^<wideright>[scaleby=.7]{\rnode{B}{PP$_i$}}.
%\!a = <steepleft>{P}(<vert>{with})
%   ^<steepright>{DP} <vartri>{whom$\fam0 ^x$}.
%\!b = {[+Q]}[labelgapt=-2pt]<vert>{e\rlap{$\fam0 ^x$}}.
%\!c = <steepleft>{DP}
%   ^<steepright>{\elc{V$'$}}
%   <steepleft>{V$^0$}(<shortvert>{ate})
%   ^<steepright>{DP}<vartri>{dinner}.
%\nccurve[angleA=-120,angleB=-95,ncurvA=.7,ncurvB=1.2,
%   nodesepA=0,nodesepB=.5ex,arrows=<->]{A}{B}
%\ncput*[npos=.4]{\it merger}
%\def\radiusa{8.8}%
%\def\pointa{\radiusa;30}
%\def\radiusb{3.5}%
%\def\pointb{\radiusb;40}%
%\psLNode(B)(joan){.5}{centera}
%\rput(centera){\pscircle{\radiusa}
%   \rput[l](\pointa){$\,\longleftarrow$ copying}}
%\psLNode(vp)(B){.5}{centerb}
%\rput(centerb){\pscircle{\radiusb}
%   \rput[l](\pointb){$\,\longleftarrow$ sprouting}}
%\endjtree

\
\jtree[xunit=3.5em,yunit=1em,elcxoffset=.5ex,
   elcyoffset=.5ex,bbadjust=height 1ex depth 4ex]
\defbranch<wideright>(1)(-2/3)
\defbranch<shortright>(.5)(-2/3)
\defbranch<steepright>(.7)(-3/2)
\defbranch<steepleft>(.7)(3/2)
\! = {CP}
<left>{\rnode{A}{PP$_i$}}!a ^<wideright>{C$'$}
<left>{C$^0$}!b ^<wideright>{\elc{IP}}
<left>{DP}(<vartri>{\rnode{joan}{Joan}})
   ^<shortright>{\elc{I$'$}}
<left>{I$^0$} ^<shortright>{\rnode{vp}{VP}}
<left>[scaleby=1.3]{VP}!c
   ^<wideright>[scaleby=.7]{\rnode{B}{PP$_i$}}.
\!a = <steepleft>{P}(<vert>{with})
   ^<steepright>{DP} <vartri>{whom$\fam0 ^x$}.
\!b = {[+Q]}[labelgapt=-.6ex]<vert>{e\rlap{$\fam0 ^x$}}.
\!c = <steepleft>{DP}
   ^<steepright>{\elc{V$'$}}
   <steepleft>{V$^0$}(<shortvert>{ate})
   ^<steepright>{DP}<vartri>{dinner}.
\nccurve[angleA=-120,angleB=-95,ncurvA=.7,ncurvB=1.2,
   nodesepA=0,nodesepB=.5ex,arrows=<->]{A}{B}
\ncput*[npos=.4]{\it merger}
\psLNode(B)(joan){.5}{centerA}
\psLNode(vp)(B){.5}{centerB}
\rput(centerA){\pscircle{8.8}
   \rput[l](8.8;30){$\,\longleftarrow$ copying}}
\rput(centerB){\pscircle{3.5}
   \rput[l](3.5;40){$\,\longleftarrow$ sprouting}}
\endjtree


\bigskip

\CLnum
\jtree[xunit=3.5em,yunit=1em,elcxoffset=.5ex,
   elcyoffset=.5ex,bbadjust=height 1ex depth 4ex]
\defbranch<wideright>(1)(-2/3)
\defbranch<shortright>(.5)(-2/3)
\defbranch<steepright>(.7)(-3/2)
\defbranch<steepleft>(.7)(3/2)
\! = {CP}
<left>{\rnode{A}{PP$_i$}}!a ^<wideright>{C$'$}
<left>{C$^0$}!b ^<wideright>{\elc{IP}}
<left>{DP}(<vartri>{\rnode{joan}{Joan}})
   ^<shortright>{\elc{I$'$}}
<left>{I$^0$} ^<shortright>{\rnode{vp}{VP}}
<left>[scaleby=1.3]{VP}!c
   ^<wideright>[scaleby=.7]{\rnode{B}{PP$_i$}}.
\!a = <steepleft>{P}(<vert>{with})
   ^<steepright>{DP} <vartri>{whom$\fam0 ^x$}.
\!b = {[+Q]}[labelgapt=-.6ex]<vert>{e\rlap{$\fam0 ^x$}}.
\!c = <steepleft>{DP}
   ^<steepright>{\elc{V$'$}}
   <steepleft>{V$^0$}(<shortvert>{ate})
   ^<steepright>{DP}<vartri>{dinner}.
\nccurve[angleA=-120,angleB=-95,ncurvA=.7,ncurvB=1.2,
   nodesepA=0,nodesepB=.5ex,arrows=<->]{A}{B}
\ncput*[npos=.4]{\it merger}
\psLNode(B)(joan){.5}{centerA}
\psLNode(vp)(B){.5}{centerB}
\rput(centerA){\pscircle{8.8}
   \rput[l](8.8;30){$\,\longleftarrow$ copying}}
\rput(centerB){\pscircle{3.5}
   \rput[l](3.5;40){$\,\longleftarrow$ sprouting}}
\endjtree|endCLnum  %|
\medskip

1. |\fam0| is used on lines 16 and 17 so that letters are taken
from the standard font set in math mode rather than the math
italic font set.
\medskip
2. |\ncput| is used on line 24 to label the curve drawn by
|\nccurve|.  PSTricks has 3 commands of this sort for labeling
node connections: |\ncput| puts the label directly on the
connection, |\naput| puts it above the connection, and |\nbput|
puts it below the connection.  There is a parameter |labelsep|
which controls how far above or below the connection the label is
put; a parameter |nrot| which can be used to rotate the label;
and a parameter |npos| which determines where on the connection
the label is put.  In this case, the label is put $.4$ of the way
from the starting label to the finishing label.
See the PSTricks User's Guide for all the details.
\medskip
2. Lines 25--31 are devoted to drawing the circles and
positioning the labels ``copying'' and ``sprouting''.  The
strategy is first to locate the centers of the two circles by
interpolating between some of the nodes which must be circles.
That is done in lines 25 and 26.  It is most convenient to then
draw circles centered at the origin, complete with labels, then
put them at the centers which have been computed.  The labels are
positioned by using PSTricks to locate points using polar
coordinates.  |(8.8;40)| on line 28, for example, is the point
such that the line joining the point and the origin is 8.8 units
long and makes an angle of $40\,\rm degrees$ with the positive
horizontal direction. It is on the circle which is drawn by
|\pscircle{8.8}|.


